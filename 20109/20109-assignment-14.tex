% !TEX program = xelatex
\documentclass{article}
\usepackage[]{amsthm} %lets us use \begin{proof}
\usepackage{amsmath}
\usepackage{enumerate}
\usepackage{xparse}
\usepackage[makeroom]{cancel}
\usepackage[]{amssymb} %gives us the character \varnothing
\usepackage{fontspec}
\usepackage{polyglossia}
\usepackage{relsize}
\usepackage{graphicx}
\usepackage[left=2.0cm, top=2.0cm, right=2.0cm, bottom=2.0cm]{geometry}

\setdefaultlanguage{hebrew}
\setotherlanguage{english}

\setmainfont{[Arial.ttf]}
\newfontfamily\hebrewfont{[Arial.ttf]}

\newcommand\niton{\not\mathrel{\text{\reflectbox{$\in$}}}}
\newcommand\underrel[2]{\mathrel{\mathop{#2}\limits_{#1}}}
\DeclareMathOperator*{\equals}{=}
\DeclareMathOperator\Sp{Sp}
\def\reals{\mathbb{R}}
\def\zerovec{\underline{0}}

\title{מטלת מנחה 14 - אלגברה לינארית 1}
\author{328197462}
\date{15/01/2023}

\begin{document}
\maketitle

\section*{שאלה 1}

יהיו $U, W_1, W_2$ תתי-מרחבים לינאריים של מרחב לינארי $V$.

\subsection*{סעיף א}

יהא $v\in(U\cap W_1)+(U\cap W_2)$
ועלינו להוכיח $v\in U\cap (W_1+W_2)$.
\\
מהגדרת החיבור, קיימים $v_1\in U\cap W_1, v_2\in U\cap W_2$ כך ש $v=v_1+v_2$.\\
אי לכך, $v_1, v_2\in U$ ומסגירות החיבור הוקטורי במרחב הלינארי נסיק $v=v_1+v_2\in U$. \\
כמו כן, מאחר ו $v_1\in W_1, v_2\in W_2$ נקבל מהגדרת החיבור כי $v=v_1+v_2\in W_1+W_2$. \\
הראינו שייכות לשתי הקבוצות $U, W_1+W_2$ ולכן נסיק $v\in U\cap (W_1+W_2)$

\subsection*{סעיף ב}

עבור $V=\reals^2$ נגדיר:
\[
    \begin{matrix}
        U=\Sp(\{ (1,1) \})   &
        W_1=\Sp(\{ (1,0) \}) &
        W_2=\Sp(\{ (0,1) \})
    \end{matrix}
\]
אז לפי סעיף א של שאלה זו מתקיים $(U\cap W_1)+(U\cap W_2)\subseteq U\cap (W_1+W_2)$. \\\\
ניקח $v=(1,1)$ ונראה כי $v\in U\cap (W_1+W_2)$ וגם $v\notin (U\cap W_1)+(U\cap W_2)$.\\
נחשב:
\begin{align*}
    U\cap (W_1+W_2)
     & =\Sp(\{ (1,1) \}) \cap (\Sp(\{ (1,0) \}) + \Sp(\{ (0,1) \})) \equals_{\text{שאלה 7.6.8}} \\
     & =\Sp(\{ (1,1) \}) \cap (\Sp(\{ (1,0), (0,1) \}))=                                        \\
     & =\Sp(\{ (1,1) \}) \cap \reals^2 =                                                        \\
     & =\Sp(\{ (1,1) \})\ni (1,1)=v
\end{align*}
\begin{align*}
    (U\cap W_1)+(U\cap W_2)
     & =(\Sp(\{ (1,1) \})\cap \Sp(\{ (1,0) \}))+(\Sp(\{ (1,1) \}) + \Sp(\{ (0,1) \}))= \\
     & =\{\zerovec\} + \{\zerovec\} =                                                  \\
     & = \{\zerovec\}\niton (1,1)=v
\end{align*}

ולכן מתקיימת הכלה חזקה בין תתי-המרחבים.

\pagebreak

\section*{שאלה 2}

\end{document}