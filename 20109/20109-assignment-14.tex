% !TEX program = xelatex
\documentclass{article}
\usepackage[]{amsthm} %lets us use \begin{proof}
\usepackage{amsmath}
\usepackage{enumerate}
\usepackage{xparse}
\usepackage[makeroom]{cancel}
\usepackage[]{amssymb} %gives us the character \varnothing
\usepackage{fontspec}
\usepackage{polyglossia}
\usepackage{relsize}
\usepackage{graphicx}
\usepackage[left=2.0cm, top=2.0cm, right=2.0cm, bottom=2.0cm]{geometry}

\setdefaultlanguage{hebrew}
\setotherlanguage{english}

\setmainfont{[Arial.ttf]}
\newfontfamily\hebrewfont{[Arial.ttf]}

\newcommand\niton{\not\mathrel{\text{\reflectbox{$\in$}}}}
\newcommand\underrel[2]{\mathrel{\mathop{#2}\limits_{#1}}}
\DeclareMathOperator*{\equals}{=}
\DeclareMathOperator\Sp{Sp}
\def\reals{\mathbb{R}}
\def\zerovec{\underline{0}}

\title{מטלת מנחה 14 - אלגברה לינארית 1}
\author{328197462}
\date{15/01/2023}

\begin{document}
\maketitle

\section*{שאלה 1}

יהיו $U, W_1, W_2$ תתי-מרחבים לינאריים של מרחב לינארי $V$.

\subsection*{סעיף א}

יהא $v\in(U\cap W_1)+(U\cap W_2)$
ועלינו להוכיח $v\in U\cap (W_1+W_2)$.
\\
מהגדרת החיבור, קיימים $v_1\in U\cap W_1, v_2\in U\cap W_2$ כך ש $v=v_1+v_2$.\\
אי לכך, $v_1, v_2\in U$ ומסגירות החיבור הוקטורי במרחב הלינארי נסיק $v=v_1+v_2\in U$. \\
כמו כן, מאחר ו $v_1\in W_1, v_2\in W_2$ נקבל מהגדרת החיבור כי $v=v_1+v_2\in W_1+W_2$. \\
הראינו שייכות לשתי הקבוצות $U, W_1+W_2$ ולכן נסיק $v\in U\cap (W_1+W_2)$

\subsection*{סעיף ב}

עבור $V=\reals^2$ נגדיר:
\[
    \begin{matrix}
        U=\Sp(\{ (1,1) \})   &
        W_1=\Sp(\{ (1,0) \}) &
        W_2=\Sp(\{ (0,1) \})
    \end{matrix}
\]
אז לפי סעיף א של שאלה זו מתקיים $(U\cap W_1)+(U\cap W_2)\subseteq U\cap (W_1+W_2)$. \\\\
ניקח $v=(1,1)$ ונראה כי $v\in U\cap (W_1+W_2)$ וגם $v\notin (U\cap W_1)+(U\cap W_2)$.\\
נחשב:
\begin{align*}
    U\cap (W_1+W_2)
     & =\Sp(\{ (1,1) \}) \cap (\Sp(\{ (1,0) \}) + \Sp(\{ (0,1) \})) \equals_{\text{שאלה 7.6.8}} \\
     & =\Sp(\{ (1,1) \}) \cap (\Sp(\{ (1,0), (0,1) \}))=                                        \\
     & =\Sp(\{ (1,1) \}) \cap \reals^2 =                                                        \\
     & =\Sp(\{ (1,1) \})\ni (1,1)=v
\end{align*}
\begin{align*}
    (U\cap W_1)+(U\cap W_2)
     & =(\Sp(\{ (1,1) \})\cap \Sp(\{ (1,0) \}))+(\Sp(\{ (1,1) \}) + \Sp(\{ (0,1) \}))= \\
     & =\{\zerovec\} + \{\zerovec\} =                                                  \\
     & = \{\zerovec\}\niton (1,1)=v
\end{align*}

ולכן מתקיימת הכלה חזקה בין תתי-המרחבים.

\pagebreak

\section*{שאלה 2}

יהיו $U=\Sp\{u_1, u_2\}$, $W=\Sp\{ w_1, w_2\}$ תתי-מרחבים לינאריים של $V$ כך שהקבוצות הפורשות אותם הן בסיסים.\\
מניחים כי $A=\{ u_1, u_2, w_1 \}$ תלויה לינארית.

\subsection*{סעיף א}

נראה כי $w_1\in U$ בדרך השלילה. \\
נניח בשלילה כי $w_1\notin \Sp\{u_1, u_2\}$.
מאחר והקבוצה $\{ u_1, u_2 \}$ היא בסיס ולכן בלתי תלויה לינארית,
נסיק לפי שאלה 8.1.8 כי $\{ u_1, u_2 \}\cup \{ w_1 \}=A$ בלתי תלויה לינארית, בסתירה לנתון!
\\\\
כעת, מאחר ו$w_1=1\cdot w_1+0\cdot w_2\in \Sp\{ w_1, w_2\}=W$, נקבל $w_1\in U\cap W$.

\subsection*{סעיף ב}

ניזכר במשפט המימדים 8.3.6
\[
    \dim(U+W)=\dim U+\dim W-\dim(U\cap W)
\]
לשני תתי-המרחבים $U,W$ יש בסיסים בגודל 2 ומכאן $\dim U=\dim W=2$. עלינו למצוא את מימד תת-המרחב $U\cap W$.
\\\\
לפי משפט $3.8.4$, עבור $U\cap W\subseteq U, W$ נסיק $\dim(U\cap W)\leq 2$. \\
בנוסף, אם $\dim(U\cap W)=2$, אז נסיק את השוויון $U\cap W=U=W$ בסתירה לנתון כי $U,W$ תתי-מרחבים שונים. \\
מכאן נובע אי-השוויון $\dim(U\cap W)\leq 1$. מאחר ו $w_1\ne \zerovec\in U\cap W$ (הוקטור נמצא בקבוצה בלתי תלויה לינארית),
נסיק $\dim(U\cap W)\geq 1$ ובסך הכל $\dim(U\cap W)=1$.
\\\\
נציב במשפט המימדים ונקבל $\dim(U+W)=\dim U + \dim W - \dim(U\cap W)=2+2-1=3$.
\\\\
הקבוצה $\{u_1, u_2, w_2\}$ בעלת 3 וקטורים ומוכלת ב $U+W$. נראה כי הקבוצה בת"ל, כלומר $w_2\notin U$. \\
נניח כי $w_2\in U$. מסעיף א של שאלה זו נקבל $\{ w_1, w_2\}\subseteq \Sp\{u_1, u_2\}$, ולפי שאלה $7.5.16$ נסיק:
\[
    W=\Sp\{ w_1, w_2\}\subseteq \Sp\{u_1, u_2\}=U
\]
משוויון המימדים נובע, לפי משפט $3.8.4$, כי $U=W$ וזאת בסתירה לנתון! מצאנו $w_2\notin U$ ולכן לפי שאלה 8.1.8 הקבוצה בת"ל.\\\\
מצאנו כי $\{u_1, u_2, w_2\}$ בת"ל ובעלת 3 וקטורים ולכן קבוצה היא בסיס ל$U+W$.

\pagebreak

\section*{שאלה 3}

יהיו תתי המרחבים הבאים של $V=\reals_4[x]$
\begin{align*}
    \begin{matrix}
        U=\Sp\{ u_1=x^3+4x^2-x+3, & u_2=x^3+5x^2+5,   & u_3=3x^3+10x^2+5    \} \\
        W=\Sp\{ w_1=x^3+4x^2+6,   & w_2=x^3+2x^2-x+5, & w_3=2x^3+2x^2-3x+9  \}
    \end{matrix}
\end{align*}
נסמן בשאלה את הבסיס הסטנדרטי הסדור של $V$ ב $E=(x^3, x^2, x, 1)$.

\subsection*{בסיס ל$U$}
תחילה, וקטורי הקואורדינטות של הקבוצה הפורשת הנתונה של $U$, לפי הבסיס הסטנדרטי, הם:
\begin{align*}
    [u_1]_E=\begin{pmatrix}
        1  \\
        4  \\
        -1 \\
        3
    \end{pmatrix} & , &
    [u_2]_E=\begin{pmatrix}
        1 \\
        5 \\
        0 \\
        5
    \end{pmatrix} & , &
    [u_3]_E=\begin{pmatrix}
        3  \\
        10 \\
        0  \\
        5
    \end{pmatrix}
\end{align*}
תת-המרחב $U'=\Sp\{ [u_1]_E, [u_2]_E, [u_3]_E \}$ הוא תת-מרחב של $\mathbb{F}^n$. נמצא לו בסיס. \\
לשם כך, נכתוב את הוקטורים כשורות במטריצה ונדרג:
\begin{align*}
    \begin{pmatrix}
        1 & 4  & -1 & 3 \\
        1 & 5  & 0  & 5 \\
        3 & 10 & 0  & 5
    \end{pmatrix}
    \xrightarrow[R_3\rightarrow R_3-3R_1]{R_2\rightarrow R_2-R_1}
    \begin{pmatrix}
        1 & 4  & -1 & 3  \\
        0 & 1  & 1  & 2  \\
        0 & -2 & 3  & -4
    \end{pmatrix}
    \xrightarrow[]{R_3\rightarrow R_3+2R_2}
    \begin{pmatrix}
        1 & 4 & -1 & 3 \\
        0 & 1 & 1  & 2 \\
        0 & 0 & 5  & 0
    \end{pmatrix}
    \xrightarrow[]{R_3\rightarrow \frac{1}{5}R_3}
    \begin{pmatrix}
        1 & 4 & -1 & 3 \\
        0 & 1 & 1  & 2 \\
        0 & 0 & 1  & 0
    \end{pmatrix}
\end{align*}
לפי שאלה $7.5.12$, מרחב השורות של המטריצה המדורגת הינו גם $U'$. כמו כן, שורות המטריצה המדורגת אינן שורות אפס ולכן לפי למה $8.5.1$ בת"ל. \\
קיבלנו כי הקבוצה הבאה בת"ל ופורשת את $\Sp\{ [u_1]_E, [u_2]_E, [u_3]_E \}$, ולכן בסיס לה:
\[
    B'=\left\{
    \begin{pmatrix}
        1  \\
        4  \\
        -1 \\
        3
    \end{pmatrix},
    \begin{pmatrix}
        0 \\
        1 \\
        1 \\
        2
    \end{pmatrix},
    \begin{pmatrix}
        0 \\
        0 \\
        1 \\
        0
    \end{pmatrix}
    \right\}
\]
וקטורים אלה הם וקטורי הקואורדינטות לפי $E$ של איברי הקבוצה $B=\{b_1=x^3+4x^2-x+3, b_2=x^2+x+2, b_3=x\}$.\\
כעת, לפי טענה $8.4.12$, מאחר ו$B'$ בסיס ל$U'$ נסיק כי $B$ בסיס ל$U$,
וכן כי $\dim U=3$

\subsection*{בסיס ל$W$}

נשתמש בתהליך זהה. וקטורי הקואורדינטות של הקבוצה הפורשת הנתונה ל$W$:
\begin{align*}
    [w_1]_E=\begin{pmatrix}
        1 \\
        4 \\
        0 \\
        6
    \end{pmatrix} & , &
    [w_2]_E=\begin{pmatrix}
        1  \\
        2  \\
        -1 \\
        5
    \end{pmatrix} & , &
    [w_3]_E=\begin{pmatrix}
        2  \\
        2  \\
        -3 \\
        9
    \end{pmatrix}
\end{align*}
תת-המרחב $W'=\Sp\{ [u_1]_E, [u_2]_E, [u_3]_E \}$ הוא תת-מרחב של $\mathbb{F}^n$. נמצא לו בסיס. \\
לשם כך, נכתוב את הוקטורים כשורות במטריצה ונדרג:
\begin{align*}
    \begin{pmatrix}
        1 & 4 & 0  & 6 \\
        1 & 2 & -1 & 5 \\
        2 & 2 & -3 & 9
    \end{pmatrix}
    \xrightarrow[R_3\rightarrow R_3-2R_1]{R_2\rightarrow R_2-R_1}
    \begin{pmatrix}
        1 & 4  & 0  & 6  \\
        0 & -2 & -1 & -1 \\
        0 & -6 & -3 & -3
    \end{pmatrix}
    \xrightarrow[]{R_3\rightarrow R_3-3R_2}
    \begin{pmatrix}
        1 & 4  & 0  & 6  \\
        0 & -2 & -1 & -1 \\
        0 & 0  & 0  & 0
    \end{pmatrix}
    \xrightarrow[]{R_3\rightarrow -R_3}
    \begin{pmatrix}
        1 & 4 & 0 & 6 \\
        0 & 2 & 1 & 1 \\
        0 & 0 & 0 & 0
    \end{pmatrix}
\end{align*}
לפי שאלה $7.5.12$ ולמה $8.5.1$, הקבוצה הבאה בת"ל ופורשת את $W'$ ולכן מהווה בסיס.
\[
    C'=\left\{
    \begin{pmatrix}
        1 \\
        4 \\
        0 \\
        6
    \end{pmatrix},
    \begin{pmatrix}
        0 \\
        2 \\
        1 \\
        1
    \end{pmatrix}
    \right\}
\]
וקטורי הקבוצה הם וקטורים הקואורדינטות לפי $E$ של $C=\{ c_1=x^3+4x^2+6, c_2=2x^2+x+1 \}$.\\
אי לכך, לפי טענה $8.4.12$, מאחר ו$C'$ בסיס ל$W'$ נסיק כי $C$ בסיס ל$W$,
וכן באופן ישיר $\dim W=2$

\subsection*{בסיס ל$U+W$}

היות ו$U=\Sp(B), W=\Sp(C)$, נסיק לפי שאלה $7.6.8$ כי
\[
    U+W=\Sp(B\cup C)=\Sp\{x^3+4x^2-x+3, x^2+x+2, x, x^3+4x^2+6, 2x^2+x+1\}
\]
באופן דומה,
\[
    U'+W'=\Sp\{ (1, 4, -1, 3), (0, 1, 1, 2), (0, 0, 1, 0), (1, 4, 0, 6), (0, 2, 1, 1) \}
\]
נמצא בסיס ל$U'+W'$. לשם כך נחזור על התהליך מהחלקים הקודמים של השאלה:
\begin{align*}
    \begin{pmatrix}
        1 & 4 & -1 & 3 \\
        0 & 1 & 1  & 2 \\
        0 & 0 & 1  & 0 \\
        1 & 4 & 0  & 6 \\
        0 & 2 & 1  & 1 \\
    \end{pmatrix}
    \xrightarrow[R_5\rightarrow R_5-2R_2]{R_4\rightarrow R_4-R_1}
    \begin{pmatrix}
        1 & 4 & -1 & 3  \\
        0 & 1 & 1  & 2  \\
        0 & 0 & 1  & 0  \\
        0 & 0 & 1  & 3  \\
        0 & 0 & -1 & -5 \\
    \end{pmatrix}
    \xrightarrow[R_5\rightarrow R_5+R_3]{R_4\rightarrow R_4-R_3}
    \begin{pmatrix}
        1 & 4 & -1 & 3  \\
        0 & 1 & 1  & 2  \\
        0 & 0 & 1  & 0  \\
        0 & 0 & 0  & 3  \\
        0 & 0 & 0  & -5 \\
    \end{pmatrix} \\
    \xrightarrow[]{R_4\rightarrow \frac{1}{3}R_4}
    \begin{pmatrix}
        1 & 4 & -1 & 3  \\
        0 & 1 & 1  & 2  \\
        0 & 0 & 1  & 0  \\
        0 & 0 & 0  & 1  \\
        0 & 0 & 0  & -5 \\
    \end{pmatrix}
    \xrightarrow[]{R_5\rightarrow R_5+5R_4}
    \begin{pmatrix}
        1 & 4 & -1 & 3 \\
        0 & 1 & 1  & 2 \\
        0 & 0 & 1  & 0 \\
        0 & 0 & 0  & 1 \\
        0 & 0 & 0  & 0 \\
    \end{pmatrix}
\end{align*}
לפי שאלה 7.5.12 ולמה 8.5.1 4 השורות הראשונות של המטריצה המדורגת בת"ל ופורשות את $U'+W'$. \\
שוב, לפי טענה $8.4.12$, 4 הפולינומים שוקטורי הקואורדינטות שלהם הם 4 שורות המטריצה מהווים בסיס ל$U+W$.
נקבל $\dim(U+W)=4$, והיות ו$U+W\subseteq \reals_4[x]$ נקבל ממשפט $8.3.4$ כי $U+V=\reals_4[x]$. \\
ניקח את הבסיס הסטנדרטי למרחב לינארי זה - E שהוגדר בתחילת השאלה.

\subsection*{סעיף ב}

ראשית, על מנת למצוא את המימד של $U\cap W$, ניעזר במשפט המימדים $8.3.6$:
\[
    \dim(U+W)=\dim U + \dim W -\dim(U\cap W)
\]
נציב ונקבל $4=2+2-\dim(U\cap W)$ ומכאן $\dim (U\cap W)=1$.\\\\
כעת, יהא $p(x)\in \reals_4[x]$. על מנת ש$p(x)$ יהיה שייך לשני תתי-המרחבים הלינאריים $U,W$,
נדרוש שיהיו קיימים $\lambda_1, \lambda_2, \lambda_3, \mu_1, \mu_2$ סקלרים כך ש:
\begin{align*}
    \lambda_1b_1+\lambda_2b_2+\lambda_3b_3=\mu_1c_1+\mu_2c_2   \\
    \lambda_1b_1+\lambda_2b_2+\lambda_3b_3-\mu_1c_1-\mu_2c_2=0 \\
    [\lambda_1b_1+\lambda_2b_2+\lambda_3b_3-\mu_1c_1-\mu_2c_2]_E=[0]_E
\end{align*}
ומלמה $8.4.3$ ושאלה $8.4.5$ נקבל:
\begin{align*}
    \lambda_1[b_1]_E+\lambda_2[b_2]_E+\lambda_3[b_3]_E-\mu_1[c_1]_E-\mu_2[c_2]_E=0 \\
    \lambda_1 \begin{pmatrix}
        1  \\
        4  \\
        -1 \\
        3
    \end{pmatrix}+
    \lambda_2\begin{pmatrix}
        0 \\
        1 \\
        1 \\
        2
    \end{pmatrix}+
    \lambda_3\begin{pmatrix}
        0 \\
        0 \\
        1 \\
        0
    \end{pmatrix}-
    \mu_1\begin{pmatrix}
        1 \\
        4 \\
        0 \\
        6
    \end{pmatrix}-
    \mu_2\begin{pmatrix}
        0 \\
        2 \\
        1 \\
        1
    \end{pmatrix}=0
\end{align*}
\[
    \begin{pmatrix}
        1  & 0 & 0 & -1 & 0  \\
        4  & 1 & 0 & -4 & -2 \\
        -1 & 1 & 1 & 0  & -1 \\
        3  & 2 & 0 & -6 & -1
    \end{pmatrix}
    \cdot
    \begin{pmatrix}
        \lambda_1 \\
        \lambda_2 \\
        \lambda_3 \\
        \mu_1     \\
        \mu_2
    \end{pmatrix}=0
\]

\pagebreak
נדרג את המטריצה על מנת לפתור את המערכת:
\begin{align*}
    \begin{pmatrix}
        1  & 0 & 0 & -1 & 0  \\
        4  & 1 & 0 & -4 & -2 \\
        -1 & 1 & 1 & 0  & -1 \\
        3  & 2 & 0 & -6 & -1
    \end{pmatrix}
    \xrightarrow[R_4\rightarrow R_4-3R_1]{R_2\rightarrow R_2-4R_1 \; R_3\rightarrow R_3+R_1}
    \begin{pmatrix}
        1 & 0 & 0 & -1 & 0  \\
        0 & 1 & 0 & 0  & -2 \\
        0 & 1 & 1 & -1 & -1 \\
        0 & 2 & 0 & -3 & -1
    \end{pmatrix}
    \xrightarrow[R_4\rightarrow R_4-2R_2]{R_3\rightarrow R_3-R_2}
    \begin{pmatrix}
        1 & 0 & 0 & -1 & 0  \\
        0 & 1 & 0 & 0  & -2 \\
        0 & 0 & 1 & -1 & 1  \\
        0 & 0 & 0 & -3 & 3
    \end{pmatrix} \\
    \xrightarrow[]{R_4\rightarrow -\frac{1}{3}R_4}
    \begin{pmatrix}
        1 & 0 & 0 & -1 & 0  \\
        0 & 1 & 0 & 0  & -2 \\
        0 & 0 & 1 & -1 & 1  \\
        0 & 0 & 0 & 1  & -1
    \end{pmatrix}
    \xrightarrow[R_3\rightarrow R_3+R_4]{R_1\rightarrow R_1+R_4}
    \begin{pmatrix}
        1 & 0 & 0 & 0 & -1 \\
        0 & 1 & 0 & 0 & -2 \\
        0 & 0 & 1 & 0 & 0  \\
        0 & 0 & 0 & 1 & -1
    \end{pmatrix}
    \rightarrow
    \begin{cases}
        \lambda_1-\mu_2=0  \\
        \lambda_2-2\mu_2=0 \\
        \lambda_3=0        \\
        \mu_1-\mu_2=0      \\
    \end{cases}
\end{align*}
קיבלנו כי $\mu_2$ משתנה חופשי. ניקח סקלר $a$ כלשהו כך ש $\mu_2=a$, אז $\mu_1=a$ ונקבל:
\begin{align*}
    p(x) & =\mu_1c_1+\mu_2c_2=          \\
         & =a(x^3+4x^2+6) +a(2x^2+x+1)= \\
         & = a(x^3+6x^2+x+7)
\end{align*}
במילים אחרות, $U\cap V=\Sp\{ x^3+6x^2+x+7 \}$ והקבוצה $\{ x^3+6x^2+x+7 \}$ בסיס ל$U\cap W$

\subsection*{סעיף ג}

לפי משפט $8.3.5$, הקבוצה הבלתי-תלויה לינארית $C$ של וקטורים מ$\reals_4[x]$ ניתנת להשלמה לבסיס. \\
כלומר, קיימים $c_3, c_4\in V$ כך ש $C\cup \{c_3, c_4\}$ בסיס ל$\reals_4[x]$. \\
ניקח $T=\Sp\{ c_3, c_4 \}$. אז לפי שאלה 7.6.8 $W+T=\Sp(C\cup \{c_3, c_4\})=\reals_4[x]$. \\
כמו כן $T$ נפרשת על ידי $2$ וקטורים בת"ל (מעצם הגדרתם כבסיס ל$\reals_4[x]$) ולכן $\dim T=2$ ולפי מסקנה $8.3.7$ מתקבל $W\oplus T=\reals_4[x]$. \\
לסיכום, תת-המרחב הנפרש על ידי שני וקטורים $c_3, c_4$ כאלה מהווה קבוצה $T$ מתאימה.
\\\\
נמצא וקטורים אלה. על מנת ש $\{ c_1, c_2, c_3, c_4 \}$יהיה בסיס ל $\reals_4[x]$ נדרוש כי $\{ [c_1]_E, [c_2]_E, [c_3]_E, [c_4]_E \}$ יהיה בסיס ל $\mathbb{F}^4$ לפי $8.4.12$. \\
תנאי הכרחי ומספיק לכך שהקבוצה בת 4 וקטורים תהווה בסיס ל$F^n$ הוא היות קבוצת הוקטורים בלתי תלויה לינארית. \\
נכתוב את ארבעת הוקטורים כשורות במטריצה:
\[
    \begin{pmatrix}
        1 & 4 & 0 & 6 \\
        0 & 2 & 1 & 1 \\
        ? & ? & ? & ? \\
        ? & ? & ? & ?
    \end{pmatrix}
\]
נבחר למשל $[c_3]_E=(0,0,1,0), [c_4]_E=(0,0,0,1)$. לכל שורה במטריצה לעיל יש איבר פותח ולכן מרחב השורות שלה אכן מהווה בסיס ל$\mathbb{F}^4$ לפי למה $8.5.1$
נקבל שהקבוצה $T=\Sp\{ c_3=x, c_4=1 \}$ אכן מקיימת $W\oplus T=\reals_4[x]$ לפי מה שהוכחתי לעיל.

\pagebreak

\section*{שאלה 4}

יהיו $U,W$ תתי-מרחבים של $\reals^4$, $\dim U>\dim W$. \\
נתון כי $(0,0,1,0)\notin U+W$, וכן $U\cap W=\Sp\{(1,2,3,4), (1,1,1,1), (-1,0,1,2)\}$ \\
עלינו למצוא את המימד של $U+W$ וכן בסיס ל$W$.
\\\\
מאחר ו$U+W\subseteq \reals^4$
מתקיים, לפי משפט $8.3.4$, $\dim(U+W)\leq 4$.\\
אם נניח בשלילה כי $\dim(U+W)=4$, נקבל מחלקו השני של המשפט $U+W=\reals^4$, בסתירה לנתון $(0,0,1,0)\notin U+W$.\\
לכן, $\dim(U+W)\leq 3$.
\\\\
המרחב הנפרש ע"י קבוצת היוצרים הנתונה ל$U\cap W$ הוא מרחב השורות של המטריצה להלן. \\
לפי שאלה $7.5.12$, למטריצות שקולות שורה אותו מרחב שורות, ולכן נדרג:
\begin{align*}
    \begin{pmatrix}
        1  & 2 & 3 & 4 \\
        1  & 1 & 1 & 1 \\
        -1 & 0 & 1 & 2
    \end{pmatrix}
    \xrightarrow[R_3\rightarrow R_3+R_1]{R_2\rightarrow R_2-R_1}
    \begin{pmatrix}
        1 & 2  & 3  & 4  \\
        0 & -1 & -2 & -3 \\
        0 & 2  & 3  & 6
    \end{pmatrix}
    \xrightarrow[]{R_3\rightarrow R_3+2R_2}
    \begin{pmatrix}
        1 & 2  & 3  & 4  \\
        0 & -1 & -2 & -3 \\
        0 & 0  & 0  & 0
    \end{pmatrix}
    \xrightarrow[]{R_2\rightarrow -R_2}
    \begin{pmatrix}
        1 & 2 & 3 & 4 \\
        0 & 1 & 2 & 3 \\
        0 & 0 & 0 & 0
    \end{pmatrix}
\end{align*}
שתי השורות הראשונות במטריצה הם בלתי תלויות לינארית ולכן מהווים בסיס ל$U\cap W$, ומכאן $\dim(U+W)=2$.
\\\\
כעת, היות ו$U\subseteq U+W$ ו$U\cap W\subseteq W$, נקבל לפי משפט 8.3.4 והנתון:
\[
    2=\dim(U\cap W)\leq \dim W < \dim U \leq \dim (U+W)\leq 3
\]
האפשרות היחידה לפתרון היא $\dim(U\cap W)=\dim W=2$, $\dim U = \dim(U+W)=3$. \\
לכן, לפי חלקו השני של המשפט, נקבל $U\cap W = W$. \\
אי לכך, מאחר ו $\{(1,2,3,4), (0,1,2,3)\}$ בסיס ל$U+W$ הקבוצה מהווה בסיס ל$W$.

\end{document}