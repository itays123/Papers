% !TEX program = xelatex
\documentclass{article}
\usepackage[]{amsthm} %lets us use \begin{proof}
\usepackage{amsmath}
\usepackage{enumerate}
\usepackage{xparse}
\usepackage[makeroom]{cancel}
\usepackage[]{amssymb} %gives us the character \varnothing
\usepackage{fontspec}
\usepackage{polyglossia}
\usepackage{relsize}
\usepackage[left=2.0cm, top=2.0cm, right=2.0cm, bottom=2.0cm]{geometry}

\setdefaultlanguage{hebrew}
\setotherlanguage{english}

\setmainfont{[Arial.ttf]}
\newfontfamily\hebrewfont{[Arial.ttf]}

\newcommand\underrel[2]{\mathrel{\mathop{#2}\limits_{#1}}}
\DeclareMathOperator*{\equals}{=}

\title{מטלת מנחה 13 - אלגברה לינארית 1}
\author{328197462}
\date{25/12/2022}

\begin{document}
\maketitle

\section*{שאלה 1}

עלינו להראות כי
\[
    \begin{vmatrix}
        a & b & b \\
        c & d & e \\
        f & g & g
    \end{vmatrix} +
    \begin{vmatrix}
        a & b & b \\
        e & c & d \\
        f & g & g
    \end{vmatrix} +
    \begin{vmatrix}
        a & b & b \\
        d & e & c \\
        f & g & g
    \end{vmatrix} = 0
\]

נשים לב כי שלושת המטריצות זהות פרט לשורה אחת, לכן, נפעיל את משפט
$4.3.4$ פעמיים ונקבל:
\[
    \begin{vmatrix}
        a & b & b \\
        c & d & e \\
        f & g & g
    \end{vmatrix} +
    \begin{vmatrix}
        a & b & b \\
        e & c & d \\
        f & g & g
    \end{vmatrix} +
    \begin{vmatrix}
        a & b & b \\
        d & e & c \\
        f & g & g
    \end{vmatrix} \underrel{4.3.4}{=}
    \begin{vmatrix}
        a     & b     & b     \\
        c + e & d + c & e + d \\
        f     & g     & g
    \end{vmatrix} +
    \begin{vmatrix}
        a & b & b \\
        d & e & c \\
        f & g & g
    \end{vmatrix} \underrel{4.3.4}{=}
    \begin{vmatrix}
        a     & b     & b     \\
        c+d+e & c+d+e & c+d+e \\
        f     & g     & g
    \end{vmatrix}
\]
המטריצה שהתקבלה באגף הימני ביותר היא בעלת שתי עמודות זהות.
לכן, לפי משפט $4.3.5$,
הדטרמיננטה שלה היא 0 ובכך סיימנו את ההוכחה.

\section*{שאלה 2}

\subsection*{סעיף א}
נחשב:
\[
    D_1 =
    \begin{vmatrix}
        a      & b      & 0      & \cdots & \cdots & 0 \\
        0      & a      & b      & 0      & \cdots & 0 \\
        0      & \ddots & \ddots & \ddots &        & 0 \\
        \vdots &        & \ddots & \ddots & \ddots & 0 \\
        0      & 0      & \cdots & 0      & a      & b \\
        b      & 0      & 0      & \cdots & 0      & a
    \end{vmatrix}
    \equals^\text{פיתוח לפי}_\text{שורה 1}
    a \begin{vmatrix}
        a      & b      & 0      & \cdots & 0 \\
        0      & \ddots & \ddots &        & 0 \\
        \vdots & \ddots & \ddots & \ddots & 0 \\
        0      & \cdots & 0      & a      & b \\
        0      & 0      & \cdots & 0      & a
    \end{vmatrix}
    + (-1)^{n+1}b \begin{vmatrix}
        b      & 0      & \cdots & \cdots & 0 \\
        a      & b      & 0      & \cdots & 0 \\
        0      & \ddots & \ddots &        & 0 \\
        \vdots & \ddots & \ddots & \ddots & 0 \\
        0      & \cdots & 0      & a      & b \\
    \end{vmatrix}
\]
באגף ימין קיבלנו שתי מטריציות משולשיות מסדר $(n-1)\times(n-1)$. \\
לפי משפט $4.3.8$, ערך כל דטרמיננטה שווה למכפלת האלכסון הראשי של המטרציה שלה.
נקבל:
\[
    D_1 = a \cdot a^{n-1} + (-1)^{n+1}b \cdot b^{n-1} =
    a^n  + (-1)^{n+1}b^n
\]

\end{document}
