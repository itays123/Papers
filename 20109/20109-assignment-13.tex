% !TEX program = xelatex
\documentclass{article}
\usepackage[]{amsthm} %lets us use \begin{proof}
\usepackage{amsmath}
\usepackage{enumerate}
\usepackage{xparse}
\usepackage[makeroom]{cancel}
\usepackage[]{amssymb} %gives us the character \varnothing
\usepackage{fontspec}
\usepackage{polyglossia}
\usepackage{relsize}
\usepackage[left=2.0cm, top=2.0cm, right=2.0cm, bottom=2.0cm]{geometry}

\setdefaultlanguage{hebrew}
\setotherlanguage{english}

\setmainfont{[Arial.ttf]}
\newfontfamily\hebrewfont{[Arial.ttf]}

\newcommand\underrel[2]{\mathrel{\mathop{#2}\limits_{#1}}}
\DeclareMathOperator*{\equals}{=}
\DeclareMathOperator\cis{cis}
\DeclareMathOperator\Sp{Sp}

\title{מטלת מנחה 13 - אלגברה לינארית 1}
\author{328197462}
\date{25/12/2022}

\begin{document}
\maketitle

\section*{שאלה 1}

עלינו להראות כי
\[
    \begin{vmatrix}
        a & b & b \\
        c & d & e \\
        f & g & g
    \end{vmatrix} +
    \begin{vmatrix}
        a & b & b \\
        e & c & d \\
        f & g & g
    \end{vmatrix} +
    \begin{vmatrix}
        a & b & b \\
        d & e & c \\
        f & g & g
    \end{vmatrix} = 0
\]

נשים לב כי שלושת המטריצות זהות פרט לשורה אחת, לכן, נפעיל את משפט
$4.3.4$ פעמיים ונקבל:
\[
    \begin{vmatrix}
        a & b & b \\
        c & d & e \\
        f & g & g
    \end{vmatrix} +
    \begin{vmatrix}
        a & b & b \\
        e & c & d \\
        f & g & g
    \end{vmatrix} +
    \begin{vmatrix}
        a & b & b \\
        d & e & c \\
        f & g & g
    \end{vmatrix} \underrel{4.3.4}{=}
    \begin{vmatrix}
        a     & b     & b     \\
        c + e & d + c & e + d \\
        f     & g     & g
    \end{vmatrix} +
    \begin{vmatrix}
        a & b & b \\
        d & e & c \\
        f & g & g
    \end{vmatrix} \underrel{4.3.4}{=}
    \begin{vmatrix}
        a     & b     & b     \\
        c+d+e & c+d+e & c+d+e \\
        f     & g     & g
    \end{vmatrix}
\]
המטריצה שהתקבלה באגף הימני ביותר היא בעלת שתי עמודות זהות.
לכן, לפי משפט $4.3.5$,
הדטרמיננטה שלה היא 0 ובכך סיימנו את ההוכחה.

\section*{שאלה 2}

\subsection*{סעיף א}
נחשב:
\[
    D_1 =
    \begin{vmatrix}
        a      & b      & 0      & \cdots & \cdots & 0 \\
        0      & a      & b      & 0      & \cdots & 0 \\
        0      & \ddots & \ddots & \ddots &        & 0 \\
        \vdots &        & \ddots & \ddots & \ddots & 0 \\
        0      & 0      & \cdots & 0      & a      & b \\
        b      & 0      & 0      & \cdots & 0      & a
    \end{vmatrix}
    \equals^\text{פיתוח לפי}_\text{שורה 1}
    a \begin{vmatrix}
        a      & b      & 0      & \cdots & 0 \\
        0      & \ddots & \ddots &        & 0 \\
        \vdots & \ddots & \ddots & \ddots & 0 \\
        0      & \cdots & 0      & a      & b \\
        0      & 0      & \cdots & 0      & a
    \end{vmatrix}
    + (-1)^{n+1}b \begin{vmatrix}
        b      & 0      & \cdots & \cdots & 0 \\
        a      & b      & 0      & \cdots & 0 \\
        0      & \ddots & \ddots &        & 0 \\
        \vdots & \ddots & \ddots & \ddots & 0 \\
        0      & \cdots & 0      & a      & b \\
    \end{vmatrix}
\]
באגף ימין קיבלנו שתי מטריציות משולשיות מסדר $(n-1)\times(n-1)$. \\
לפי משפט $4.3.8$, ערך כל דטרמיננטה שווה למכפלת האלכסון הראשי של המטרציה שלה.
נקבל:
\[
    D_1 = a \cdot a^{n-1} + (-1)^{n+1}b \cdot b^{n-1} =
    a^n  + (-1)^{n+1}b^n
\]

\subsection*{סעיף ב}

נחשב:
\[
    D_2 = \begin{vmatrix}
        1      & 2   & 3 & \cdots & n-2 & n-1 & n      \\
        2      & 3   & 4 & \cdots & n-1 & n   & n      \\
        3      & 4   & 5 & \cdots & n   & n   & n      \\
        \vdots &     &   &        &     &     & \vdots \\
        n-2    & n-1 & n & \cdots & n   & n   & n      \\
        n-1    & n   & n & \cdots & n   & n   & n      \\
        n      & n   & n & \cdots & n   & n   & n      \\
    \end{vmatrix}
\]

\pagebreak

נחסר מהשורה האחרונה את השורה שלפניה. נקבל, לפי משפט $4.3.6$,
\[
    D_2 = \begin{vmatrix}
        1      & 2   & 3 & \cdots & n-2 & n-1 & n      \\
        2      & 3   & 4 & \cdots & n-1 & n   & n      \\
        3      & 4   & 5 & \cdots & n   & n   & n      \\
        \vdots &     &   &        &     &     & \vdots \\
        n-2    & n-1 & n & \cdots & n   & n   & n      \\
        n-1    & n   & n & \cdots & n   & n   & n      \\
        1      & 0   & 0 & \cdots & 0   & 0   & 0      \\
    \end{vmatrix}
\]
נבצע עוד $n-2$
פעולות דומות - נחסר מכל שורה את השורה שלפניה,
החל מהשורה הלפני אחרונה ועד השורה השנייה. \\
שוב, לפי משפט $4.3.6$,
ערך הדטרמיננטה נשאר זהה ונקבל:
\[
    D_2 = \begin{vmatrix}
        1      & 2 & 3 & \cdots & n-2 & n-1 & n      \\
        1      & 1 & 1 & \cdots & 1   & 1   & 0      \\
        1      & 1 & 1 & \cdots & 1   & 0   & 0      \\
        \vdots &   &   &        &     &     & \vdots \\
        1      & 1 & 1 & \cdots & 0   & 0   & 0      \\
        1      & 1 & 0 & \cdots & 0   & 0   & 0      \\
        1      & 0 & 0 & \cdots & 0   & 0   & 0      \\
    \end{vmatrix}
\]
נרצה להגיע לצורה משולשית תחתית.
לשם כך, נחליף את השורה הראשונה עם האחרונה, את השנייה עם הלפני אחרונה וכו'. \\
בסך הכל נבצע $\frac n 2$ החלפות אם $n$ זוגי,
ו$\frac{n-1}{2}$ החלפות אם $n$ אי-זוגי,
כלומר $\lfloor \frac n 2 \rfloor$ החלפות שורה. \\
לפי משפט $4.3.2$,
כל החלפת שורה משנה את סימן הדטרמיננטה. עבור $\lfloor \frac n 2 \rfloor$
החלפות נקבל:
\[
    D_2 = (-1)^{\lfloor \frac n 2 \rfloor}\begin{vmatrix}
        1      & 0 & 0 & \cdots & 0   & 0   & 0      \\
        1      & 1 & 0 & \cdots & 0   & 0   & 0      \\
        1      & 1 & 1 & \cdots & 0   & 0   & 0      \\
        \vdots &   &   &        &     &     & \vdots \\
        1      & 1 & 1 & \cdots & 1   & 0   & 0      \\
        1      & 1 & 1 & \cdots & 1   & 1   & 0      \\
        1      & 2 & 3 & \cdots & n-2 & n-1 & n      \\
    \end{vmatrix}
\]
המטריצה שבאגף ימין היא מטריצה משולשית תחתית. \\
לפי משפט $4.3.8$,
הדטרמיננטה שלה היא מכפלת האלכסון הראשי, ובמקרה זה $1\cdot 1 \cdot \cdots n=n$,
ונקבל:
\[
    D_2 = (-1)^{\lfloor \frac n 2 \rfloor}n
\]

\pagebreak

\section*{שאלה 3}

\subsection*{סעיף א}

תחילה נמצא את ההצגה הקוטבית של $w=r\cis{} \theta = 1-i$.
\[
    \begin{cases}
        r = \sqrt{1^2 + (-1)^2}=\sqrt{2} \\
        \tan \theta = \frac{-1}{1} = -1 \; \Rightarrow \; \theta = -\frac{\pi}{4} + \pi k, k \in \mathbb{Z}
    \end{cases}
\]
משיקולי רביע ניקח $\theta = \frac{7\pi}{4}$,
אז $t=\cis \frac{3\pi}{4}, w=\sqrt{2} \cis \frac{7\pi}{4}$. \\
\\
כעת, נחשב:
\[
    \frac{w}{\overline{t}} =
    \frac{\sqrt{2}\cis \frac{7\pi}{4}}{\overline{\cis{\frac{3\pi}{4}}}} =
    \frac{\sqrt{2}\cis \frac{7\pi}{4}}{\cis{-\frac{3\pi}{4}}} =
    \sqrt{2} \cis{(\frac{7\pi}{4} - (-\frac{3\pi}{4}))} =
    \sqrt{2} \cis{\frac{10\pi}{4}} =
    \sqrt{2} \cis{\frac{5\pi}{2}} =
    \sqrt{2} \cis{\frac{\pi}{2}}
\]
\\
נפתור את המשוואה $z^3=\sqrt{2} \cis{\frac{\pi}{2}}$.
לפי "נוסחת השורשים", נקבל:
\[
    z_k=\sqrt[6]{2} \cis{(\frac{\pi}{6}+\frac{2\pi k}{3})}, k=0,1,2
\]
כלומר
$z_2=\sqrt[6]{2} \cis{\frac{9\pi}{6}}$,
$z_1=\sqrt[6]{2} \cis{\frac{5\pi}{6}}$,
$z_0=\sqrt[6]{2} \cis{\frac{\pi}{6}}$.

\subsection*{סעיף ב}

לפי "נוסחת השורשים", פתרונות המשוואה $z^n=1=1\cis{0}$ יהיו:
\[
    z_k=\sqrt[n]{1} \cis{(\frac{0}{n} + \frac{2\pi k}{n})} =
    \cis{(\frac{2\pi k}{n})},
    k = 0, 1, 2, ..., n-1
\]
\\
מכפלת הפתרונות תהא:
\[
    \prod_{k=0}^{n-1} \cis{(\frac{2\pi k}{n})} =
    \cis{\frac{2\pi\cdot 0}{n}} + \cis{\frac{2\pi\cdot 1}{n}} + \cdots + \cis{\frac{2\pi\cdot (n-1)}{n}} =
    \cis{(\frac{2\pi\cdot 0}{n} + \frac{2\pi\cdot 1}{n} + \cdots + \frac{2\pi\cdot (n-1)}{n})} =
\]
\[
    \cis{(\sum_{k=0}^{n-1}\frac{2\pi k}{n})} =
    \cis{(\frac{2\pi}{n} \cdot \sum_{k=0}^{n-1}k)} \equals_{\text{סכום סדרה חשבונית}}
    \cis{(\frac{2\pi}{n} \cdot \frac{n}{2} \cdot (n-1+0))} =
    \cis{(\pi \cdot (n-1))}
\]
\\
אילו $n$ אי-זוגי,
אז יש $m\in \mathbb{Z}: n=2m+1$,
ונקבל:
\[
    \prod_{k=0}^{n-1} \cis{(\frac{2\pi k}{n})} =
    \cis{(\pi \cdot (n-1))} =
    \cis{(\pi \cdot (2m+1-1))} =
    \cis{2m\pi} = \cis{0}=1
\]

\pagebreak

\section*{שאלה 4}

הקבוצה $V=\{(\alpha, \beta)\;|\; \alpha, \beta \in \mathbb{R}\}$
בצירוף הפעולות שהוגדרו לא מהווה מרחב לינארי מעל $\mathbb{R}$.\\
נראה כי היא אינה מקיימת את אקסיומה ג מאקסיומות הכל בסקלר (פילוג הכפל בסקלר מעל החיבור):
\\\\
נבחר למשל $v=(1,0)\in V$, $\lambda=8, \mu = 2 \in \mathbb{R}$.
מתקיים:
\[
    (\lambda + \mu)v =
    (8+2)\cdot (1,0) \underrel{\text{לפי הגדרה}}{=}
    (1,10\cdot 0) =(1,0)
\]
ואולם
\[
    \lambda v +\mu v =
    8\cdot (1,0) + 2\cdot (1,0) \equals_{\text{לפי הגדרה}}
    (1,0) + (1,0) = (2,0) \ne (1,0) = (\lambda + \mu)v
\]
מאחר והקבוצה בצירוף הפעולות לא מקיימת את אקסיומות המרחב הלינארי נסיק כי היא אינה מרחב לינארי.

\section*{שאלה 5}

נבחן את ארבע הקבוצות בשאלה:

\subsubsection*{.1 קבוצת הפונקציות $W=\{f: \mathbb{R} \rightarrow \mathbb{R} \; | \; f(x+1) = f(x)+1 \;\;\;\; x\in \mathbb{R} \text{לכל } \}$}

נראה כי הקבוצה $W$ לעיל אינה מהווה מרחב לינארי מעל $\mathbb{R}$: \\
הקבוצה אינה סגורה לפעולת חיבור הפונקציות המוגדרת בעמוד 155 בכרך ב: \\
ניקח למשל $u=x, v=x+1\in W$, אז:
\[
    f(x)=u+v=
    x+x+1=
    2x+1
\]
אבל $f(x)\notin W$
כי
\[
    \begin{matrix}
        f(1)=3 & f(0) = 1 & f(0+1)\ne f(0)+1
    \end{matrix}
\]

\subsubsection*{2. קבוצת הפולינומים $M=\{p(x)\in \mathbb{R}_4[x] \; | \; p(x) = p(x-1) \;\;\;\; x\in \mathbb{R} \text{לכל } \}$}

נוכיח כי הקבוצה $M\subseteq \mathbb{R}_4[x]$לעיל מהווה מרחב לינארי מעל $\mathbb{R}$ בעזרת משפט $7.3.2$:
\[
    M=\{p(x)\in \mathbb{R}_4[x] \; | \; p(x) = p(x-1) \;\;\;\; x\in \mathbb{R} \text{לכל } \}
\]
תחילה, $W\ne \emptyset$
כי הפולינום $p(x)=0\in \mathbb{R}_4[x]$ מקיים לכל $x\in \mathbb{R}$ $p(x)=0=p(x-1)$ ומכאן ש $p(x)\in M$.\\
נראה סגירות לחיבור המוגדר בעמוד 155 בכרך ב. יהיו $u,v\in M$. נקבל:
\[
    \begin{matrix}
        (u + v)(x) = u(x)+v(x) \underrel{\text{ מ"ל} \mathbb{R}_4[x]}{\in} \mathbb{R}_4[x] \\
        (u+v)(x)=u(x)+v(x)\underrel{u,v\in M}{=} u(x-1)+v(x-1)=(u+v)(x-1)                  \\
        \Rightarrow u+v\in M
    \end{matrix}
\]
נראה סגירות לכפל בסקלר. יהא $v\in M$ וכן $\lambda \in R$. נקבל:
\[
    \begin{matrix}
        \lambda v \underrel{\text{ מ"ל} \mathbb{R}_4[x]}{\in} \mathbb{R}_4[x] \\
        (\lambda v)(x) = \lambda \cdot v(x) \underrel{v\in M}{=} \lambda \cdot v(x-1)= (\lambda v)(x-1)
    \end{matrix}
\]
הראינו כי שלוש התכונות מתקיימות ולכן לפי $7.3.2$ $M$ מהווה מרחב לינארי.\\
נמצא קבוצה פורשת סופית לקבוצה $M$ שהוגדרה לעיל. \\
תחילה נמצא איבר כללי לאיברי הסדרה. יהא $p(x)=ax^3+bx^2+cx+d\in M$. אז מתקיים:
\[
    \begin{matrix}
        p(x-1)=a(x-1)^3+b(x-1)^2+c(x-1)+d=     \\
        a(x^3-3x^2+3x-1)+b(x^2-2x+1)+c(x-1)+d= \\
        ax^3+(b-3a)x^2+(c-2b+3a)x+(d-c+b-a)x = p(x)
    \end{matrix}
\]
נשווה את מקדמיהם של שני הפולינומים ונקבל:
\[
    \left\{
    \begin{matrix}
        a=a                                              \\
        b-3a=b     & \Rightarrow                   & a=0 \\
        c-2b+3a=c  & \underrel{a=0}{\Rightarrow}   & b=0 \\
        d-c+b-a =d & \underrel{a=b=0}{\Rightarrow} & c=0
    \end{matrix}
    \right.
\]
קיבלנו $p(x)=d$, ולכן $M=\{ p(x)=d \; | \; d\in \mathbb{R} \} = \Sp\{1\}$.

\subsubsection*{3. קבוצת המטריצות $S = \{ \begin{pmatrix}
            a & b \\
            c & d
        \end{pmatrix} \in M_2(\mathbb{R}) \; | \;
        ad=0\}$}
נראה כי הקבוצה $S$ לעיל אינה מהווה מרחב לינארי מעל $\mathbb{R}$: \\
הקבוצה אינה סגורה לפעולת חיבור מטריצות: ניקח למשל $\begin{pmatrix}
        0 & 1 \\
        1 & 1
    \end{pmatrix}, \begin{pmatrix}
        1 & 1 \\
        1 & 0
    \end{pmatrix} \in S$ ונקבל:
\[
    \begin{pmatrix}
        0 & 1 \\
        1 & 1
    \end{pmatrix} + \begin{pmatrix}
        1 & 1 \\
        1 & 0
    \end{pmatrix} = \begin{pmatrix}
        1 & 2 \\
        2 & 1
    \end{pmatrix}
    \underrel{1\cdot 1 \ne 0}{\notin} S
\]

\subsubsection*{4. קבוצת השלשות $
        L = \{(z_1,z_2,z_3)\in \mathbb{C}^3 \; | \; z_2 = \overline{z_1}\}
    $}
נראה כי הקבוצה $L\subseteq \mathbb{C}^3$ לעיל היא מרחב לינארי מעל $\mathbb{R}$: \\
תחילה, $L\ne \emptyset$
כי $(1,1,1)\in \mathbb{C}^3$ מקיימת $1=\overline{1}$ (1 ממשי ולפי משפט $6.4.2$),
ואי לכך $(1,1,1)\in L$. \\
נראה סגירות לחיבור שלשות. יהיו $(z_1,\overline{z_1}, z_3), (w_1, \overline{w_1}, w_3)\in L$. נקבל:
\[
    (z_1,\overline{z_1}, z_3),+ (w_1, \overline{w_1}, w_3) =
    (z_1+w_1, \overline{z_1}+\overline{w_1}, z_3+w_3) \equals_{6.4.2}
    (z_1+w_1, \overline{z_1+w_1}, z_3+w_3)\in L
\]
בנוסף, נראה סגירות לכפל בסקלר מהשדה $\alpha \in \mathbb{R}$. יהא $(z_1, \overline{z_1}, z_3)\in L$ ונקבל אפוא:
\[
    \alpha (z_1, \overline{z_1}, z_3) =
    (\alpha z_1, \alpha \overline{z_1}, \alpha z_3) \equals_{\text{שאלה 6.4.3} }
    (\alpha z_1, \overline{\alpha z_1}, \alpha z_3) \in L
\]
אי לכך, מתקיימים תנאי משפט $7.3.2$ והקבוצה, בצירוף הפעולות הרגילות, מהווה מרחב לינארי מעל $\mathbb{R}$. \\
נמצא קבוצה פורשת סופית לקבוצה, ולשם כך נמצא איבר כללי. יהא $(z_1, \overline{z_1}, z_3)\in L$. נקבל אפוא:
\[
    (z_1, \overline{z_1}, z_3) \underrel{\text{הצגה קרטזית}}{=}
    (a_1+ib_1, a_1-ib_1, a_3+ib_3) =
    a_1(1,1,0)+b_1(i,-i, 0)+a_3(0,0,1)+b_3(0,0,i)
\]
ולכן נקבל:
\[
    L=\{a_1(1,1,0)+b_1(i,-i, 0)+a_3(0,0,1)+b_3(0,0,i) \; | \; a_1, a_3, b_1, b_3 \in \mathbb{R} \}
\]
כלומר, מעל $\mathbb{F}=\mathbb{R}$,
\[
    L=\Sp\{(1,1,0),(i,-i, 0),(0,0,1),(0,0,i) \}
\]

\pagebreak

\section*{שאלה 6}

\end{document}
