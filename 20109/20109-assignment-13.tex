% !TEX program = xelatex
\documentclass{article}
\usepackage[]{amsthm} %lets us use \begin{proof}
\usepackage{amsmath}
\usepackage{enumerate}
\usepackage{xparse}
\usepackage[makeroom]{cancel}
\usepackage[]{amssymb} %gives us the character \varnothing
\usepackage{fontspec}
\usepackage{polyglossia}
\usepackage{relsize}
\usepackage[left=2.0cm, top=2.0cm, right=2.0cm, bottom=2.0cm]{geometry}

\setdefaultlanguage{hebrew}
\setotherlanguage{english}

\setmainfont{[Arial.ttf]}
\newfontfamily\hebrewfont{[Arial.ttf]}

\newcommand\underrel[2]{\mathrel{\mathop{#2}\limits_{#1}}}
\DeclareMathOperator*{\equals}{=}
\DeclareMathOperator\cis{cis}

\title{מטלת מנחה 13 - אלגברה לינארית 1}
\author{328197462}
\date{25/12/2022}

\begin{document}
\maketitle

\section*{שאלה 1}

עלינו להראות כי
\[
    \begin{vmatrix}
        a & b & b \\
        c & d & e \\
        f & g & g
    \end{vmatrix} +
    \begin{vmatrix}
        a & b & b \\
        e & c & d \\
        f & g & g
    \end{vmatrix} +
    \begin{vmatrix}
        a & b & b \\
        d & e & c \\
        f & g & g
    \end{vmatrix} = 0
\]

נשים לב כי שלושת המטריצות זהות פרט לשורה אחת, לכן, נפעיל את משפט
$4.3.4$ פעמיים ונקבל:
\[
    \begin{vmatrix}
        a & b & b \\
        c & d & e \\
        f & g & g
    \end{vmatrix} +
    \begin{vmatrix}
        a & b & b \\
        e & c & d \\
        f & g & g
    \end{vmatrix} +
    \begin{vmatrix}
        a & b & b \\
        d & e & c \\
        f & g & g
    \end{vmatrix} \underrel{4.3.4}{=}
    \begin{vmatrix}
        a     & b     & b     \\
        c + e & d + c & e + d \\
        f     & g     & g
    \end{vmatrix} +
    \begin{vmatrix}
        a & b & b \\
        d & e & c \\
        f & g & g
    \end{vmatrix} \underrel{4.3.4}{=}
    \begin{vmatrix}
        a     & b     & b     \\
        c+d+e & c+d+e & c+d+e \\
        f     & g     & g
    \end{vmatrix}
\]
המטריצה שהתקבלה באגף הימני ביותר היא בעלת שתי עמודות זהות.
לכן, לפי משפט $4.3.5$,
הדטרמיננטה שלה היא 0 ובכך סיימנו את ההוכחה.

\section*{שאלה 2}

\subsection*{סעיף א}
נחשב:
\[
    D_1 =
    \begin{vmatrix}
        a      & b      & 0      & \cdots & \cdots & 0 \\
        0      & a      & b      & 0      & \cdots & 0 \\
        0      & \ddots & \ddots & \ddots &        & 0 \\
        \vdots &        & \ddots & \ddots & \ddots & 0 \\
        0      & 0      & \cdots & 0      & a      & b \\
        b      & 0      & 0      & \cdots & 0      & a
    \end{vmatrix}
    \equals^\text{פיתוח לפי}_\text{שורה 1}
    a \begin{vmatrix}
        a      & b      & 0      & \cdots & 0 \\
        0      & \ddots & \ddots &        & 0 \\
        \vdots & \ddots & \ddots & \ddots & 0 \\
        0      & \cdots & 0      & a      & b \\
        0      & 0      & \cdots & 0      & a
    \end{vmatrix}
    + (-1)^{n+1}b \begin{vmatrix}
        b      & 0      & \cdots & \cdots & 0 \\
        a      & b      & 0      & \cdots & 0 \\
        0      & \ddots & \ddots &        & 0 \\
        \vdots & \ddots & \ddots & \ddots & 0 \\
        0      & \cdots & 0      & a      & b \\
    \end{vmatrix}
\]
באגף ימין קיבלנו שתי מטריציות משולשיות מסדר $(n-1)\times(n-1)$. \\
לפי משפט $4.3.8$, ערך כל דטרמיננטה שווה למכפלת האלכסון הראשי של המטרציה שלה.
נקבל:
\[
    D_1 = a \cdot a^{n-1} + (-1)^{n+1}b \cdot b^{n-1} =
    a^n  + (-1)^{n+1}b^n
\]

\subsection*{סעיף ב}

נחשב:
\[
    D_2 = \begin{vmatrix}
        1      & 2   & 3 & \cdots & n-2 & n-1 & n      \\
        2      & 3   & 4 & \cdots & n-1 & n   & n      \\
        3      & 4   & 5 & \cdots & n   & n   & n      \\
        \vdots &     &   &        &     &     & \vdots \\
        n-2    & n-1 & n & \cdots & n   & n   & n      \\
        n-1    & n   & n & \cdots & n   & n   & n      \\
        n      & n   & n & \cdots & n   & n   & n      \\
    \end{vmatrix}
\]

\pagebreak

נחסר מהשורה האחרונה את השורה שלפניה. נקבל, לפי משפט $4.3.6$,
\[
    D_2 = \begin{vmatrix}
        1      & 2   & 3 & \cdots & n-2 & n-1 & n      \\
        2      & 3   & 4 & \cdots & n-1 & n   & n      \\
        3      & 4   & 5 & \cdots & n   & n   & n      \\
        \vdots &     &   &        &     &     & \vdots \\
        n-2    & n-1 & n & \cdots & n   & n   & n      \\
        n-1    & n   & n & \cdots & n   & n   & n      \\
        1      & 0   & 0 & \cdots & 0   & 0   & 0      \\
    \end{vmatrix}
\]
נבצע עוד $n-2$
פעולות דומות - נחסר מכל שורה את השורה שלפניה,
החל מהשורה הלפני אחרונה ועד השורה השנייה. \\
שוב, לפי משפט $4.3.6$,
ערך הדטרמיננטה נשאר זהה ונקבל:
\[
    D_2 = \begin{vmatrix}
        1      & 2 & 3 & \cdots & n-2 & n-1 & n      \\
        1      & 1 & 1 & \cdots & 1   & 1   & 0      \\
        1      & 1 & 1 & \cdots & 1   & 0   & 0      \\
        \vdots &   &   &        &     &     & \vdots \\
        1      & 1 & 1 & \cdots & 0   & 0   & 0      \\
        1      & 1 & 0 & \cdots & 0   & 0   & 0      \\
        1      & 0 & 0 & \cdots & 0   & 0   & 0      \\
    \end{vmatrix}
\]
נרצה להגיע לצורה משולשית תחתית.
לשם כך, נחליף את השורה הראשונה עם האחרונה, את השנייה עם הלפני אחרונה וכו'. \\
בסך הכל נבצע $\frac n 2$ החלפות אם $n$ זוגי,
ו$\frac{n-1}{2}$ החלפות אם $n$ אי-זוגי,
כלומר $\lfloor \frac n 2 \rfloor$ החלפות שורה. \\
לפי משפט $4.3.2$,
כל החלפת שורה משנה את סימן הדטרמיננטה. עבור $\lfloor \frac n 2 \rfloor$
החלפות נקבל:
\[
    D_2 = (-1)^{\lfloor \frac n 2 \rfloor}\begin{vmatrix}
        1      & 0 & 0 & \cdots & 0   & 0   & 0      \\
        1      & 1 & 0 & \cdots & 0   & 0   & 0      \\
        1      & 1 & 1 & \cdots & 0   & 0   & 0      \\
        \vdots &   &   &        &     &     & \vdots \\
        1      & 1 & 1 & \cdots & 1   & 0   & 0      \\
        1      & 1 & 1 & \cdots & 1   & 1   & 0      \\
        1      & 2 & 3 & \cdots & n-2 & n-1 & n      \\
    \end{vmatrix}
\]
המטריצה שבאגף ימין היא מטריצה משולשית תחתית. \\
לפי משפט $4.3.8$,
הדטרמיננטה שלה היא מכפלת האלכסון הראשי, ובמקרה זה $1\cdot 1 \cdot \cdots n=n$,
ונקבל:
\[
    D_2 = (-1)^{\lfloor \frac n 2 \rfloor}n
\]

\pagebreak

\section*{שאלה 3}

\subsection*{סעיף א}

תחילה נמצא את ההצגה הקוטבית של $w=r\cis{} \theta = 1-i$.
\[
    \begin{cases}
        r = \sqrt{1^2 + (-1)^2}=\sqrt{2} \\
        \tan \theta = \frac{-1}{1} = -1 \; \Rightarrow \; \theta = -\frac{\pi}{4} + \pi k, k \in \mathbb{Z}
    \end{cases}
\]
משיקולי רביע ניקח $\theta = \frac{7\pi}{4}$,
אז $t=\cis \frac{3\pi}{4}, w=\sqrt{2} \cis \frac{7\pi}{4}$. \\
\\
כעת, נחשב:
\[
    \frac{w}{\overline{t}} =
    \frac{\sqrt{2}\cis \frac{7\pi}{4}}{\overline{\cis{\frac{3\pi}{4}}}} =
    \frac{\sqrt{2}\cis \frac{7\pi}{4}}{\cis{-\frac{3\pi}{4}}} =
    \sqrt{2} \cis{(\frac{7\pi}{4} - (-\frac{3\pi}{4}))} =
    \sqrt{2} \cis{\frac{10\pi}{4}} =
    \sqrt{2} \cis{\frac{5\pi}{2}} =
    \sqrt{2} \cis{\frac{\pi}{2}}
\]
\\
נפתור את המשוואה $z^3=\sqrt{2} \cis{\frac{\pi}{2}}$.
לפי "נוסחת השורשים", נקבל:
\[
    z_k=\sqrt[6]{2} \cis{(\frac{\pi}{6}+\frac{2\pi k}{3})}, k=0,1,2
\]
כלומר
$z_2=\sqrt[6]{2} \cis{\frac{9\pi}{6}}$,
$z_1=\sqrt[6]{2} \cis{\frac{5\pi}{6}}$,
$z_0=\sqrt[6]{2} \cis{\frac{\pi}{6}}$.

\subsection*{סעיף ב}

לפי "נוסחת השורשים", פתרונות המשוואה $z^n=1=1\cis{0}$ יהיו:
\[
    z_k=\sqrt[n]{1} \cis{(\frac{0}{n} + \frac{2\pi k}{n})} =
    \cis{(\frac{2\pi k}{n})},
    k = 0, 1, 2, ..., n-1
\]
\\
מכפלת הפתרונות תהא:
\[
    \prod_{k=0}^{n-1} \cis{(\frac{2\pi k}{n})} =
    \cis{\frac{2\pi\cdot 0}{n}} + \cis{\frac{2\pi\cdot 1}{n}} + \cdots + \cis{\frac{2\pi\cdot (n-1)}{n}} =
    \cis{(\frac{2\pi\cdot 0}{n} + \frac{2\pi\cdot 1}{n} + \cdots + \frac{2\pi\cdot (n-1)}{n})} =
\]
\[
    \cis{(\sum_{k=0}^{n-1}\frac{2\pi k}{n})} =
    \cis{(\frac{2\pi}{n} \cdot \sum_{k=0}^{n-1}k)} \equals_{\text{סכום סדרה חשבונית}}
    \cis{(\frac{2\pi}{n} \cdot \frac{n}{2} \cdot (n-1+0))} =
    \cis{(\pi \cdot (n-1))}
\]
\\
אילו $n$ אי-זוגי,
אז יש $m\in \mathbb{Z}: n=2m+1$,
ונקבל:
\[
    \prod_{k=0}^{n-1} \cis{(\frac{2\pi k}{n})} =
    \cis{(\pi \cdot (n-1))} =
    \cis{(\pi \cdot (2m+1-1))} =
    \cis{2m\pi} = \cis{0}=1
\]

\end{document}
