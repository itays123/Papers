% !TEX program = xelatex
\documentclass{article}
\usepackage[]{amsthm} %lets us use \begin{proof}
\usepackage{amsmath}
\usepackage{enumerate}
\usepackage{xparse}
\usepackage[makeroom]{cancel}
\usepackage[]{amssymb} %gives us the character \varnothing
\usepackage{fontspec}
\usepackage{polyglossia}
\usepackage{relsize}
\usepackage[left=2.0cm, top=2.0cm, right=2.0cm, bottom=2.0cm]{geometry}

\setdefaultlanguage{hebrew}
\setotherlanguage{english}

\setmainfont{[Arial.ttf]}
\newfontfamily\hebrewfont{[Arial.ttf]}

\title{מטלת מנחה 12 - אלגברה לינארית 1}
\author{328197462}
\date{04/12/2022}

\begin{document}
\maketitle

\section*{שאלה 1}

\subsection*{סעיף א}
נפתור לפי טענה $2.6.5$. על מנת שקבוצת הוקטורים תהא תלויה לינארית,
ננסה למצוא פתרון לא טרוייואלי למערכת ההומוגנית שמטריצת המקדמים המצומצמת שלה היא $A$ המפורטת מטה, בעזרת דירוג:

\[
    A = \begin{pmatrix}
        1 & 1 & 1 & 1 \\
        3 & 0 & 1 & 5 \\
        0 & 1 & 1 & 0 \\
        1 & 0 & 0 & 1 \\
    \end{pmatrix}
    \xrightarrow[R_4\rightarrow R_4-R_1]{R_2\rightarrow R_2-3R_1}
    \begin{pmatrix}
        1 & 1  & 1  & 1 \\
        0 & -3 & -2 & 2 \\
        0 & 1  & 1  & 0 \\
        0 & -1 & -1 & 0 \\
    \end{pmatrix}
    \xrightarrow{R_2\leftrightarrow R_3}
    \begin{pmatrix}
        1 & 1  & 1  & 1 \\
        0 & 1  & 1  & 0 \\
        0 & -3 & -2 & 2 \\
        0 & -1 & -1 & 0 \\
    \end{pmatrix}
    \xrightarrow[R_4\rightarrow R_4+R_2]{R_3\rightarrow R_3+3R_2}
    \begin{pmatrix}
        1 & 1 & 1 & 1 \\
        0 & 1 & 1 & 0 \\
        0 & 0 & 1 & 2 \\
        0 & 0 & 0 & 0 \\
    \end{pmatrix}
\]
קיבלנו 3 משתנים קשורים ומשתנה אחד חופשי. לכן, נסיק כי קיים יותר מפתרון יחיד למערכת, בפרט יש לה פתרון לא טריוויאלי. \\
לפי טענה 2.6.5 נסיק כי הקבוצה תלויה לינארית.

\subsection*{סעיף ב}

שוב, על מנת שקבוצת הוקטורים מעל $\mathbb{Z}_3$ תהא בלתי תלויה לינארית, למערכת המשוואות ההומוגנית שמטריצת המקדמים שלה היא $B$ המפורטת מטה צריך להיות פתרון יחיד (הפתרון הטריוויאלי) בלבד.

\[
    B = \begin{pmatrix}
        1 & 1 & 1 & 0 \\
        1 & 2 & 1 & 2 \\
        2 & 1 & 1 & 2 \\
        2 & 2 & 2 & 2
    \end{pmatrix}
    \xrightarrow[R_3\rightarrow R_3-2R_1, R_4\rightarrow R_4-2R_1]{R_2\rightarrow R_2-R_1}
    \begin{pmatrix}
        1 & 1  & 1  & 0 \\
        0 & 1  & 0  & 2 \\
        0 & -1 & -1 & 2 \\
        0 & 0  & 0  & 2
    \end{pmatrix}
    \xrightarrow{R_3\rightarrow R_3+R_2}
    \begin{pmatrix}
        1 & 1 & 1  & 0 \\
        0 & 1 & 0  & 2 \\
        0 & 0 & -1 & 1 \\
        0 & 0 & 0  & 2
    \end{pmatrix}
\]

קיבלנו כי כל המשתנים קשורים ולכן יש פתרון יחיד. מאחר והמערכת הומוגנית,זהו הפתרון הטריוויאלי ונסיק כי הקבוצה בלתי תלויה לינארית

\end{document}