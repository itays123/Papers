% !TEX program = xelatex
\documentclass{article}
\usepackage[]{amsthm} %lets us use \begin{proof}
\usepackage{amsmath}
\usepackage{enumerate}
\usepackage{xparse}
\usepackage[makeroom]{cancel}
\usepackage[]{amssymb} %gives us the character \varnothing
\usepackage{fontspec}
\usepackage{polyglossia}
\usepackage{relsize}
\usepackage[left=2.0cm, top=2.0cm, right=2.0cm, bottom=2.0cm]{geometry}

\setdefaultlanguage{hebrew}
\setotherlanguage{english}

\setmainfont{[Arial.ttf]}
\newfontfamily\hebrewfont{[Arial.ttf]}

\newcommand\underrel[2]{\mathrel{\mathop{#2}\limits_{#1}}}

\title{מטלת מנחה 12 - אלגברה לינארית 1}
\author{328197462}
\date{04/12/2022}

\begin{document}
\maketitle

\section*{שאלה 1}

\subsection*{סעיף א}
נפתור לפי טענה $2.6.5$. על מנת שקבוצת הוקטורים תהא תלויה לינארית,
ננסה למצוא פתרון לא טרוייואלי למערכת ההומוגנית שמטריצת המקדמים המצומצמת שלה היא $A$ המפורטת מטה, בעזרת דירוג:

\[
    A = \begin{pmatrix}
        1 & 1 & 1 & 1 \\
        3 & 0 & 1 & 5 \\
        0 & 1 & 1 & 0 \\
        1 & 0 & 0 & 1 \\
    \end{pmatrix}
    \xrightarrow[R_4\rightarrow R_4-R_1]{R_2\rightarrow R_2-3R_1}
    \begin{pmatrix}
        1 & 1  & 1  & 1 \\
        0 & -3 & -2 & 2 \\
        0 & 1  & 1  & 0 \\
        0 & -1 & -1 & 0 \\
    \end{pmatrix}
    \xrightarrow{R_2\leftrightarrow R_3}
    \begin{pmatrix}
        1 & 1  & 1  & 1 \\
        0 & 1  & 1  & 0 \\
        0 & -3 & -2 & 2 \\
        0 & -1 & -1 & 0 \\
    \end{pmatrix}
    \xrightarrow[R_4\rightarrow R_4+R_2]{R_3\rightarrow R_3+3R_2}
    \begin{pmatrix}
        1 & 1 & 1 & 1 \\
        0 & 1 & 1 & 0 \\
        0 & 0 & 1 & 2 \\
        0 & 0 & 0 & 0 \\
    \end{pmatrix}
\]
קיבלנו 3 משתנים קשורים ומשתנה אחד חופשי. לכן, נסיק כי קיים יותר מפתרון יחיד למערכת, בפרט יש לה פתרון לא טריוויאלי. \\
לפי טענה 2.6.5 נסיק כי הקבוצה תלויה לינארית.

\subsection*{סעיף ב}

שוב, על מנת שקבוצת הוקטורים מעל $\mathbb{Z}_3$ תהא בלתי תלויה לינארית, למערכת המשוואות ההומוגנית שמטריצת המקדמים שלה היא $B$ המפורטת מטה צריך להיות פתרון יחיד (הפתרון הטריוויאלי) בלבד.

\[
    B = \begin{pmatrix}
        1 & 1 & 1 & 0 \\
        1 & 2 & 1 & 2 \\
        2 & 1 & 1 & 2 \\
        2 & 2 & 2 & 2
    \end{pmatrix}
    \xrightarrow[R_3\rightarrow R_3-2R_1, R_4\rightarrow R_4-2R_1]{R_2\rightarrow R_2-R_1}
    \begin{pmatrix}
        1 & 1  & 1  & 0 \\
        0 & 1  & 0  & 2 \\
        0 & -1 & -1 & 2 \\
        0 & 0  & 0  & 2
    \end{pmatrix}
    \xrightarrow{R_3\rightarrow R_3+R_2}
    \begin{pmatrix}
        1 & 1 & 1  & 0 \\
        0 & 1 & 0  & 2 \\
        0 & 0 & -1 & 1 \\
        0 & 0 & 0  & 2
    \end{pmatrix}
\]

קיבלנו כי כל המשתנים קשורים ולכן יש פתרון יחיד. מאחר והמערכת הומוגנית,זהו הפתרון הטריוויאלי ונסיק כי הקבוצה בלתי תלויה לינארית

\section*{שאלה 2}

יהא $F$
שדה אינסופי ו
$n\geq 1$
מספר טבעי.
יהיו
$v_1, v_2, ..., v_n, w\in F^n$
וקטורים שונים.

נשים לב כי למשוואה ההומוגנית
$x_1v_1+x_2v_2+\cdots+x_nv_n=\underline{0}$
יש פתרון יחיד (הפתרון הטריוויאלי) או אינסוף פתרונות, מאחר ומדובר בשדה אינסופי.
אז נניח בשלילה כי למשוואה יש פתרון יחיד - הפתרון הטריוויאלי.

נקבל, לפי הגדרה, כי הקבוצה
$A=\{ v_1, v_2, ..., v_n \}$
בלתי תלויה לינארית.
כעת, $A$
בת $n$
איברים ובת"ל, לכן לפי 2.7.11
מהווה בסיס ל $F^n$,
ובפרט פורשת את $F^n$.

לכן, עבור $w\in F^n$
יש צירוף לינארי של איברי $A$,
כלומר קיימים
$x_1, x_2, ..., x_n\in F$
כך ש
$x_1v_1+x_2v_2+\cdots+x_nv_n=w$,
אבל לפי הנתון אין ל$w$
צירוף לינארי כזה!

\pagebreak

\section*{שאלה 3}

נתונות $A_{m\times n}, B_{n\times m}: A \cdot B = I_m$

\subsection*{סעיף א}

עלינו להוכיח כי למערכת ההומוגנית $B\underline{x} = \underline{0}$
יש פתרון יחיד. מאחר ומדובר במערכת הומוגנית, ברור כי פתרון זה יהיה הפתרון הטריוויאלי.
\\\\
אכן, יהא $\underline{c}\in F^m$
פתרון למערכת ונרצה להוכיח $\underline{c} = \underline{0}$.
מתקיים:
\[
    \underline{c} \underrel{3.5.3}{=}
    I_m \cdot \underline{c} \underrel{\text{לפי הנתון}}{=}
    (AB) \underline{c} \underrel{\text{קיבוציות}}{=}
    A(B\underline{c}) \underrel{\text{\underline{c} פתרון למערכת}}{=}
    A_{m\times n} \cdot \underline{0}_{n\times 1} =
    \underline{0}_{m\times 1}
\]
ובכך הוכחנו כי הפתרון הטריוויאלי הוא פתרון יחיד.

\subsection*{סעיף ב}

ניעזר בטענה $2.6.5$. \\
לפי סעיף א' למערכת $B\underline{x}=\underline{0}$ \\
יש פתרון יחיד. לכן, לפי טענה זו, וקטורי העמודות של מטריצת המקדמים $B$,
שהם קבוצה של $m$
מאיברי $F^n$,
מהווים קבוצה בלתי-תלויה לינארית. \\
כעת, מאחר והקבוצה בלתי-תלויה לינארית, נובע לפי מסקנה $2.6.7$
כי $m\leq n$
ובכך סיימנו את הההוכחה.

\subsection*{סעיף ג}

נניח כי $B\cdot X= I_n$. \\
באופן דומה לסעיף א',
נקבל כי למערכת ההומוגנית $X\underline{x} = \underline{0}$
יש פתרון יחיד, ובאופן דומה לסעיף ב' נסיק $n\leq m$.
\\\\
נקבל $m=n$,
נציב בנתון נקבל $A\cdot B=I_n$,
כלומר $A,B$
הפיכות ומתקיים $B^{-1}=A$. \\\\
נחזור להנחה. נקבל $B\cdot X = I_n = B\cdot A$. \\
מאחר ו$B$
הפיכה, נפעיל את כלל הצמצום $3.8.3$
ונקבל $X=A$ ובכך סיימנו את ההוכחה.

\section*{שאלה 4}

תחילה נוכיח כי $(Ab^2-A)=A(B+I)(B-I)$:
\[
    (B+I)(B-I) \underrel{\text{לפי פילוג}}{=}
    B(B-I) + I(B-I) \underrel{\text{לפי פילוג + $3.5.3$}}{=}
    B^2-B\cdot I+B-I \underrel{3.5.3}{=}
    B^2-B+B-I =
    B^2-I
\]
\[
    A(B+I)(B-I) \underrel{\text{חישוב קודם}}{=}
    A(B^2-I) \underrel{\text{לפי פילוג + $3.5.3$}}{=}
    AB^2-A
\]
\\
כעת, ידוע כי $AB^2-A$ הפיכה.
נקבל, לפי שאלה $3.8.3$:
\[
    (AB^2-A)^{-1}=(A(B+I)(B-I))^{-1}=(B-I)^{-1}(B+I)^{-1}A^{-1}
\]
נסיק כי קיימות המטריצות $A^{-1}, (B+I)^{-1}, (B-I)^{-1}$. \\
בנוסף, נקבל לפי פילוג ו $3.5.3+3.8.2$:
$
    BA+A=(B+I)A
$. \\
לכן, לפי משפט $3.8.4$,
המטריצה $BA+A$
הפיכה כמכפלת מטריצות הפיכות ובכך סיימנו את ההוכחה.

\pagebreak

\section*{שאלה 5}

\subsection*{סעיף א}

לפי הנתון, המטריצה $I+2A$ הפיכה. \\
לכן, לפי משפט $3.8.4$,
גם המטריצה $(I+2A)^t$ הפיכה.

נחשב:
\[
    (I+2A)^t \underrel{3.3.7}{=}
    I^t+(2A)^t = \underrel{3.3.7}{=}
    I + 2(A^t) = \underrel{\text{לפי הנתון}}{=}
    I + 2(-A) = \underrel{3.3.5}{=}
    I-2A
\]
ולכן גם המטריצה $I-2A$ הפיכה.

\subsection*{סעיף ב}

עבור $C=(I-2A)(I+2A)^{-1}$ נחשב את $C^t$:
\[
    C^t=((I-2A)(I+2A)^{-1})^t \underrel{3.4.5}{=}
    ((I+2A)^{-1})^t(I-2A)^t \underrel{3.8.4}{=}
    ((I+2A)^t)^{-1}(I-2A)^t \underrel{3.2.4 + \text{חישוב קודם}}{=}
    (I-2A)^{-1}(I+2A)
\]

נראה כי $(I-2A), (I+2A)$ מתחלפות:
\[
    (I-2A)(I+2A) \underrel{\text{פילוג}}{=}
    I(I+2A) - 2A(I+2A) \underrel{\text{לפי פילוג + $3.5.3$}}{=}
    I + 2A - 2A - 2A(2A) \underrel{3.5.6}{=}
    I- 4A^2
\]
באופן דומה $(I+2A)(I-2A) = I-4A^2$, ולכן המטריצות מתחלפות.
\\\\
נקבל:
\[
    C^tC=
    (I-2A)^{-1}(I+2A)(I-2A)(I+2A)^{-1}\underrel{\text{קיבוציות וחישוב קודם}}{=}
    (I-2A)^{-1}(I-2A)(I+2A)(I+2A)^{-1} \underrel{\text{קיבוציות}}{=}
    I \cdot I = I
\]

\end{document}