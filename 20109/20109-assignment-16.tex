% !TEX program = xelatex
\documentclass{article}
\usepackage[]{amsthm} %lets us use \begin{proof}
\usepackage{amsmath}
\usepackage{enumerate}
\usepackage{xparse}
\usepackage[makeroom]{cancel}
\usepackage[]{amssymb} %gives us the character \varnothing
\usepackage{fontspec}
\usepackage{polyglossia}
\usepackage{relsize}
\usepackage{graphicx}
\usepackage[left=2.0cm, top=2.0cm, right=2.0cm, bottom=2.0cm]{geometry}

\setdefaultlanguage{hebrew}
\setotherlanguage{english}

\setmainfont{[Arial.ttf]}
\newfontfamily\hebrewfont{[Arial.ttf]}

\newcommand\niton{\not\mathrel{\text{\reflectbox{$\in$}}}}
\newcommand\underrel[2]{\mathrel{\mathop{#2}\limits_{#1}}}
\DeclareMathOperator*{\equals}{=}
\DeclareMathOperator\Sp{Sp}
\DeclareMathOperator\Image{Im}
\def\reals{\mathbb{R}}
\def\field{\mathbb{F}}
\def\zerovec{\underline{0}}

\title{מטלת מנחה 16 - אלגברה לינארית 1}
\author{328197462}
\date{31/01/2023}

\begin{document}
\maketitle

\section*{שאלה 1}

\subsection*{סעיף א}

נביע באמצעות $a$ את הפולינום האופייני של $A=\begin{pmatrix}
        0 & a & 1  \\
        a & 0 & -1 \\
        0 & 0 & a
    \end{pmatrix}$
\begin{align*}
    p(t)=|tI-A| & =\begin{vmatrix}
        t  & -a & -1  \\
        -a & t  & 1   \\
        0  & 0  & t-a
    \end{vmatrix}\equals^{R_3\text{פיתוח לפי }}
    (t-a)\begin{vmatrix}
        t  & -a \\
        -a & t
    \end{vmatrix}=
    (t-a)(t^2-a^2)=
    (t-a)^2(t+a)
\end{align*}
הערכים העצמיים של המטריצה יהיו $\lambda=0$ בריבוי אלגברי 3 כאשר $a=0$, ובמקרה אחר יהיו $\lambda=a$ בריבוי אלגברי 2 ו$\lambda=-a$ בריבוי אלגברי 1.
\\\\
נדון בריבוי הגיאומטרי של $\lambda=0$ כאשר $a=0$. זהו ממד מרחב האפס של המטריצה $0I-A$, שהוא מרחב האפס של $A$ - $P(A)$. \\
לפי $8.6.1$, עבור $A$ מסדר 3 נקבל $\dim P(A)=3-\rho(A)=2$. הריבוי הגיאומטרי של $\lambda=0$ שונה מהריבוי האלגברי ולכן לפי 11.5.4 המטריצה לא לכסינה כאשר $a=0$.
\\\\
כעת נדון בריבויים הגיאומטריים של הערכים העצמיים של $A$ במקרה הנוסף. הריבוי הגיאומטרי של $\lambda=-a$ הוא לפחות 1 (שהרי יש בממד העצמי וקטור שאינו וקטור האפס) ולכל היותר (לפי  $11.5.3$) כריבוי האלגברי - $1$. נסיק כי בסך הכל הריבוי הגיאומטרי של $\lambda=-a$ הוא 1.\\
הריבוי הגיאומטרי של $\lambda=a$ הוא ממד מרחב האפס של המטריצה $aI-A=\begin{pmatrix}
        a  & -a & -1 \\
        -a & a  & 1  \\
        0  & 0  & 0
    \end{pmatrix}$. נחשב את דרגת המטריצה:
\begin{align*}
    \rho(aI-A)=\rho\begin{pmatrix}
        a  & -a & -1 \\
        -a & a  & 1  \\
        0  & 0  & 0
    \end{pmatrix}\equals^{R_2\rightarrow R_2+R_1}_{8.5.1}
    \rho\begin{pmatrix}
        a & -a & -1 \\
        0 & 0  & 0  \\
        0 & 0  & 0
    \end{pmatrix}=1
\end{align*}
ושוב לפי 8.6.1 נקבל שהריבוי הגיאומטרי של $\lambda=a$ הוא 2. במקרה זה, קיבלנו שעבור כל הערכים העצמיים הריבוי האלגברי שווה לריבוי הגיאומטרי ולכן לפי 11.5.4 המטריצה A לכסינה עבור $a\ne 0$.

\subsection*{סעיף ב}

עבור $a=-1$ נקבל $A=\begin{pmatrix}
        0  & -1 & 1  \\
        -1 & 0  & -1 \\
        0  & 0  & -1
    \end{pmatrix}$.
הערכים העצמיים של המטריצה $A$ הם $\lambda=1$ בר"א ור"ג 1, ו$\lambda=-1$ בר"א ור"ג 2.\\
לכן, $A$ לכסינה ודומה למטריצה $D=\begin{pmatrix}
        1 & 0  & 0  \\
        0 & -1 & 0  \\
        0 & 0  & -1
    \end{pmatrix}$. \\
כמו כן, קיים במרחב העצמי של $\lambda=1$ וקטור השונה מאפס $v_1$ כך ש$Av_1=v_1$, וקיימים במרחב העצמי של $\lambda=-1$ שני וקטורים בלתי תלויים לינארית ושונים מאפס (כי ממד המרחב הוא 2) $v_2, v_3$ המקיימים $Av_2=-v_2, Av_3=-v_3$. \\
הוקטור $v_1$ השייך לערך עצמי שונה לא תלוי לינארית ב$v_2, v_3$ לפי 11.2.4 ולכן השלשה $(v_1, v_2, v_3)$ בת"ל בת שלושה וקטורים עצמיים. \\
המטריצה $P=(v_1 \ | \ v_2 \ | \ v_3)$ הפיכה ומתקיים לפי 11.3.7 $D=P^{-1}AP$.

\pagebreak

נמצא ערכים מתאימים ל$v_1, v_2, v_3$. עלינו למצוא פתרון לא טריוואלי $v_1$ למערכת ההומוגנית $(I-A)x=0$. נדרג:
\begin{align*}
    I-A=
    \begin{pmatrix}
        1 & 1 & -1 \\
        1 & 1 & 1  \\
        0 & 0 & 2
    \end{pmatrix}
    \rightarrow
    \begin{pmatrix}
        1 & 1 & -1 \\
        0 & 0 & 2  \\
        0 & 0 & 2
    \end{pmatrix}
    \rightarrow
    \begin{pmatrix}
        1 & 1 & 0 \\
        0 & 0 & 1 \\
        0 & 0 & 0
    \end{pmatrix}
\end{align*}
הוקטור $v_1=(-1,1,0)$ פותר משוואה זו. באופן דומה נדרג את $-I-A$:
\begin{align*}
    -I-A=\begin{pmatrix}
        -1 & 1  & -1 \\
        1  & -1 & 1  \\
        0  & 0  & 0
    \end{pmatrix}\rightarrow
    \begin{pmatrix}
        -1 & 1 & -1 \\
        0  & 0 & 0  \\
        0  & 0 & 0
    \end{pmatrix}
\end{align*}
נסמן $x_3=t, x_2=s$, אז $x_1=s-t$ ו-$(s-t, s, t)$ פתרונות למשוואה. ניקח למשל $v_2=(1,1,0), v_3=(-1, 0, 1)$. \\
נקבל, לפי השיקולים שהוסברו לעיל, כי $P=\begin{pmatrix}
        -1 & 1 & -1 \\
        1  & 1 & 0  \\
        0  & 0 & 1
    \end{pmatrix}$ מלכסנת את $A$.
\\\\
נמצא את המטריצה ההופכית $P^{-1}$. מציאתה תסייע לנו בחישוב.
\begin{align*}
    (P | I) & =
    \left(
    \begin{matrix}
            -1 & 1 & -1 \\
            1  & 1 & 0  \\
            0  & 0 & 1
        \end{matrix}
    \left|
    \begin{matrix}
            1 & 0 & 0 \\
            0 & 1 & 0 \\
            0 & 0 & 1
        \end{matrix}
    \right.
    \right)
    \xrightarrow{R_1\rightarrow R_1+R_3}
    \left(
    \begin{matrix}
            -1 & 1 & 0 \\
            1  & 1 & 0 \\
            0  & 0 & 1
        \end{matrix}
    \left|
    \begin{matrix}
            1 & 0 & 1 \\
            0 & 1 & 0 \\
            0 & 0 & 1
        \end{matrix}
    \right.
    \right)
    \xrightarrow[]{R_2\rightarrow R_2+R_1}
    \left(
    \begin{matrix}
            -1 & 1 & 0 \\
            0  & 2 & 0 \\
            0  & 0 & 1
        \end{matrix}
    \left|
    \begin{matrix}
            1 & 0 & 1 \\
            1 & 1 & 1 \\
            0 & 0 & 1
        \end{matrix}
    \right.
    \right)
    \rightarrow                                                                \\
            & \xrightarrow[R_2\rightarrow \frac{1}{2}R_2]{R_1\rightarrow -R_1}
    \left(
    \begin{matrix}
            1 & -1 & 0 \\
            0 & 1  & 0 \\
            0 & 0  & 1
        \end{matrix}
    \left|
    \begin{matrix}
            -1  & 0   & -1  \\
            0.5 & 0.5 & 0.5 \\
            0   & 0   & 1
        \end{matrix}
    \right.
    \right)
    \xrightarrow{R_1\rightarrow R_1+R_2}
    \left(
    \begin{matrix}
            1 & 0 & 0 \\
            0 & 1 & 0 \\
            0 & 0 & 1
        \end{matrix}
    \left|
    \begin{matrix}
            -0.5 & 0.5 & -0.5 \\
            0.5  & 0.5 & 0.5  \\
            0    & 0   & 1
        \end{matrix}
    \right.
    \right)
    =(I | P^{-1})
\end{align*}
כעת, היות ו$D=P^{-1}AP$, נקבל $A=PDP^{-1}$ ולכן:
\begin{align*}
    A^{2023} & =(PDP^{-1})^{2023} =
    (PDP^{-1})(PDP^{-1})\cdots(PDP^{-1})\equals_{\text{קיבוציות}}
    PD(P^{-1}P)D(P^{-1}P)\cdots(P^{-1}P)DP^{-1}=PD^{2023}P^{-1}=             \\
             & = \begin{pmatrix}
        -1 & 1 & -1 \\
        1  & 1 & 0  \\
        0  & 0 & 1
    \end{pmatrix}\begin{pmatrix}
        1 & 0  & 0  \\
        0 & -1 & 0  \\
        0 & 0  & -1
    \end{pmatrix}^{2023}
    \begin{pmatrix}
        -0.5 & 0.5 & -0.5 \\
        0.5  & 0.5 & 0.5  \\
        0    & 0   & 1
    \end{pmatrix}=
    \begin{pmatrix}
        -1 & 1 & -1 \\
        1  & 1 & 0  \\
        0  & 0 & 1
    \end{pmatrix}\begin{pmatrix}
        1 & 0  & 0  \\
        0 & -1 & 0  \\
        0 & 0  & -1
    \end{pmatrix}
    \begin{pmatrix}
        -0.5 & 0.5 & -0.5 \\
        0.5  & 0.5 & 0.5  \\
        0    & 0   & 1
    \end{pmatrix}=                                              \\
             & =\begin{pmatrix}
        -1 & -1 & 1  \\
        1  & -1 & 0  \\
        0  & 0  & -1
    \end{pmatrix}
    \begin{pmatrix}
        -0.5 & 0.5 & -0.5 \\
        0.5  & 0.5 & 0.5  \\
        0    & 0   & 1
    \end{pmatrix}=
    \begin{pmatrix}
        0  & -1 & 1  \\
        -1 & 0  & 0  \\
        0  & 0  & -1
    \end{pmatrix}
\end{align*}

\pagebreak

\section*{שאלה 2}

\subsection*{סעיף א}

נניח בשלילה כי קיימת מטריצה $A$ מדרגה 3 עם פולינום אופייני $p(x)=x^7-x^5+x^3$.\\
משאלה $11.4.5$ נסיק כי $A$ מטריצה מסדר $7\times 7$ בהכרח, שכן אחרת הפולינום האופייני של המטריצה היה ממעלה שאינה $7$.\\
כמו כן, $\lambda=0$ שורש של הפולינום האופייני של $A$ ולכן ערך עצמי של המטריצה עם ריבוי אלגברי 3. \\\\
נדון בריבוי הגיאומטרי של $\lambda=0$. ערך זה שווה למימד מרחב הפתרונות של המשוואה $Ax=0$, וערכו, לפי $8.6.1$, יהיה $7-\rho(A)=4$. \\
מצד שני, לפי 11.5.3 ידוע לנו שהריבוי הגיאומטרי של $\lambda=0$ לא עולה על הריבוי האלגברי (במקרה זה 3) וזו סתירה!

\subsection*{סעיף ב}

נתונה לנו ה"ל $T: \reals^2\rightarrow\reals^2$ עם פולינום אופייני $x^2+2x-3$.
\\\\
עלינו להוכיח כי ההעתקה $3T+I$ היא איזומורפיזם $|3T+I|\ne 0$. \\
שורשי הפולינום האופייני של ההעתקה $T$, שהם $x_1=-3, x_2=1$, מהווים הערכים העצמיים היחידים של ההעתקה, זאת לפי $11.4.1$.
קרי, לכל $\lambda\ne -3, 1$ מתקיים $|\lambda I - T|\ne 0$, ובפרט עבור $\lambda=-\frac{1}{3}$ מקבלים $|-\frac{1}{3}I-T|\ne 0$.\\
מכאן נובע לפי שאלה 10.7.7 כי ההעתקה $-\frac{1}{3}I-T$ ומ$9.9.2$ נקבל ש$-\frac{1}{3}I-T$ איזומורפיזם.\\
קל להיווכח שההעתקה המבוקשת $3T+I$, שהיא כפל בסקלר של איזומורפיזם, היא איזומורפיזם בעצמה - תכונות העל והחח"ע מתקיימות באופן מיידי.
\\\\
נבנה את הפולינום האופייני של $T^3$ בעזרת מידע הידוע לנו על ההעתקה. לפי שאלה $11.3.2$ עבור הערכים העצמיים $\lambda=-3,1$ של $T$ נקבל ש$(-3)^3=-27, 1^3=1$ ערכים עצמיים של $T^3$.\\
לפי 11.2.6 לא ייתכנו ערכים עצמיים נוספים עבור $T^3$ מממד $2$, ולכן הערכים העצמיים שמצאנו הם שורשיו היחידים של הפולינום האופייני. כמו כן, לפי שאלה 11.4.5 הפולינום האופייני הוא פולינום מתוקן ממעלה $2$. משתי מסקנות אלה נקבל את הפולינום האופייני:
\begin{align*}
    p(t)=(t+27)(t-1)=t^2+26t-27
\end{align*}

\subsection*{סעיף ג}

נסמן $p(x)=|xI-A|$ ונדון בערכים העצמיים של המטריצה $A_{4\times 4}$ הנתונה. \\
נציין שלפי שאלה 11.4.5 מעלת הפולינום היא 4 בדיוק.
\\\\
תחילה, היות ו$A$ סינגולית, נקבל ש$\lambda=0$ הוא ערך עצמי של $A$ ושורש של הפולינום האופייני שלה. הריבוי הגיאומטרי של ערך עצמי זה הוא לכל הפחות 1 (אחרת לא היה ערך עצמי) ומכך נקבל לפי 11.5.3 כי הריבוי האלגברי של $\lambda=0$, שנסמנו ב$\alpha$, הוא לכל הפחות 1.
נסיק כי $p(x)$ מתחלק ב$x^\alpha$ ותוצאת החלוקה היא פולינום ממעלה $3\geq 4-\alpha$
\\\\
כעת, מהנתון $|-2I+A|=(-1)^4|2I-A|=|2I-A|=0$ נסיק כי $\lambda=2$ הוא ערך עצמי נוסף של המטריצה בעל ריבוי גיאומטרי של לכל הפחות 1.
נסמן את הריבוי האלגברי של ערך עצמי זה ב$\beta \geq 1$. מתוצאה זו והתוצאה לעיל נסיק כי $p$ מתחלק ב$x^\alpha(x-2)^\beta$ ותוצאת החילוק היא פולינום ממעלה $2\geq 4-\alpha-\beta$
\\\\
כעת, מהנתון $\rho(2I+A)=2$, כאשר $2I+A$ מטריצה מסדר 4, נקבל ש$2I+A$ אינה הפיכה. \\
בפרט $|2I+A|=(-1)^4|-2I-A|=|-2I-A|=0$ ולכן $\lambda=-2$ ערך עצמי של $A$. הריבוי הגיאומטרי של ערך עצמי זה הוא כמימד מרחב האפס של $(-2I-A)$, המתקבל לנו לפי $8.6.1$:
\begin{align*}
    \dim P(-2I-A)=4-\rho(-2I-A)=4-\rho(2I+A)=2
\end{align*}
נסמן את הריבוי האלגברי של הערך העצמי ב$\gamma\geq 2$. אז מתוצאה זו ותוצאות לעיל נסיק כי $p$ מתחלק ב$x^\alpha(x-2)^\beta(x+2)^\gamma$, ותוצאת החילוק תהיה פולינום ממעלה $0\geq 4-\alpha-\beta-\gamma$.
\\\\
היות ו$p$ אינו פולינום האפס, תוצאת החלוקה חייבת להיות פולינום ממעלה 0 ולכן $4-\alpha-\beta-\gamma=0$. מכך ש$\alpha,\beta, \gamma$ מספרים טבעיים ו-$\gamma\geq 2$ נקבל כי האפשרות היחידה לפתרון תהיה $\alpha=\beta=1, \gamma=2$. הפולינום $p$ הוא פולינום מתוקן ממעלה 4 המתחלק ב$x(x-2)(x+2)^2$ ומכאן נקבל $p(x)=x(x-2)(x+2)^2$.
\\\\
הערכים העצמיים $\lambda=0, \lambda=2$ בעלי ריבוי אלגברי 1 וריבוי גיאומטרי החסום מלמעלה (ע"י הריבוי האלגברי) ומלמטה (הסברנו לעיל) ע"י 1. הערך העצמי $\lambda=-2$ הוא בעל ריבוי אלגברי וגיאומטרי 2. נקבל לפי 11.5.4 כי $A$ לכסינה ובכך סיימנו את ההוכחה.
\pagebreak
\section*{שאלה 3}
תהא $A_{n\times n}$ לכסינה, כלומר קיימת מטריצה הפיכה $P$ ומטריצה אלכסונית $D$ כך ש $D=P^{-1}AP$ ומכאן ישירות $A=PDP^{-1}$.\\
נשים לב שלכל $k$ טבעי מתקיים:
\begin{align*}
    A^k=(PDP^{-1})^k & =(PDP^{-1})(PDP^{-1})\cdots(PDP^{-1})=                   \\
                     & =PD(P^{-1}P)D(P^{-1}P)\cdots (P^{-1}P)DP^{-1}=PD^kP^{-1}
\end{align*}
מסמנים את הפולינום האופייני של $A$ בתור $p(t)=\sum_{k=0}^n a_kt^k$ ומגדירים $p(A)=\sum_{k=0}^n a_kA^k$. אז:
\begin{align*}
    p(A) & =
    \sum_{k=0}^n a_k A^k=
    \sum_{k=0}^n a_k PD^kP^{-1}\equals_{\text{פילוג}}
    P \sum_{k=0}^n a_k D^k P^{-1}= Pp(D)P^{-1}
\end{align*}
היות ו$D$ אלכסונית, נסמן $
    D=\begin{pmatrix}
        \lambda_1 & 0         & \cdots & 0         \\
        0         & \lambda_2 & \cdots & 0         \\
        \vdots    & \vdots    & \ddots & \vdots    \\
        0         & 0         & \cdots & \lambda_n
    \end{pmatrix}
$ כך ש$\lambda_1, \lambda_2, ..., \lambda_n$ סקלרים. \\
סקלרים אלה הם איברי האלכסון במטריצה אלכסונית הדומה ל$A$, ולכן לפי $11.2.3$ג (בהתאמה למטריצות) נסיק כי אלו הם הערכים העצמיים של $A$ (לא בהכרח כולם שונים), ולכן מהווים שורשים של הפולינום האופייני!
\begin{align*}
    p(D)=\sum_{k=0}^n  a_k D^k & =\sum_{k=0}^n\left( a_k \begin{pmatrix}
            \lambda_1^k & 0           & \cdots & 0           \\
            0           & \lambda_2^k & \cdots & 0           \\
            \vdots      & \vdots      & \ddots & \vdots      \\
            0           & 0           & \cdots & \lambda_n^k
        \end{pmatrix} \right)= \\
                               & =\begin{pmatrix}
        a_0+a_1\lambda_1+\cdots+a_n\lambda_1^n & 0                                      & \cdots & 0                                      \\
        0                                      & a_0+a_1\lambda_2+\cdots+a_n\lambda_2^n & \cdots & 0                                      \\
        \vdots                                 & \vdots                                 & \ddots & \vdots                                 \\
        0                                      & 0                                      & \cdots & a_0+a_1\lambda_n+\cdots+a_n\lambda_n^n
    \end{pmatrix}=                                \\
                               & = \begin{pmatrix}
        p(\lambda_1) & 0            & \cdots & 0            \\
        0            & p(\lambda_2) & \cdots & 0            \\
        \vdots       & \vdots       & \ddots & \vdots       \\
        0            & 0            & \cdots & p(\lambda_n)
    \end{pmatrix}\equals_{p(\lambda_i)=0}
    0
\end{align*}
ולכן $p(A)=Pp(D)P^{-1}=0$

\pagebreak

\section*{שאלה 4}

\subsection*{סעיף א}

הטענה לא נכונה. ניקח למשל $A=\begin{pmatrix}
        0 & 0 \\
        0 & 0
    \end{pmatrix}, B=\begin{pmatrix}
        0 & 1 \\
        0 & 0
    \end{pmatrix}$ מדרגות שונות ונראה שיש להן אותו פולינום אופייני:
\begin{align*}
    p_A(x)=|xI-A| & =\left|x\cdot\begin{pmatrix}
        1 & 0 \\
        0 & 1
    \end{pmatrix} - \begin{pmatrix}
        0 & 0 \\
        0 & 0
    \end{pmatrix}\right|=
    \begin{vmatrix}
        x & 0 \\
        0 & x
    \end{vmatrix}=x^2                                                               \\
    p_B(x)=|xI-B| & =\left|x\cdot\begin{pmatrix}
        1 & 0 \\
        0 & 1
    \end{pmatrix} - \begin{pmatrix}
        0 & 1 \\
        0 & 0
    \end{pmatrix}\right|=
    \begin{vmatrix}
        x & -1 \\
        0 & x
    \end{vmatrix}=x^2                                                               \\
\end{align*}

\subsection*{סעיף ב}

לפי 11.3.6 $A=\begin{pmatrix}
        2         & -\sqrt{3} \\
        -\sqrt{3} & -2
    \end{pmatrix}$ דומה למטריצה האלכסונית $B=\begin{pmatrix}
        -1 & 0 \\
        0  & 1
    \end{pmatrix}$ אם $-1,1$ ערכים עצמיים של $A$, כלומר שורשים של הפולינום האופייני שלה. ואכן, הפולינום האופייני של $A$ הוא:
\begin{align*}
    p(x)=|xI-A|=\left|x\cdot\begin{pmatrix}
        1 & 0 \\
        0 & 1
    \end{pmatrix} - \begin{pmatrix}
        2        & -\sqrt{3} \\
        \sqrt{3} & -2
    \end{pmatrix}\right|=
    \begin{vmatrix}
        x-2       & \sqrt{3} \\
        -\sqrt{3} & x+2
    \end{vmatrix}=(x-2)(x+2)+3=x^2-1=(x-1)(x+1)
\end{align*}
שורשי הפולינום האופייני הם אכן $\pm 1$ והמטריצה לכסינה ודומה ל$B$.

\subsection*{סעיף ג}

הטענה שגויה. \\
נתחיל במציאות ערכים עצמיים למטריצות. הפולינומים האופייניים של המטריצות יהיו:
\begin{align*}
    p_A(x)=|xI-A| & =\begin{vmatrix}
        x+3 & -1  & -1  \\
        -1  & x+3 & -1  \\
        -1  & -1  & x+3
    \end{vmatrix}\equals^{R_1\rightarrow R_1+R_2+R_3}
    \begin{vmatrix}
        x+1 & x+1 & x+1 \\
        -1  & x+3 & -1  \\
        -1  & -1  & x+3
    \end{vmatrix}=
    (x+1)\begin{vmatrix}
        1  & 1   & 1   \\
        -1 & x+3 & -1  \\
        -1 & -1  & x+3
    \end{vmatrix}=                                                       \\
                  & \equals^{R_1\rightarrow R_1+R_2}
    (x+1)\begin{vmatrix}
        0  & x+4 & 0   \\
        -1 & x+3 & -1  \\
        -1 & -1  & x+3
    \end{vmatrix}=
    (x+1)\cdot (-1)(x+4)\begin{vmatrix}
        -1 & -1  \\
        -1 & x+3
    \end{vmatrix}=                                        \\
                  & = -(x+1)(x+4)((-1)(x+3)-(-1)(-1))=
    -(x+1)(x+4)(-(x+3)-1)=
    (x+1)(x+4)^2                                                                           \\
    p_B(x)=|xI-B| & =\begin{vmatrix}
        x+4 & 0   & 0   \\
        0   & x+1 & 0   \\
        0   & -1  & x+1
    \end{vmatrix}\equals_{\text{מטריצה משולשית}}(x+4)(x+1)^2
\end{align*}
למטריצות פולינומים אופייניים שונים ולכן לפי 11.4.3 אינן דומות.

\pagebreak

\section*{שאלה 5}

יהיו $u,v\in \reals^n$ שונים מאפס כך ש $||u||=||v||$. \\
צריך למצוא ערכי $a$ כך ש$(u+av)\cdot (u-av)=0$. נחשב:
\begin{align*}
    (u+av)\cdot (u-av) =
    u\cdot u + av \cdot u - av \cdot u - av \cdot av \equals_{\text{לינאריות}}
    ||u||^2 - a^2||v||^2\equals_{||u||=||v||}
    ||u||^2(1-a^2)
\end{align*}
היות ו$u\ne 0$ ובהתאם גם הנורמה שלו, מכפלה זו מתאפסת רק כאשר $1-a^2=0$, כלומר $a=\pm 1$.

\section{שאלה 6}

\subsection*{סעיף א}

ראשית, הקבוצה $U_1^\perp\cap U_2^\perp$ היא איחוד של תתי-מרחבים לינאריים של $\reals^n$ ולכן תת-מרחב לינארי בעצמה. בפרט, $\zerovec \in U_1^\perp\cap U_2^\perp$. עלינו להוכיח כי זהו הוקטור היחיד בתת-המרחב. \\
אז, יהא $v\in U_1^\perp\cap U_2^\perp$ ונוכיח כי $v=0$.
\\\\
מהנתון ש$\reals^n$ סכום ישר של $U_1$ ו$U_2$ נובע כי קיימים $u_1\in U_1, u_2\in U_2$ יחידים כך ש $v=u_1+u_2$.\\
היות ו$v\in U_1^\perp\cap U_2^\perp$ ובפרט שייך לשני ההיטלים האורתוגינליים של שני תתי-המרחבים, עבור $u_1, u_2$ מתקיים $u_1\cdot v= u_2\cdot v= 0$. לכן,
\begin{align*}
    ||v||^2=v \cdot v = (u_1+u_2)\cdot v \equals_{\text{לינאריות}}u_1\cdot v + u_2\cdot v = 0
\end{align*}
נקבל ש$||v||=0$ ומתכונת החיוביות של הנורמה נסיק $v=0$. בכך הוכחנו את הטענה.
\\\\
$U_1^\perp, U_2^\perp$ הם שני מרחבים של המרחב $\reals^n$ מממד $n$. נדון במימדים של שני תתי-מרחבים אלה. לפי משפט הפירוק האורתוגונלי, מתקיים:
\begin{align*}
    \dim U_1^\perp + \dim U_2^\perp \equals_{12.3.2} & = (n-\dim U_1) + (n-\dim U_2) =                                                 \\
                                                     & = 2n - (\dim U_1+\dim U_2) \equals_{8.3.7} 2n-\dim \reals^n = n = \dim \reals^n
\end{align*}
בצירוף המסקנה מקודם, נסיק לפי שאלה 8.3.12 כי $U_1^\perp\oplus U_2^\perp=\reals^n$.

\subsection*{סעיף ב}

הטענה לא נכונה. ניתן דוגמה נגדית ב$\reals^3$. ניקח:
\begin{align*}
    \begin{matrix}
        U_1=\Sp\{ (1,0,0), (0,1,0) \} & U_2 = \Sp\{ (1,0,0), (0,0,1) \}
    \end{matrix}
\end{align*}
אז מתקיים, לפי משפט המימדים $8.3.6$, כי:
\begin{align*}
    \dim (U_1+U_2) & =\dim U_1 + \dim U_2 - \dim(U_1\cap U_2)=                                                  \\
                   & = \dim \Sp\{ (1,0,0), (0,1,0) \} + \dim \Sp\{ (1,0,0), (0,0,1) \} - \dim \Sp\{ (1,0,0) \}= \\
                   & = 2 + 2 - 1 = 3
\end{align*}
ומכאן, לפי משפט $8.3.4$ב נסיק כי $U_1+U_2=\reals^3$.
\\\\
לא קשה להבין כי:
\begin{align*}
    \begin{matrix}
        U_1^\perp=\Sp\{ (0,0,1) \} & U_2^\perp = \Sp\{ (0,1,0) \}
    \end{matrix}
\end{align*}
סכום תתי-מרחבים אלה יהיה, לפי שאלה $7.6.8$, $U_1^\perp+U_2^\perp=\Sp \{ (0,0,1), (0,1,0) \}$. מימד מרחב זה הוא 2 ולכן, שוב לפי $8.3.4$ב, $U_1^\perp + U_2^\perp\ne \reals^3$.

\end{document}