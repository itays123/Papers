\documentclass{article}
\usepackage{fontspec}
\newfontfamily\hebrewfont[Script=Hebrew]{Calibri}
\usepackage{polyglossia}
\usepackage{amsmath, amssymb}
\usepackage[left=2.0cm, top=2.0cm, right=2.0cm, bottom=2.0cm]{geometry}
\usepackage{bidi}
\setdefaultlanguage{hebrew}
\setotherlanguage{english}

\title{מטלת מנחה 11 - אלגברה לינארית 2}
\author{328197462}
\date{31/03/2023}

\def\reals{\mathbb{R}}
\def\complex{\mathbb{C}}
\def\field{\mathbb{F}}
\DeclareMathOperator*{\equals}{=}
\DeclareMathOperator{\trace}{tr}
\DeclareMathOperator{\adj}{^\ast}
\DeclareMathOperator{\tra}{^t}
\DeclareMathOperator{\inv}{^{-1}}
\DeclareMathOperator{\perc}{^\perp}
\begin{document}
\maketitle

\section*{שאלה 1}
\subsection*{סעיף א}

נוכיח ישירות לפי הגדרה כי לכל $A,B\in M^\complex_{n\times n}$ מתקיים $(T_P A, B)=(A, T_{P\adj}, B)$. \\
אכן, יהיו $A,B$ מטריצות ונקבל:

\begin{align*}
    (T_P A, B)     & \equals^\text{הגדרה}
    \trace (B \adj P\inv AP)              \\
    (A, T_{P\adj}) & \equals^\text{הגדרה}
    \trace (((P \adj) \inv B P \adj)\adj A) \equals^{2.1.4}
    \trace (P B \adj P \inv A) \equals^{\ast}
    \trace (B \adj P \inv A P) = (T_P A, B)
\end{align*}

נימוק ל(*): לפי לינארית 1, $\trace (CD) = \trace (DC)$.

\subsection*{סעיף ב}

איברי הבסיס הסטנדרטי הם $
    E_{11} = \begin{pmatrix}
        1 & 0 \\
        0 & 0
    \end{pmatrix},
    E_{12} = \begin{pmatrix}
        0 & 1 \\
        0 & 0
    \end{pmatrix},
    E_{21} = \begin{pmatrix}
        0 & 0 \\
        1 & 0
    \end{pmatrix},
    E_{22} = \begin{pmatrix}
        0 & 0 \\
        0 & 1
    \end{pmatrix}
$. \\
כידוע מסעיף א, $(T_P) \adj = T_{P\adj}$. המטריצה $P\adj = \begin{pmatrix}
        i  & 1  \\
        -1 & -i
    \end{pmatrix} \adj =
    \begin{pmatrix}
        -i & -1 \\
        1  & i
    \end{pmatrix}$ \\
נחשב את המטריצה ההופכית $(P\adj) \inv$:
\begin{align*}
    ( P \adj | I ) & =
    \left(
    \begin{matrix}
            -i & -1 \\
            1  & i
        \end{matrix}
    \left|
    \begin{matrix}
            1 & 0 \\
            0 & 1
        \end{matrix}
    \right.
    \right)
    \xrightarrow[]{R_1\rightarrow iR_1}
    \left(
    \begin{matrix}
            1 & -i \\
            1 & i
        \end{matrix}
    \left|
    \begin{matrix}
            i & 0 \\
            0 & 1
        \end{matrix}
    \right.
    \right)
    \xrightarrow{R_2\rightarrow R_2-R_1}
    \left(
    \begin{matrix}
            1 & -i \\
            0 & 2i
        \end{matrix}
    \left|
    \begin{matrix}
            i  & 0 \\
            -i & 1
        \end{matrix}
    \right.
    \right)                                                        \\
    \cdots         & \xrightarrow{R_2\rightarrow -\frac 1 2 i R_2}
    \left(
    \begin{matrix}
            1 & -i \\
            0 & 1
        \end{matrix}
    \left|
    \begin{matrix}
            i          & 0            \\
            -\frac 1 2 & -\frac 1 2 i
        \end{matrix}
    \right.
    \right)
    \xrightarrow{R_1\rightarrow R_1+iR_2}
    \left(
    \begin{matrix}
            1 & 0 \\
            0 & 1
        \end{matrix}
    \left|
    \begin{matrix}
            \frac 1 2 i & \frac 1 2    \\
            -\frac 1 2  & -\frac 1 2 i
        \end{matrix}
    \right.
    \right) = ( I | (P \adj ) \inv)
\end{align*}
נחשב את תמונות ההעתקה $T_{P\adj}$ עבור איברי הבסיס הסטנדרטי.

\begin{align*}
    (T_{P\adj})E_{11} & = \begin{pmatrix}
                              \frac 1 2 i & \frac 1 2    \\
                              -\frac 1 2  & -\frac 1 2 i
                          \end{pmatrix} \begin{pmatrix}
                                            1 & 0 \\
                                            0 & 0
                                        \end{pmatrix} \begin{pmatrix}
                                                          -i & -1 \\
                                                          1  & i
                                                      \end{pmatrix}
    = \begin{pmatrix}
          \frac 1 2 i & 0 \\
          -\frac 1 2  & 0
      \end{pmatrix} \begin{pmatrix}
                        -i & -1 \\
                        1  & i
                    \end{pmatrix} =
    \begin{pmatrix}
        \frac 1 2   & -\frac 1 2 i \\
        \frac 1 2 i & \frac 1 2
    \end{pmatrix}                                       \\
    (T_{P\adj})E_{12} & = \begin{pmatrix}
                              \frac 1 2 i & \frac 1 2    \\
                              -\frac 1 2  & -\frac 1 2 i
                          \end{pmatrix} \begin{pmatrix}
                                            0 & 1 \\
                                            0 & 0
                                        \end{pmatrix} \begin{pmatrix}
                                                          -i & -1 \\
                                                          1  & i
                                                      \end{pmatrix}
    = \begin{pmatrix}
          0 & \frac 1 2 i \\
          0 & - \frac 1 2
      \end{pmatrix} \begin{pmatrix}
                        -i & -1 \\
                        1  & i
                    \end{pmatrix} =
    \begin{pmatrix}
        \frac 1 2 i & - \frac 1 2   \\
        - \frac 1 2 & - \frac 1 2 i
    \end{pmatrix}                                      \\
    (T_{P\adj})E_{21} & = \begin{pmatrix}
                              \frac 1 2 i & \frac 1 2    \\
                              -\frac 1 2  & -\frac 1 2 i
                          \end{pmatrix} \begin{pmatrix}
                                            0 & 0 \\
                                            1 & 0
                                        \end{pmatrix} \begin{pmatrix}
                                                          -i & -1 \\
                                                          1  & i
                                                      \end{pmatrix}
    = \begin{pmatrix}
          \frac 1 2    & 0 \\
          -\frac 1 2 i & 0
      \end{pmatrix} \begin{pmatrix}
                        -i & -1 \\
                        1  & i
                    \end{pmatrix} =
    \begin{pmatrix}
        - \frac 1 2 i & - \frac 1 2 \\
        \frac 1 2     & \frac 1 2 i
    \end{pmatrix}                                      \\
    (T_{P\adj})E_{22} & = \begin{pmatrix}
                              \frac 1 2 i & \frac 1 2    \\
                              -\frac 1 2  & -\frac 1 2 i
                          \end{pmatrix} \begin{pmatrix}
                                            0 & 0 \\
                                            0 & 1
                                        \end{pmatrix} \begin{pmatrix}
                                                          -i & -1 \\
                                                          1  & i
                                                      \end{pmatrix}
    = \begin{pmatrix}
          0 & \frac 1 2     \\
          0 & - \frac 1 2 i
      \end{pmatrix} \begin{pmatrix}
                        -i & -1 \\
                        1  & i
                    \end{pmatrix} =
    \begin{pmatrix}
        \frac 1 2     & \frac 1 2 i \\
        - \frac 1 2 i & \frac 1 2
    \end{pmatrix}
\end{align*}
ולכן נקבל:

\[
    [T_{P \adj}]_E = \frac 1 2 \begin{pmatrix}
        1  & i  & -i & 1  \\
        -i & -1 & -1 & i  \\
        i  & -1 & -1 & -i \\
        1  & -i & i  & 1
    \end{pmatrix}
\]

\pagebreak

\section*{שאלה 2}

נשים לב כי:

\begin{align*}
    U \adj & = (P+iQ) \adj \equals^{2.1.4} P \tra - iQ\tra & D \tra = \begin{pmatrix}
                                                                          P \tra   & Q \tra \\
                                                                          - Q \tra & P \tra
                                                                      \end{pmatrix}
\end{align*}

\subsection*{סעיף א}

נניח כי $U=U\adj$, כלומר $P+iQ=P\tra -iQ \tra$. \\
נשווה חלק ממשי וחלק מדומה. מקבלים $P=P \tra, Q = -Q \tra$, ולכן:
\[
    D \tra = \begin{pmatrix}
        P \tra   & Q \tra \\
        - Q \tra & P \tra
    \end{pmatrix}= \begin{pmatrix}
        P & -Q \\
        Q & P
    \end{pmatrix} = D
\]
ובכך השלמנו את ההוכחה.

\subsection*{סעיף ב}

נניח כי $U$ אוניטרית. כלומר:
\[
    I=U\cdot U \adj = (P+iQ)(P\tra - iQ \tra) \equals^\text{פילוג}
    PP \tra - iPQ \tra + iQP \tra + QQ\tra =
    (PP\tra + QQ\tra) + i(QP \tra - PQ \tra)
\]
נשווה חלק ממשי וחלק מדומה ונקבל $PP\tra + QQ\tra=I, QP \tra - PQ \tra=0$. לכן:
\[
    D \cdot D^t =\begin{pmatrix}
        P & -Q \\
        Q & P
    \end{pmatrix} \begin{pmatrix}
        P \tra   & Q \tra \\
        - Q \tra & P \tra
    \end{pmatrix} =
    \begin{pmatrix}
        PP\tra + QQ \tra & PQ\tra - QP\tra  \\
        QP\tra - PQ\tra  & QQ\tra + PP \tra
    \end{pmatrix} =
    \begin{pmatrix}
        I_{n\times n} & O_{n\times n} \\
        O_{n\times n} & I_{n\times n}
    \end{pmatrix} =
    I_{2n \times 2n}
\]
ולכן $D$ אורתוגונלית.


\pagebreak

\section*{שאלה 3}

\pagebreak

\section*{שאלה 4}

ראשית:
\begin{align*}
    H\adj  & = (I-2ww\adj) \adj \equals^{2.1.4}
    I - 2(w \adj) \adj w\adj = I - 2ww\adj = H            \\
    HH\adj & = (I - 2ww\adj)(I - 2ww\adj) \equals^{פילוג}
    I - 4ww\adj + 4 (ww\adj)^2
\end{align*}
נדרוש $I=HH\adj$. נקבל $4(ww\adj)^2 - 4ww\adj=0$, ולכן $ww\adj = (ww\adj)^2$. \\
נשים לב כי:
\[
    (ww\adj)^2=(ww\adj) (w w\adj)\equals^{קיבוציות}w(w\adj w)w\adj=w||w||^2w\adj = ||w||^2 ww\adj
\]
נציב את שתי המסקנות האחרונות שלנו ביחד, ונקבל $ww\adj = ||w||^2ww\adj$, כלומר $||w||^2=1$ ולכן $||w||=1$ מתכונת החיוביות.\\\\
מצאנו תנאי הכרחי עבור $w$: אם $H$ אוניטרית, אז $||w||=1$. נוכיח כי זהו תנאי מספיק.\\
עבור $||w||=1$,
\[
    HH\adj = I - 4ww\adj + 4 (ww\adj)^2=I-4ww\adj + 4||w||^2ww\adj \equals^{||w||=1}I
\]
\\
נוכיח את תכונת השיקוף.
\begin{align*}
    Hw=(I-2ww\adj)w=Iw-2ww\adj w \equals^{קיבוציות} w - 2||w||^2w=-w
\end{align*}
יהא $v\in {w}\perc$. אז $<v,w>=w\adj v = 0$, ומקבלים:
\begin{align*}
    Hv=(I-2ww\adj)v=Iv-2ww\adj v = v - 2w\cdot 0 = v
\end{align*}

\end{document}