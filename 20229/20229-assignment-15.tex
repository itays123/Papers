\documentclass{article}
\usepackage{fontspec}
\newfontfamily\hebrewfont[Script=Hebrew]{Calibri}
\usepackage{polyglossia}
\usepackage{amsmath, amssymb}
\usepackage[left=2.0cm, top=2.0cm, right=2.0cm, bottom=2.0cm]{geometry}
\usepackage{bidi}
\setdefaultlanguage{hebrew}
\setotherlanguage{english}

\title{מטלת מנחה 15 - אלגברה לינארית 2}
\author{328197462}
\date{02/06/2023}

\def\reals{\mathbb{R}}
\def\complex{\mathbb{C}}
\def\field{\mathbb{F}}
\DeclareMathOperator*{\equals}{=}
\DeclareMathOperator{\trace}{tr}
\DeclareMathOperator{\adj}{^\ast}
\DeclareMathOperator{\tra}{^t}
\DeclareMathOperator{\inv}{^{-1}}
\DeclareMathOperator{\perc}{^\perp}
\DeclareMathOperator{\Sp}{Sp}
\DeclareMathOperator{\Image}{Im}
\DeclareMathOperator{\diag}{diag}

\begin{document}
\maketitle

\section*{שאלה 1}

\subsection*{סעיף א}

\subsection*{סעיף ב}

תהא $T$ העתקה כמוגדר.
יהא $U\subseteq V$ תת-מרחב של $V$ ממימד 1. על פי הנתון, $U$ תת-מרחב $T$-שמור. \\
כלומר, קיים $u_0\in U$ כך שלכל $u\in U$ מתקיים $Tu=\lambda u_0$. בפרט, $Tu_0=\alpha u_0$ עבור $\alpha\in \field$ כלשהו.\\\\
נבחר ערך $\alpha$ זה ונוכיח כי $T=\alpha I$. \\
עבור $u\in U$ מקבלים $Tu=T(\lambda u_0)= \lambda Tu_0 = \alpha \cdot \lambda u_0=\alpha u$. \\
נבחר אם כן $v\in V - U$ ונוכיח כי $Tv=\alpha v$. \\
נתבונן בתת-המרחב $W=\Sp\{v\}$. תת-מרחב זה הוא $T$-שמור, לכן $Tv=\beta v$. \\
נתבונן בתת-המרחב $W'=\Sp\{ u_0+v \}$. שוב, מתקיים $T(u_0+v)=\gamma \cdot (u_0+v)=\gamma u_0+\gamma v$.\\
מצד שני, $T(u_0+v)=Tu_0+Tv=\alpha u_0 + \beta v$. \\
הוקטורים $u_0,v$ בלתי-תלויים לינארית (אינם פרופורציוניים), לכן לוקטור $T(u_0+v)$ יש הצגה יחידה כקומבינציה לינארית של $u_0$ ו-$v$. \\
מכאן נסיק $\alpha=\gamma=\beta$ ולכן $Tv=\alpha v$ והשלמנו את מלאכת ההוכחה.

\pagebreak

\section*{שאלה 2}

\subsection*{סעיף א}

נסמן ב $m(x)$ את הפולינום המינימלי של $T_W$, וב$M(x)$ את הפולינום המינימלי של $T$. עלינו להוכיח כי $m$ מחלק את $M$.\\
על פי הגדרה, $T$ מאפסת את $M$. מכאן שלכל $v\in V$, $M(T)v=0$, ובפרט עבור $v\in W$.
כמו כן, לכל $v\in W$ מקבלים $T_Wv=Tv$, ולכן $M(T_W)v=0$. \\
קיבלנו כי $M$ מאפסת את $T_W$. לכן, משאלה 9.9.1א, $m$ מחלק את $M$. \\\\
כעת נניח כי ההעתקה $T$ לכסינה. לפי 10.2.11 בהתאמה למטריצות, נקבל כי $M(x)=(x-\lambda_1)(x-\lambda_2)\cdots(x-\lambda_k)$ כאשר הסקלארים $\lambda_i$ שונים זה מזה. \\
היות ו$m$ מחלק את $M$, $m$ הוא מכפלת חלק או כל הגורמים הלינאריים $x-\lambda_i$ השונים זה מזה ומחלקים את $M$, ולכן לפי 10.2.11 ההעתקה $T_W$ לכסינה.

\subsection*{סעיף ב}

\pagebreak

\section*{שאלה 4}

עלינו למצוא פירוק של הפולינום המינימלי של $T|_W$, שנסמנו $M_W$, ל $k$ פולינומים זרים בזוגות $P_1, P_2, ..., P_k$ כך שלכל $i$, $\ker P_i(T|_W)=W\cap W_i$. \\
כמו כן, על פי שאלה 2 במטלה זו, הפולינום המינימלי של $T|_W$ מחלק את $M(t)$. היות והפולינומים $M_1, M_2, ..., M_k$ זרים בזוגות, המשמעות היא שכל גורם בכל פירוק של $M_W$ יחלק אחד בדיוק מבין סדרת פולינומים אלו (אחרת, יהיה להם מחלק משותף שאינו 1). \\
נסמן אפוא ב $M_W=p_1\cdot p_2\cdots p_m$ את הפירוק המקסימלי של $M_w$, ונבחר את $P_i$ להיות מכפלת כל הפולינומיים האי-פריקים $p_j$ המחלקים את $M_i$.
ברור כי כל פולינום אי-פריק $p_j$ יהיה גורם במכפלה אחת בדיוק, ולכן $M_W=P_1\cdot P_2\cdot ... \cdot P_k$. \\\\
נסמן $U_i=\ker P_i(T_W)$. לפי הפירוק הפרימרי, $W=U_1\oplus U_2\oplus\cdots\oplus U_k$. \\
נוכיח כי $U_i\subseteq W\cap W_i$. יהא $w\in U_i$. ברור כי $w\in W$, שכן $U_i\subseteq W$. עלינו להראות כי $w\in W_i=\ker M_i(T)$. \\
נניח בשלילה כי $M_i(T)w=M_i(T_W)w\ne 0$. היות ו$P_i$ מחלק את $M_i$, נקבל גם $P_i(T_W)w\ne 0$ ולכן $w\notin U_i$ וזו סתירה! \\\\
נקבל מצד אחד כי $W=U_1\oplus U_2\oplus\cdots\oplus U_k\subseteq (W\cap W_1)+(W\cap W_2)+\cdots + (W\cap W_k)$. \\
מצד שני, $(W\cap W_1)+(W\cap W_2)+\cdots+(W\cap W_k)\subseteq W$ ולכן מתקיים שוויון. \\
נוסיף כי הקבוצות $(W\cap W_1),(W\cap W_2), ...$ זרות: אם יש איבר משותף בין שתי קבוצות כלשהן בסכום, אז בפרט קיימים $i,j$ כך ש $W_i\cap W_j\ne \{0\}$, בסתירה לסכום הישר בפירוק הפרימרי של $V$!\
אי-לכך המרחב $W$ הוא סכום ישר של המרחבים $(W\cap W_i)$.

\pagebreak

\section*{שאלה 5}

תהא $T$ העתקה נורמלית במרחב אוניטרי ויהא $W$ תת-מרחב $T$-שמור. \\
לפי משפט הלכסון האוניטרי, ההעתקה $T$ לכסינה, ולפי שאלה 2 בממן זה גם הצמצום שלה $T_W$ מהווה העתקה לכסינה. \\
אי-לכך, קיים בסיס $(w)=(w_1, w_2, ..., w_k)$ של $W$ המורכב מוקטורים עצמיים. \\
וקטורים אלה, על פי למה 3.2.5, הם גם וקטורים עצמיים של $T_W\adj$, השייכים לערכים העצמיים $\lambda_1, \lambda_2, ..., \lambda_k$.

\end{document}