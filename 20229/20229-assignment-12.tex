\documentclass{article}
\usepackage{fontspec}
\newfontfamily\hebrewfont[Script=Hebrew]{Calibri}
\usepackage{polyglossia}
\usepackage{amsmath, amssymb}
\usepackage[left=2.0cm, top=2.0cm, right=2.0cm, bottom=2.0cm]{geometry}
\usepackage{bidi}
\setdefaultlanguage{hebrew}
\setotherlanguage{english}

\title{מטלת מנחה 12 - אלגברה לינארית 2}
\author{328197462}
\date{14/04/2023}

\def\reals{\mathbb{R}}
\def\complex{\mathbb{C}}
\def\field{\mathbb{F}}
\DeclareMathOperator*{\equals}{=}
\DeclareMathOperator{\trace}{tr}
\DeclareMathOperator{\adj}{^\ast}
\DeclareMathOperator{\tra}{^t}
\DeclareMathOperator{\inv}{^{-1}}
\DeclareMathOperator{\perc}{^\perp}
\DeclareMathOperator{\Sp}{Sp}
\DeclareMathOperator{\Image}{Im}
\begin{document}
\maketitle

\section*{שאלה 1}
\subsection*{סעיף א}

המטריצה $A_1$ צמודה לעצמה, ובפרט נורמלית.
\begin{align*}
    A_2 \adj    & = \begin{pmatrix}
                        -i  & 0 \\
                        -2i & 1
                    \end{pmatrix},
    A_3 \adj = \begin{pmatrix}
                   1  & 1   \\
                   -i & 2-i
               \end{pmatrix}                                    \\
    A_2A_2 \adj & = \begin{pmatrix}
                        5    & \ast \\
                        \ast & \ast
                    \end{pmatrix}, A_2 \adj A_2 = \begin{pmatrix}
                                                      1    & \ast \\
                                                      \ast & \ast
                                                  \end{pmatrix} \\
    A_3A_3 \adj & = \begin{pmatrix}
                        2    & 2+2i \\
                        2-2i & 6
                    \end{pmatrix} = A_3 \adj A_3
\end{align*}

לכן $A_2$ לא נורמלית ו $A_3$ נורמלית. נמצא מטריצות אוניטריות המלכסנות את $A_1, A_3$.
\begin{align*}
    p_1(x)=|xI-A_1| & =\begin{vmatrix}
                           x & -i \\
                           i & x
                       \end{vmatrix} = x^2-(-i)i = x^2 -1 = (x-1)(x+1) \\
    p_3(x)=(xI-A_3) & = \begin{vmatrix}
                            x-1 & -i      \\
                            -1  & x-(2+i)
                        \end{vmatrix} = (x-1)(x-(2+i))-(-i)(-1) =      \\
                    & = x^2-(2+i)x-x+2+i-i= x^2-(3+i)x+2
\end{align*}
עבור $A_1$, נמצא בסיסים אורתונורמליים למרחבים העצמיים $V_{\lambda=1}, V_{\lambda=-1}$.

\begin{itemize}
    \item עבור $V_{\lambda=1}$ מדובר במרחב האפס של המטריצה: \\
          \[
              I-A=\begin{pmatrix}
                  1 & -i \\
                  i & 1
              \end{pmatrix}\xrightarrow{R_2\rightarrow R_2-iR_1}
              \begin{pmatrix}
                  1 & -i \\
                  0 & 0
              \end{pmatrix}
          \]
          נקבל $V_{\lambda=1}=\Sp(\{ (i, 1) \})$.
          ננרמל ונקבל $B_{\lambda=1}=\{ \frac{i}{\sqrt{2}}, \frac{1}{\sqrt{2}} \}$
    \item עבור $V_{\lambda=-1}$ מדובר במרחב האפס של המטריצה: \\
          \[
              -I-A=\begin{pmatrix}
                  -1 & -i \\
                  i  & -1
              \end{pmatrix}\xrightarrow{R_1\rightarrow R_2+iR_1}
              \begin{pmatrix}
                  -1 & -i \\
                  0  & 0
              \end{pmatrix}\xrightarrow{R_1\rightarrow -R_1}
              \begin{pmatrix}
                  1 & i \\
                  0 & 0
              \end{pmatrix}
          \]
          נקבל $V_{\lambda=-1}=\Sp(\{ (1, i) \})$.
          ננרמל ונקבל $B_{\lambda=-1}=\{ \frac{1}{\sqrt{2}}, \frac{i}{\sqrt{2}} \}$
\end{itemize}
מטריצה אוניטרית המלכסנת את $A_1$ תהיה $Q=\frac{1}{\sqrt{2}}\begin{pmatrix}
        i & 1 \\
        1 & i
    \end{pmatrix}$ ומקבלים $Q\adj A Q = \begin{pmatrix}
        1 & 0  \\
        0 & -1
    \end{pmatrix}$ \\
עבור $A_3$, שורשי הפולינום האופייני יהיו: $\frac{(3+i)\pm \sqrt{(3+i)^2-4\cdot 2}}{2}=\frac{(3+i)\pm \sqrt{6i}}{2}=\frac{-(3+i)\pm \sqrt{3}(1+i)}{2}$ \\
נסמן $\lambda_1=\frac{3+\sqrt{3}}{2}+\frac{1+\sqrt{3}}{2}i, \lambda_2=\frac{3-\sqrt{3}}{2}+\frac{1-\sqrt{3}}{2}i$. המרחבים העצמיים יהיו:
\begin{itemize}
    \item עבור $V_{\lambda_1}$ מדובר במרחב האפס של המטריצה: \\
          \[
              \lambda_1I-A=\begin{pmatrix}
                  \frac{1+\sqrt{3}}{2}+\frac{1+\sqrt{3}}{2}i & -i                                           \\
                  -1                                         & \frac{-1+\sqrt{3}}{2}+\frac{-1+\sqrt{3}}{2}i
              \end{pmatrix}\xrightarrow{R_1\rightarrow \frac{-1+\sqrt{3}}{2}+\frac{1-\sqrt{3}}{2}i}
              \begin{pmatrix}
                  1  & \frac{1-\sqrt{3}}{2} + \frac{1-\sqrt{3}}{2}i \\
                  -1 & \frac{-1+\sqrt{3}}{2}+\frac{-1+\sqrt{3}}{2}i
              \end{pmatrix}
              \rightarrow
              \begin{pmatrix}
                  1 & \frac{1-\sqrt{3}}{2} + \frac{1-\sqrt{3}}{2}i \\
                  0 & 0
              \end{pmatrix}
          \]
          נקבל $V_{\lambda_1}=\Sp\{ (0.366 + 0.366i, 1) \}=\Sp\{ \frac{1}{1.126}(0.366 + 0.366i, 1) \}$ ו$B_1=\{(0.325+0.325i, 0.888)\}$ בסיס א"נ ל $V_{\lambda_1}$
    \item עבור $V_{\lambda_2}$ מדובר במרחב האפס של המטריצה: \\
          \[
              \lambda_2I-A=\begin{pmatrix}
                  \frac{1-\sqrt{3}}{2}+\frac{1-\sqrt{3}}{2}i & -i                                           \\
                  -1                                         & \frac{-1-\sqrt{3}}{2}+\frac{-1-\sqrt{3}}{2}i
              \end{pmatrix}\xrightarrow{R_1\rightarrow \frac{-1-\sqrt{3}}{2}+\frac{1+\sqrt{3}}{2}i}
              \begin{pmatrix}
                  1  & \frac{1+\sqrt{3}}{2} + \frac{1+\sqrt{3}}{2}i \\
                  -1 & \frac{-1-\sqrt{3}}{2}+\frac{-1-\sqrt{3}}{2}i
              \end{pmatrix}
              \rightarrow
              \begin{pmatrix}
                  1 & \frac{1+\sqrt{3}}{2} + \frac{1+\sqrt{3}}{2}i \\
                  0 & 0
              \end{pmatrix}
          \]
          נקבל $V_{\lambda_2}=\Sp\{ (-1.366 -1.366i, 1) \}=\Sp\{ \frac{1}{2.1753}(-1.366 -1.366i, 1) \}$ ו$B_2=\{(-0.628-0.628i, 0.4597)\}$ בסיס א"נ ל $V_{\lambda_2}$
\end{itemize}
מטריצה אוניטרית המלכסת את $A_3$ תהיה $P=\begin{pmatrix}
        0.325+0.325i & -0.628-0.628i \\
        0.888        & 0.4597
    \end{pmatrix}$ ונקבל $P\adj A P=\begin{pmatrix}
        \frac{3+\sqrt{3}}{2}+\frac{1+\sqrt{3}}{2}i & 0                                          \\
        0                                          & \frac{3-\sqrt{3}}{2}+\frac{1-\sqrt{3}}{2}i
    \end{pmatrix}$

\subsection*{סעיף ב}

תנאי הכרחי לחיוביות (לחלוטין) של מטריצות הוא היותן צמודות לעצמן. המטריצות $C_3, C_4, C_6$ אינן צמודות לעצמן:
\begin{align*}
    C_3\adj = \begin{pmatrix}
                  0 & -1 \\
                  1 & 0
              \end{pmatrix}\ne C_3 \;\;\;
    C_4\adj = \begin{pmatrix}
                  1 & 0 \\
                  1 & 1
              \end{pmatrix}\ne C_4 \;\;\;
    C_6 \adj = \begin{pmatrix}
                   1 & 3 \\
                   2 & 1
               \end{pmatrix}\ne C_6
\end{align*}
המטריצות $C_1, C_2, C_5$ צמודות לעצמן ובפרט נורמליות. תנאי הכרחי ומספיק להיותן חיוביות (לחלוטין) נתון לנו לפי משפט 3.3.2: המטריצות חיוביות (לחלוטין) אם ורק אם ערכיהם העצמיים אי-שליליים (חיוביים). נחשב פ"א:
\begin{align*}
    p_1(x)=|xI-C_1| & =\begin{vmatrix}
                           x-1 & -1  \\
                           -1  & x-1
                       \end{vmatrix}=x^2-2x+1-1=x(x-2)             \\
    p_2(x)=|xI-C_2| & =\begin{vmatrix}
                           x & -i \\
                           i & x
                       \end{vmatrix}=x^2-1=(x-1)(x+1)              \\
    p_5(x)=|xI-C_5| & =\begin{vmatrix}
                           x-2 & -1  \\
                           -1  & x-2
                       \end{vmatrix}=(x-2)^2-1=x^2-4x+3=(x-3)(x-1)
\end{align*}
עבור $C_1$ נקבל ע"ע $\lambda_1=0, \lambda_2=2$, והמטריצה אי-שלילית (אך לא חיובית לחלוטין).\\
עבור $C_2$ נקבל ע"ע $\lambda=-1$, קיומו של ע"ע שלילי מצביע על כך שהמטריצה לא חיובית (וגם לא לחלוטין).
עבור $C_5$ נקבל ע"ע $\lambda_1=3, \lambda_2=1$, והמטריצה חיובית (לחלוטין).


\pagebreak

\section*{שאלה 2}

תהא $T$ נורמלית במרחב מכפלה פנימית מממד סופי. \\
נוכיח: \\
(i) נוכיח בעזרת הכלה דו-כיוונית: \\
יהא $v\in \ker T$.
\[
    0=(Tv,Tv)\equals^{\text{תכונות צמוד}}(v, T\adj T v)\equals^{TT\adj = T \adj T}(v, TT\adj v)=(T \adj v, T \adj v)
\]
ומתכונת החיוביות נסיק $T \adj v=0$. ההכלה בכיוון השני שקולה לחלוטין. \\\\
(ii) נוכיח בעזרת הכלה ושוויון מימדים.
יהא $u\in \Image T, v\in \ker T$. עבור $u$, קיים $w\in V: Tw=u$. נקבל:
\[
    (u,v)=(Tw, v)=(w, T \adj v)\equals^{(i)}(w,0)=0
\]
קיבלנו $\Image T \subseteq (\ker T) \perc$.
אולם, מלינארית 1 ידוע לנו כי $\dim \Image T + \dim \ker T = \dim V = \dim \ker T + \dim (\ker T) \perc$. \\
קיבלנו שוויון מימדים ולכן $\Image T = (\ker T)\perc$.\\\\
(iii) נוכיח בעזרת התכונות שהוכחנו קודם:
\[
    \Image T \equals^{(ii)} (\ker T) \perc \equals^{(i)} (\ker T \adj) \perc \equals^{(ii)} \Image T \adj
\]

\section*{שאלה 3}

נבודד את $T \adj$ מהשוויון הנתון:
\begin{align*}
    T^2                & = \frac{1}{2} (T + T \adj) \Leftrightarrow \\
    \frac{1}{2} T \adj & = T^2 - \frac{1}{2} T \Leftrightarrow      \\
    T \adj             & = 2T^2 - T
\end{align*}

נראה כי $T$ מתחלפת עם הצמודה לה:
\begin{align*}
    TT\adj   & = T(2T^2-T)=2T^3-T^2          \\
    T \adj T & = (2T^2-T)T=2T^3-T^2 = TT\adj
\end{align*}
הוכחנו כי $T$ נורמלית. נראה כ $T$ צמודה לעצמה:
יהא $\lambda$ שורש (ממשי או מרוכב) כלשהו של הפולינום האופייני של $T$. נוכיח כי בהכרח $\lambda$ ממשי.
יהא $B$ בסיס א"נ כלשהו של $V$, ותהא $A\in M_n(\complex)$ מטריצה כך ש $[T]_B=A$. \\
משוויון הפולינומים האופייניים של $T$ ו-$A$ נסיק כי $\lambda$ שורש של הפולינום האופייני של $A$. בגלל ש$A$ מרוכבת, $\lambda$ ערך עצמי של $A$. \\
לפי לינארית 1, $v$ וקטור עצמי של $A^2$ השייך ל$\lambda^2$. \\
נפעיל את ההעתקה $[\cdot]_B$ על השוויון שמצאנו לעיל:
\[
    A \adj \equals^{2.1.3} [T \adj]_B = [2T^2-T]_B = 2[T]_B^2-[T]_B = 2A^2-A
\]
לכן, $A \adj v = (2A^2-A)v=(2\lambda^2-\lambda)v$.
בנוסף, לפי למה 3.2.5 בתרגום למטריצות, $A \adj v = \overline{\lambda}v$. נשווה ונקבל $(\lambda^2-\lambda-\overline{\lambda})v=0$. היות ו $v\ne 0$, בהכרח מקבלים $\lambda^2-\lambda-\overline{\lambda}=0$. נראה כי ממשיותו של $\lambda$ הכרחית לקיום שוויון זה.\\
נסמן אפוא $\lambda=x+iy$.
\begin{align*}
    \lambda^2-\lambda-\overline{\lambda} & =(x+iy)^2-(x+iy)-(x-iy) =            \\
                                         & = x^2 + 2ixy - y^2 -x - iy -x + iy = \\
                                         & = (x^2-2x-y^2)+(2xy)i = 0            \\
                                         & \rightarrow \begin{cases}
                                                           x^2-2x-y^2=0 \\
                                                           xy = 0
                                                       \end{cases}
\end{align*}
מהשוויון השני, בהכרח $x=0$ או $y=0$. נניח כי $x=0$ ונציב בשוויון הראשון. נקבל $-y^2=0$ ובשני המקרים $y=0$ ו$\lambda$ מספר ממשי. \\
אי-לכך, ממשפט 3.3.1 מקבלים ש$T$ צמודה לעצמה, ולכן:
\[
    T^2=\frac{1}{2}(T+T\adj) = \frac{1}{2}(T+T) = T
\]

\pagebreak

\section*{שאלה 4}

תהא מטריצה $H$ ממשית סימטרית. לפי משפט הלכסון האורתוגונלי, $H$ לכסינה אורתוגונלית, ונסמן ב $Q$ מטריצה אורתוגונלית המלכסנת אותה.\\
נסמן את עמודות $Q$ ב $B=\{v_1, v_2, ..., v_n\}$. אלו $n$ וקטורים עצמיים של $H$, ולכל אחד מהם נסמן $Hv_i=\lambda_iv_i$. \\
ע"פ משפט 2.3.6, $B$ בסיס אורתונורמלי ל$\reals^n$. יהא $v\in \reals^n$ כך ש $||v||=1$. נציג את $v$ כצירוף לינארי של וקטורי $B$:
\begin{align*}
    v & =\sum_{i=1}^{n}a_iv_i & Hv =\sum_{i=1}^{n} Ha_iv_i = \sum_{i=1}^{n} a_i \lambda_i v_i
\end{align*}
נחשב:
\begin{align*}
    1=||v||^2=\langle v, v \rangle    & = \langle \sum_{i=1}^{n}a_iv_i, \sum_{i=1}^{n}a_iv_i \rangle \equals^{\text{לינאריות}}
    \sum_{i=1}^{n}\sum_{j=1}^{n}a_ia_j \langle v_i , v_j \rangle \;
    \equals^{\langle v_i, v_j \rangle=0}_{\langle v_i, v_i \rangle = 1} \;\;
    \sum_{i=1}^{n}a_i^2                                                                                                                  \\
    v \tra Hv = \langle Hv, v \rangle & = \langle \sum_{i=1}^{n}a_iv_i, \sum_{i=1}^{n}a_i\lambda_i v_i \rangle \equals^{\text{לינאריות}}
    \sum_{i=1}^{n}\sum_{j=1}^{n}a_i a_j\lambda_j\langle v_i, v_j \rangle \;
    \equals^{\langle v_i, v_j \rangle=0}_{\langle v_i, v_i \rangle = 1} \;\;
    \sum_{i=1}^{n} \lambda_i a_i^2 \leq
    \sum_{i=1}^{n} \lambda a_i^2 =
    \lambda \sum_{i=1}^{n}  a_i^2 = \lambda
\end{align*}

\section*{שאלה 5}

עבור $A=\begin{pmatrix}
        2-i & -1  & 0   \\
        -1  & 1-i & 1   \\
        0   & 1   & 2-i
    \end{pmatrix}$ מקבלים $A\adj = \begin{pmatrix}
        2+i & -1  & 0   \\
        -1  & 1+i & 1   \\
        0   & 1   & 2+i
    \end{pmatrix}$. נחשב:
\begin{align*}
    AA \adj = \begin{pmatrix}
                  6  & -3 & -1 \\
                  -3 & 4  & 3  \\
                  -1 & 3  & 6
              \end{pmatrix} = A \adj A
\end{align*}
ולכן $A$ נורמלית. \\
נסמן ב $E=(e_1, e_2, e_3)$ את הבסיס הסטנדרטי ל $V=\complex^3$ ונגדיר את ההעתקה $T_A: V \rightarrow V$ כך שלכל $v\in V$, $T_A(v)=Av$. \\
אז $E$ בסיס א"נ ומתקיים $[T_A]_E=A$, לכן לפי שאלה 3.1.4 $A$ נורמלית $T_A\Leftarrow$ נורמלית.\\
ההעתקה $T_A$ נורמלית מעל $\field=\complex$, לכן ניתן להשתמש במשפט הפירוק הספקטרלי 3.4.2. נמצא את הפולינום האופייני של $T_A$:
\begin{align*}
    p(x)=\det(xI-[T_A]_E) & =\begin{vmatrix}
                                 x-(2-i) & 1       & 0       \\
                                 1       & x-(1-i) & -1      \\
                                 0       & -1      & x-(2-i)
                             \end{vmatrix}
\end{align*}

\end{document}