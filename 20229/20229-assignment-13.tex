\documentclass{article}
\usepackage{fontspec}
\newfontfamily\hebrewfont[Script=Hebrew]{Calibri}
\usepackage{polyglossia}
\usepackage{amsmath, amssymb}
\usepackage[left=2.0cm, top=2.0cm, right=2.0cm, bottom=2.0cm]{geometry}
\usepackage{bidi}
\setdefaultlanguage{hebrew}
\setotherlanguage{english}

\title{מטלת מנחה 13 - אלגברה לינארית 2}
\author{328197462}
\date{28/04/2023}

\def\reals{\mathbb{R}}
\def\complex{\mathbb{C}}
\def\field{\mathbb{F}}
\DeclareMathOperator*{\equals}{=}
\DeclareMathOperator{\trace}{tr}
\DeclareMathOperator{\adj}{^\ast}
\DeclareMathOperator{\tra}{^t}
\DeclareMathOperator{\inv}{^{-1}}
\DeclareMathOperator{\perc}{^\perp}
\DeclareMathOperator{\Sp}{Sp}
\DeclareMathOperator{\Image}{Im}
\DeclareMathOperator{\diag}{diag}

\begin{document}
\maketitle

\section*{שאלה 1}

לאורך שאלה זו נשתמש בתכונות הבאות של תבנית ההעתקה: $\trace(A+B)=\trace A + \trace B$ וכן $\trace(A\tra) = \trace A$.\

\subsection*{סעיף א}

נוכיח כי $f$ סימטרית אם ורק אם $M$ סימטרית. \\
כיוון ראשון: נניח כי $M$ סימטרית ונוכיח כי $f$ סימטרית. לכל $A,B\in V$ נקבל אם כן:
\begin{align*}
    f(A,B) & =\trace(A\tra M B)                                                                                                         \\
    f(B,A) & = \trace(B \tra M A) = \trace ((B \tra M A)\tra) = \trace(A \tra M \tra B) \equals^{M\tra = M} \trace(A \tra M B) = f(A,B)
\end{align*}
כיוון שני: נניח כי $f$ סימטרית. בפרט נבחר $A=I, B=(M-M\tra)\tra$ ונקבל $f(A,B)=f(B,A)$.
\begin{align*}
    f(A,B) & = \trace(I \tra M (M-M\tra)\tra)=\trace(M (M-M\tra)\tra)                                          \\
    f(B,A) & \equals^{\text{פיתוח לעיל}} \trace (I \tra M \tra (M-M\tra)\tra) = \trace(M \tra (M-M \tra) \tra)
\end{align*}
על פי השוויו, נסיק $f(A,B)-f(B,A)=0$ ולכן:
\begin{align*}
    f(A,B)-f(B,A)=\trace(M (M-M\tra)\tra)-\trace(M \tra (M-M\tra)\tra)=\trace((M-M\tra)(M-M\tra)\tra)=||M-M\tra||^2
\end{align*}
ומתכונת החיוביות של המכפלה הפנימית, השוויון $||M-M\tra||=0$ גורר $M-M \tra = 0$ ולכן $M=M\tra$.

\subsection*{סעיף ב}

נסמן את הבסיס הסטנדרטי של $V$ ב $E=(E_{11}, E_{12}, E_{21}, E_{22})$, וכן נסמן $M=[m_ij]$ \\
נשים לב כי $(E_{ij})\tra = E_{ji}$. נרצה לחשב בצורה כללית כל אחת מ16 המכפלות.



\subsection*{סעיף ג}

ניזכר בטענה חשובה מלינארית 1: מרחב המטריצות הממשיות מסדר $n\times n$ הוא סכום ישר של מרחב המטריצות הסימטריות עם מרחב המטריצות האנסימטריות מאותו סדר.
לכן, קיימות ויחידות $M_1$ סימטרית ו$M_2$ אנסיטמרית כך ש $M_1+M_2=M$. \\
במקרה שלנו $M=\begin{pmatrix}
        1   & 2.5 \\
        2.5 & 5
    \end{pmatrix} +\begin{pmatrix}
        0   & -0.5 \\
        0.5 & 0
    \end{pmatrix}$ ולכן ניקח $M_1=\begin{pmatrix}
        1   & 2.5 \\
        2.5 & 5
    \end{pmatrix}, M_2=\begin{pmatrix}
        0   & -0.5 \\
        0.5 & 0
    \end{pmatrix}$ \\\\
הוכחנו כי $f_1(A,B)=\trace(A \tra M_1 B)$ תבנית סימטרית. נוכיח באותו האופן כי $f_2(A,B)=\trace(A \tra M_2 B)$ אנסימטרית:
\begin{align*}
    f_2(B,A) = \trace(B \tra M_2 A) = \trace ((B \tra M_2 A)\tra) = \trace(A \tra M_2 \tra B) = -\trace (A \tra M_2 B) = -f_2(A,B)
\end{align*}
ומקבלים:
\begin{align*}
    f(A,B)=\trace(A \tra M B) = \trace (A \tra (M_1+M_2) B) & =\trace(A \tra M_1 B + A \tra M_2 B)=                              \\
                                                            & = \trace(A \tra M_1 B) + \trace (A \tra M_2 B) = f_1(A,B)+f_2(A,B)
\end{align*}

\pagebreak

\section*{שאלה 2}

כיוון ראשון: נניח כי $f$ ניתנת להצגה כמכפלה של שתי תבניות לינאריות, ונוכיח כי $\rho(f)=1$. \\
יהא אפוא $(w)=(w_1, w_2, ..., w_n)$ בסיס ל$V$ עבורו $f$ מקבלת את הצורה הנתונה. נמצא את $[f]_{(w)}$. \\
לכל $i,j$ מקבלים $[w_i]_{(w)}=e_i, [w_j]_{(w)}=e_j$. לכן:
\begin{align*}
    f(w_i, w_j)=(b_1\cdot 0 + \cdots + b_i \cdot 1 + \cdots + b_n \cdot 0)(c_1\cdot 0 + \cdots + c_j \cdot 1 + \cdots + c_n \cdot 0)=b_ic_j
\end{align*}
ומקבלים $[f]_{(w)} = [b_ic_j]_{ij}$. \\
נסמן $c=(c_1, c_2, ..., c_n)$. שורות המטריצה $[f]_{(w)}$ הם כולם כפל בסקלר של וקטור זה. לכן, מרחב השורות של $[f]_{(w)}$ מוכל ב $\Sp(\{c\})$, ונקבל $\rho([f]_w)\leq \Sp(\{c\})\leq 1$. \\
מנגד, מהנתון $f\ne 0$ נסיק $\rho(f)>0$, וקיבלנו $\rho(f)=1$.\\\\
כיוון שני: נניח $\rho(f)=1$ ונוכיח כי $f$ ניתנת להצגה כמכפלת שתי תבניות לינאריות.\\
יהא $(w)$ בסיס כלשהו של $V$. אז המטריצה $[f]_{(w)}$ היא מדרגה 1 על פי 5.1.4. \\
תהא אם כן $c=(c_1, c_2, ..., c_n)\in F^n$ שורה כלשהי ב $[f]_{(w)}$ שאינה אפס. בהכרח קיימת שורה כזו, כי דרגת המטריצה שונה מאפס. \\
היות ודרגת המטריצה היא $1$, השורה ה $i$ של המטריצה היא בהכרח כפל בסקלר $b_i$ (יכול להיות אפס) של $c$.\\
לכל $x,y\in V$ נקבל:
\begin{align*}
    f(x,y)=[x]_{(w)}\tra [f]_{(w)} [y]_{(w)} & =\begin{pmatrix}
                                                    x_1 & x_2 & \cdots & x_n
                                                \end{pmatrix} \begin{pmatrix}
                                                                  b_1c_1 & b_1c_2 & \cdots & b_1c_n \\
                                                                  b_2c_1 & b_2c_2 & \cdots & b_2c_n \\
                                                                  \vdots & \vdots & \ddots & \vdots \\
                                                                  b_nc_1 & b_nc_2 & \cdots & b_nc_n
                                                              \end{pmatrix} \begin{pmatrix}
                                                                                y_1    \\
                                                                                y_2    \\
                                                                                \cdots \\
                                                                                y_n
                                                                            \end{pmatrix}=                                                      \\
                                             & = \sum_{i=1}^{n}(x_i \cdot \sum_{j=1}^{n}b_ic_j y_j) = \sum_{i=1}^{n}b_ix_i \cdot \sum_{j=1}^n c_jy_j
\end{align*}
ובכך סיימנו את ההוכחה.

\pagebreak

\section*{שאלה 3}

\subsection*{סעיף א}

נוכיח לפי 4.1.5. התבנית $f$ נקבעת ע"י הפולינום הבילינארי:
\begin{align*}
    \begin{pmatrix}
        x_1 & x_2
    \end{pmatrix} \begin{pmatrix}
                      1 & 2 \\
                      2 & 4
                  \end{pmatrix} \begin{pmatrix}
                                    y_1 \\
                                    y_2
                                \end{pmatrix}=
    x_1y_1 + 2x_1y_2+2x_2y_1+4x_2y_2=f((x_1, x_2), (y_1, y_2))
\end{align*}
יתר על כך, $[f]_E$ סימטרית ($E$ הוא הבסיס הסטנדרטי ל $\reals^2$) ולכן $f$ סימטרית על פי 4.2.2 והתבנית הריבועית המוסמכת ל$f$ תהיה:
\begin{align*}
    q((x_1, x_2))=f((x_1, x_2), (x_1, x_2))=x_1^2+4x_1x_2+4x_2^2
\end{align*}
נמצא ייצוג אלכסוני ל $q$ ול$f$ על פי שיטת לגראנז'. מקבלים באופן מיידי:
\begin{align*}
    q((x_1, x_2))=1\cdot (x_1+2x_2)^2 + 0 \cdot x_2^2
\end{align*}
אם כן, עלינו למצוא בסיס $(w')$ כך שלכל $v\in \reals^2$, $[v]_{(w')}=\begin{pmatrix}
        x_1' \\
        x_2'
    \end{pmatrix}=\begin{pmatrix}
        x_1+2x_2 \\
        x_2
    \end{pmatrix}=\begin{pmatrix}
        1 & 2 \\
        0 & 1
    \end{pmatrix}\begin{pmatrix}
        x_1 \\
        x_2
    \end{pmatrix}=\begin{pmatrix}
        1 & 2 \\
        0 & 1
    \end{pmatrix}[v]_E
$. \\
נקבל שמטריצת המעבר $M\inv_{(w')\rightarrow E}=\begin{pmatrix}
        1 & 2 \\
        0 & 1
    \end{pmatrix}$, וההופכית לה $M_{E\rightarrow (w')}=\begin{pmatrix}
        1 & -2 \\
        0 & 1
    \end{pmatrix}$ מטריצת המעבר מ $E$ ל$(w')$.\\
מכאן מקבלים $(w')=((1,0), (-2,1))$ ו$[f]_{(w')}=\begin{pmatrix}
        1 & 0 \\
        0 & 0
    \end{pmatrix}$

\subsection*{סעיף ב}

נבדוק באופן ישיר על פי משפט 4.5.1. מטריצת המעבר $M_{E\rightarrow (w')}=\begin{pmatrix}
        1 & -2 \\
        0 & 1
    \end{pmatrix}$ ומקבלים:
\begin{align*}
    [q]_{(w')}=M\tra [q]_E M & =\begin{pmatrix}
                                    1  & 0 \\
                                    -2 & 1
                                \end{pmatrix} \begin{pmatrix}
                                                  1 & 2 \\
                                                  2 & 4
                                              \end{pmatrix} \begin{pmatrix}
                                                                1 & -2 \\
                                                                0 & 1
                                                            \end{pmatrix}
    = \begin{pmatrix}
          1 & 2 \\
          0 & 0
      \end{pmatrix}\begin{pmatrix}
                       1 & -2 \\
                       0 & 1
                   \end{pmatrix}=\begin{pmatrix}
                                     1 & 0 \\
                                     0 & 0
                                 \end{pmatrix}
\end{align*}
\pagebreak

\section*{שאלה 4}

\subsection*{סעיף א}

ראשית, עבור הבסיס הסטנדרטי $E$, מקבלים:
\[
    [q]_E = \begin{pmatrix}
        1      & 1/2    & \cdots & 1/2    \\
        1/2    & 1      & \cdots & 1/2    \\
        \vdots & \vdots & \ddots & \vdots \\
        1/2    & 1/2    & \cdots & 1
    \end{pmatrix}
\]
ננצל את העובדה שהמטריצה ממשית סימטרית. לפי משפט הלכסון האוניטרי, $[q]_E$ לכסינה אורתוגונלית, כלומר קיימת מטריצה אורתוגונלית $P$ כך ש $P\inv[q]_EP=\diag(\lambda_1, \lambda_2, ..., \lambda_n)$.
היות ו$P$ אורתוגונלית מקבלים $P\inv=P\tra$ ולכן $P\tra [q]_E P=\diag(\lambda_1, \lambda_2, ..., \lambda_n$. \\
נמצא את הפולינום האופייני של $[q]_E:$

\begin{align*}
    p(x)=|xI-[q]_E| & =\begin{vmatrix}
                           x-1    & -1/2   & \cdots & -1/2   \\
                           -1/2   & x-1    & \cdots & -1/2   \\
                           \vdots & \vdots & \ddots & \vdots \\
                           -1/2   & -1/2   & \cdots & x-1
                       \end{vmatrix}\equals^{R_1\rightarrow \Sigma R_i}
    \begin{vmatrix}
        x-1-(n-1)/2 & x-1-(n-1)/2 & \cdots & x-1-(n-1)/2 \\
        -1/2        & x-1         & \cdots & -1/2        \\
        \vdots      & \vdots      & \ddots & \vdots      \\
        -1/2        & -1/2        & \cdots & x-1
    \end{vmatrix}=                                     \\
                    & =(x-\frac{n+1}{2})\begin{vmatrix}
                                            1      & 1      & \cdots & 1      \\
                                            -1/2   & x-1    & \cdots & -1/2   \\
                                            \vdots & \vdots & \ddots & \vdots \\
                                            -1/2   & -1/2   & \cdots & x-1
                                        \end{vmatrix}\equals^{R_i\rightarrow R_i+1/2R_1}
    =(x-\frac{n+1}{2})\begin{vmatrix}
                          1      & 1      & \cdots & 1      \\
                          0      & x-1/2  & \cdots & 0      \\
                          \vdots & \vdots & \ddots & \vdots \\
                          0      & 0      & \cdots & x-1/2
                      \end{vmatrix}=                                  \\
                    & =(x-\frac{n+1}{2})(x-\frac{1}{2})^{n-1}
\end{align*}
קיבלנו 2 ע"ע $\frac{n+1}{2}$ עם ריבוי אלגברי $1$, ו-$\frac{1}{2}$ עם ריבוי אלגברי $n-1$. \\
היות והמטריצה לכסינה, היא דומה אורתוגונלית (ולכן חופפת) ל$\diag(\frac{n+1}{2}, \frac{1}{2}, ..., \frac{1}{2})$.\\
כלומר, על פי 4.5.4, קיים בסיס $(w)$ כלשהו כך ש $[q]_{(w)}=\diag(\frac{n+1}{2}, \frac{1}{2}, ..., \frac{1}{2})$,
והצורה האלכסונית של $q$ בבסיס זה תהיה:
\[
    q(x_1, x_2, ..., x_n)=\frac{n+1}{2}x_1 + \frac{1}{2}x_2 + \cdots + \frac{1}{2}x_n
\]
כמו כן, התבנית הביליניארית הקוטבית ל$[q]$ תהיה, על פי $[q]_E$, התבנית:
\[
    f((x_1, x_2, ..., x_n), (y_1, y_2, ..., y_n))=\sum_{i=1}^{n}x_iy_i + \frac{1}{2}\sum_{1\leq i,j\leq n, i\ne j} x_iy_j
\]

\subsection*{סעיף ב}
מכיוון ש $[q]_(w)=P\tra [q]_E P$, מקבלים ש $P$ מטריצת המעבר מ$E$ ל$(w)$. היות ו-$P$ אורתוגונלית נסיק כי היא מטריצת מעבר בין בסיסים אורתונורמליים, ולכן $(w)$ בסיס א"נ של וקטורים עצמיים של $[q]_E$.
נמצא בסיסים א"נ למרחבים העצמיים $V_{(n+1)/2}, V_{1/2}$ של $[q]_E$. איחודים יהיה, לפי 2.3.6, בסיס א"נ מתאים.\\
\\\\
עבור $V_{(n+1)/2}$ נקבל מרחב עצמי מממד 1. זהו מרחב האפס של המטריצה:
\[
    \begin{pmatrix}
        (n-1)/2 & -1/2    & \cdots & -1/2    \\
        -1/2    & (n-1)/2 & \cdots & -1/2    \\
        \vdots  & \vdots  & \ddots & \vdots  \\
        -1/2    & -1/2    & \cdots & (n-1)/2
    \end{pmatrix}
\]
קל לראות כי $w_1=(1,1,...,1)$ פותר את המשוואה. הקבוצה $\{ \frac{1}{||w_1||}w_1 \}$ בת"ל ומוכלת במרחב $V_{(n+1)/2}$ מממד 1 ולכן מהווה בסיס א"נ למרחב זה.
\\\\
נעבור למציאת בסיס א"נ למרחב העצמי $V_{1/2}$. זהו מרחב האפס של המטריצה:
\[
    \begin{pmatrix}
        -1/2   & -1/2   & \cdots & -1/2   \\
        -1/2   & -1/2   & \cdots & -1/2   \\
        \vdots & \vdots & \ddots & \vdots \\
        -1/2   & -1/2   & \cdots & -1/2
    \end{pmatrix}\rightarrow
    \begin{pmatrix}
        1      & 1      & \cdots & 1      \\
        1      & 1      & \cdots & 1      \\
        \vdots & \vdots & \ddots & \vdots \\
        1      & 1      & \cdots & 1
    \end{pmatrix}
\]
נבחר למשל את הוקטורים $w_2=(1,-1,0,...,0), w_n=(0,...0, 1, -1)$ ונבצע באינדוקציה תהליך גרם-שמידט.\\
טענת האינדוקציה: לכל $2\leq k\leq n$, הוקטור $w_k*$ המתקבל בתהליך גרם-שמידט על $(w_2,..,w_n)$ הוא
\[
    \frac{1}{\sqrt{k(k-1)}}(\sum_{i=1}^{k}e_i-ke_k)=\frac{1}{\sqrt{k(k-1)}}(1,1,...,1,-(k-1),0,0,...,0)
\]
בסיס האינדוקציה: נרמול הוקטור $w_2$ ייתן לנו $w_2*=\frac{1}{\sqrt{2}}(e_1-e_2)$. \\
צעד האינדוקציה: נניח כי כל הוקטורים $w_2*..w_{k-1}*$ הם מהצורה הנתונה, ובפרט $w_{k-1}*=\frac{1}{\sqrt{(k-1)(k-2)}}(\sum_{i=1}^{k-1}e_i-(k-1)e_{k-1})$
עבור הוקטור $w_k=e_{k-1}-e_{k}$ מקבלים:

ניעזר בתכונות המכפלה הפנימית. ע"פ 1.5.7:
\begin{align*}
    (w_k, w_i*) & =\sum_{j=1}^{n}(w_k, e_j)(w_i*, e_i)=0+1(w_i*, e_{k-1})-(w_i*, e_k)==(w_i*, e_{k-1})                                                                   \\
                & =\begin{cases}
                       0                                                                                                                                    & i \ne k-1 \\
                       \frac{1}{\sqrt{(k-1)(k-2)}}((e_1, e_{k-1}) + \cdots + (e_{k-2}, e_{k-1}) - (k-2)(e_{k-1}, e_{k-1}))=\frac{-(k-2)}{\sqrt{(k-1)(k-2)}} & i=k-1
                   \end{cases}
\end{align*}
ואז:
\begin{align*}
    w_k-\sum_{i}^{k-1}(w_k, w_i*)w_i* & =w_k+\frac{(k-2)}{\sqrt{(k-1)(k-2)}}w_{k-1}*=                              \\
                                      & =e_{k-1}-e_{k}+\frac{(k-2)}{(k-1)(k-2)}(\sum_{i=1}^{k-1}e_i-(k-1)e_{k-1})= \\
                                      & =e_{k-1}-e_{k} + \frac{1}{k-1}\sum_{i=1}^{k-1}e_i-e_{k-1}=                 \\
                                      & = \frac{1}{k-1}\sum_{i=1}^{k-1}e_i - e_{k}=                                \\
                                      & = \frac{1}{k-1}(1, 1, ...,1, -(k-1), 0, ..., 0)
\end{align*}
וכמו כן,
\begin{align*}
    ||w_k-\sum_{i}^{k-1}(w_k, w_i*)w_i*|| & = ||\frac{1}{k-1}(1, 1, ...,1, -(k-1), 0, ..., 0)||=     \\
                                          & =\frac{1}{k-1}||(1, 1, ...,1, -(k-1), 0, ..., 0)|| =     \\
                                          & = \frac{1}{k-1} \cdot \sqrt{1^2+1^2+\cdots 1^2+(k-1)^2}= \\
                                          & = \frac{1}{k-1} \cdot \sqrt{(k-1)\cdot 1^2+(k-1)^2} =    \\
                                          & = \frac{1}{k-1} \cdot \sqrt{k(k-1)} =                    \\
\end{align*}
אי-לכך, $w_k*=\frac{1}{\sqrt{k(k-1)}}(1,1,...,1,-(k-1),0,0,...,0)$ והושלמה הוכחת האינדוקציה.\\
תשובה סופית: $(w)=(\frac{1}{\sqrt{n}}(1,...,1), \frac{1}{\sqrt{2}}(1,-1,...,0), \frac{1}{\sqrt{3}}(1,1,-2,...,0), ... , \frac{1}{\sqrt{n(n-1)}}(1,1,..1, n-1))$ בסיס מלכסן.

\pagebreak

\section*{שאלה 5}

\subsection*{סעיף א}

תהא $q\ne 0$ תבנית ריבועית שממד מרחב המקור שלה $V$ הוא לכל הפחות 2 כפי שנתון. נסמן $\dim V=n$. \\
היות ו$q$ תבנית ריבועית, קיים לפי 5.1.2ג בסיס כלשהו $(w)=(w_1, w_2, .., w_n)$ למרחב $V$ כך ש$[q]_{(w)}$ מטריצה אלכסונית. \\
נסמן את הסקלארים באלכסון המטריצה $[q]_{(w)}$ ב $\lambda_1, \lambda_2, ..., \lambda_n$. מקבלים שלכל $v\in V$, כאשר $[v]_{(w)}=(x_1 \; x_2 \; \cdots \; x_n)\tra$:
\begin{align*}
    q(v)=\lambda_1x_1^2 + \lambda_2x_2^2+\cdots+\lambda_nx_n^2
\end{align*}
כעת, נחלק למקרים.
\begin{itemize}
    \item אילו קיים $\lambda_i=0$ כלשהו, אז מקבלים עבור $w_i\ne 0$ כי: \[
              q(w_i)=\lambda_1 \cdot 0^2 + \cdots + \lambda_i \cdot 1^2 + \cdots + \lambda_n \cdot 0^2 = \lambda_i=0
          \]
    \item אחרת, אילו אין סקלאר $\lambda_i$ השווה לאפס, אז בפרט $\lambda_1, \lambda_2\ne 0$ (קיומם מובטח בוודאות כי $n\geq 2$). \\
          נבחר אפוא $v=iw_1+\sqrt{\lambda_1/\lambda_2}w_2\ne 0$. מקבלים $[v]_{(w)}=(i \;\;\; \sqrt{\lambda_1/\lambda_2} \;\;\; 0 \; \cdots \; 0) \tra$ ולכן: \[
              q(v)=\lambda_1 \cdot i^2 + \lambda_2 \cdot \sqrt{\frac{\lambda_1}{\lambda_2}}^2 + \cdots + \lambda_i \cdot 0^2 + \cdots + \lambda_n \cdot 0^2 = -\lambda_1 + \lambda_2 \cdot \frac{\lambda_1}{\lambda_2}=0
          \]
\end{itemize}
בשני המקרים מצאנו $v\ne 0$ כך ש $q(v)=0$ והטענה נכונה.

\subsection*{סעיף ב}

הטענה לא נכונה אילו מרחב המקור $V$ היה מעל $\reals$ ושדה הטווח היה $\reals$. \\
ניקח למשל $V=\reals^2$ ו$q$ לפי הבסיס הסנדרטי תהא $q(x_1, x_2)=x_1^2+x_2^2=||(x_1, x_2)||^2$.\\
מתכונת החיוביות של המכפלה הפנימית, $q(x_1, x_2)=||(x_1, x_2)||^2=0$ אם ורק אם $(x_1, x_2)=0$, ולכן לא קיים $v=0$ כך ש $q(v)=0$.

\end{document}