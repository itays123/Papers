% !TEX program = xelatex
\documentclass{article}
\usepackage[]{amsthm} %lets us use \begin{proof}
\usepackage{amsmath}
\usepackage{enumerate}
\usepackage{xparse}
\usepackage[makeroom]{cancel}
\usepackage[]{amssymb} %gives us the character \varnothing
\usepackage{fontspec}
\usepackage{enumerate}
\usepackage{polyglossia}
\usepackage{relsize}
\usepackage[left=2.0cm, top=2.0cm, right=2.0cm, bottom=2.0cm]{geometry}

\setdefaultlanguage{hebrew}
\setotherlanguage{english}

\setmainfont{[Arial.ttf]}
\newfontfamily\hebrewfont{[Arial.ttf]}

\newcommand\underrel[2]{\mathrel{\mathop{#2}\limits_{#1}}}
\DeclareMathOperator*{\equals}{=}
\def\reals{\mathbb{R}}
\def\naturals{\mathbb{N}}

\title{מטלת מנחה 15 - אינפי 2}
\author{328197462}
\date{31/01/2023}

\begin{document}
\maketitle

\section*{שאלה 1}

\subsection*{סעיף א1}

\begin{align*}
    \lim_{(x,y)\rightarrow(0,0)} \frac{\sin(2x^2+2y^2)+y^3}{x^2+y^2}=
    \lim_{(x,y)\rightarrow(0,0)}\frac{\sin(2x^2+2y^2)}{x^2+y^2}+\lim_{(x,y)\rightarrow(0,0)}\frac{y^3}{x^2+y^2}
\end{align*}
כי מתקיים:
\begin{align*}
    \lim_{(x,y)\rightarrow(0,0)}\frac{\sin(2x^2+2y^2)}{x^2+y^2}=
    [t=2x^2+2y^2\rightarrow 0^+]=
    \lim_{t\rightarrow 0^+}\frac{\sin t}{t/2}=
    2\lim_{t\rightarrow 0^+}\frac{\sin t}{t}=
    2
\end{align*}
וכן, לפי כלל הסנדוויץ':
\begin{align*}
    0\leq \left|\frac{y^3}{x^2+y^2}\right|=
    |y| \cdot \frac{y^2}{x^2+y^2} \leq
    |y| \cdot \frac{y^2}{y^2} =
    |y|\xrightarrow[(x,y)\rightarrow(0,0)]{}0
\end{align*}
ולכן $\frac{y^3}{x^2+y^2}\xrightarrow[(x,y)\rightarrow(0,0)]{}0$.
לסיכום נקבל:
\begin{align*}
    \lim_{(x,y)\rightarrow(0,0)} \frac{\sin(2x^2+2y^2)+y^3}{x^2+y^2}=2+0=2
\end{align*}

\subsection*{סעיף א2}

\begin{align*}
    0\leq \left|x\arctan \left( \frac{x}{x^2+(y-2)^2} \right)\right|=
    |x| \left|\arctan\left( \frac{x}{x^2+(y-2)^2} \right)\right|\leq
    |x|\cdot \frac{\pi}{2} \xrightarrow[(x,y)\rightarrow(0,2)]{} 0 \cdot \frac{\pi}{2}= 0
\end{align*}
ולכן מתקיים $x\arctan \left( \frac{x}{x^2+(y-2)^2} \right)\xrightarrow[(x,y)\rightarrow(0,2)]{} 0$

\subsection*{סעיף ב1}

עלינו לבדוק האם קיים הגבול $f(x,y)\xrightarrow[(x,y)\rightarrow(0,0)]{}1$. \\
נכתוב את הפונקציה בדרך נוחה יותר. לכל $(x,y)\ne (0,0)$:
\begin{align*}
    f(x,y)=\frac{e^{|x|+|y|}-1}{|x|+|y|}\cdot \ln(|xy|+e)
\end{align*}
מתקיים:
\begin{align*}
    \lim_{(x,y)\rightarrow(0,0)}\frac{e^{|x|+|y|}-1}{|x|+|y|} & =
    [t=|x|+|y|\rightarrow 0^+]=
    \lim_{t\rightarrow 0^+}\frac{e^t-1}{t}\equals_{\text{לופיטל}}
    \lim_{t\rightarrow 0^+}\frac{e^t}{1}=1                        \\
    \lim_{(x,y)\rightarrow(0,0)} \ln(|xy|+e)                  & =
    [p=|xy|\rightarrow 0^+]=
    \lim_{(x,y)\rightarrow(0,0)} \ln(t+e)=
    \ln(e)=1
\end{align*}
ולכן $f(x,y)\xrightarrow[(x,y)\rightarrow(0,0)]{}1\cdot 1=1$ והפונקציה רציפה.

\subsection*{סעיף ב2}

הפונקציה לא רציפה בנקודה, כי לא מתקיים הגבול $g(x,y)\xrightarrow[(x,y)\rightarrow(0,0)]{}1$. \\
ניקח למשל $P_n=(\frac{1}{n^2}, \frac{1}{n})$, ונקבל:
\begin{align*}
    \lim_{n\rightarrow\infty} g(P_n)=
    \lim_{n\rightarrow\infty} \frac{\frac{1}{n^4}-\frac{1}{n^4}}{\frac{1}{n^4}+\frac{1}{n^4}}=
    \lim_{n\rightarrow\infty} \frac{0}{\frac{2}{n^4}}=0
\end{align*}
לכן, לפי היינה, לא מתקיים $g(x,y)\xrightarrow[(x,y)\rightarrow(0,0)]{}1=g(0,0)$ והפונקציה לא רציפה בנקודה.

\pagebreak

\section*{שאלה 2}

\subsection*{סעיף א}

עלינו לבדוק האם הפונקציה בשני משתנים $f(x,y)=(x^{1/3}+y^{1/3})^3$ דיפרנציאבילית בנקודה $(0,0)$.\\\\
נחשב נגזרות חלקיות בנקודה $p_0=(0,0)$
\begin{align*}
    f_x(0,0) & =\lim_{h\rightarrow 0} \frac{f(0+h, 0)-f(0,0)}{h}=
    \lim_{h\rightarrow 0} \frac{(h^{1/3}+0)^3-0}{h}=
    \lim_{h\rightarrow 0} \frac{h}{h}=
    \lim_{h\rightarrow 0} 1=
    1                                                               \\
    f_y(0,0) & = \lim_{h\rightarrow 0} \frac{f(0, 0+h)-f(0,0)}{h} =
    \lim_{h\rightarrow 0} \frac{(0+h^{1/3})^3-0}{h} =
    \lim_{h\rightarrow 0} \frac{h}{h}=
    \lim_{h\rightarrow 0} 1=
    1
\end{align*}
\\
כעת עלינו לבדוק את קיום הגבול $\epsilon(x,y)=\frac{r(x,y)}{d((x,y),(0,0))}=\frac{f(x,y)+f_x(0,0)\cdot x+f_y(0,0)\cdot y}{\sqrt{x^2+y^2}}\xrightarrow[(x,y)\rightarrow (0,0)]{}0$. \\
ניקח למשל $P_n=(\frac{1}{n}, \frac{1}{n})$ ונקבל:
\begin{align*}
    \lim_{n\rightarrow\infty} \epsilon(P_n) & =
    \lim_{n\rightarrow\infty} \frac{f(\frac{1}{n},\frac{1}{n})-1\cdot \frac{1}{n}-1\cdot \frac{1}{n}}{((\frac{1}{n})^2+(\frac{1}{n})^2)^{1/2}}=
    \lim_{n\rightarrow\infty} \frac{((\frac{1}{n})^{1/3}+(\frac{1}{n})^{1/3})^3- 2\cdot \frac{1}{n}}{(\frac{2}{n^2})^{1/2}}=                          \\
                                            & = \lim_{n\rightarrow\infty} \frac{(2(\frac{1}{n})^{1/3})^3-2\cdot \frac{1}{n}}{(2\frac{1}{n^2})^{1/2}}=
    \lim_{n\rightarrow\infty} \frac{8\cdot \frac{1}{n}- 2\cdot \frac{1}{n}}{\sqrt{2}\cdot \frac{1}{n}}=
    \lim_{n\rightarrow\infty} \frac{6}{\sqrt{2}}=
    \frac{6}{\sqrt{2}}\ne 0
\end{align*}
לכן לפי הגדרת היינה לא מתקיים הגבול והפונקציה לא דיפרנציאבילית.

\subsection*{סעיף ב}

נציין כי הפונקציה $f(x,y)=3x^2-y^2$ דיפרנציאבילית כפולינום רב-משתנים בכל המישור.\\
עלינו למצוא נקודה במשטח $(a,b,f(a,b))$ שהמישור המשיק לה מקביל למישור $6x+4y+z+5=0$. \\
לפי הגדרה $7.64$, בכל נקודה במישור מצורה זו, משוואת המישור המשיק למשטח יהיה:
\begin{align*}
    z=f(a, b)+f_x(a,b)\cdot (x-a)+f_y(a, b)(y-b)
\end{align*}
נחשב נגזרות חלקיות:
\begin{align*}
    f_x & =6x \ \ \ f_y = -2y           \\
    z   & =3a^2-b^2+6a(x-a)-2b(y-b)     \\
        & -6ax+2by+z=3a^2-b^2-6a^2+2b^2 \\
        & -6ax+2by+z=-3a^2+b^2
\end{align*}
על מנת שהמישור יהיה מקביל למישור הנתון, מקדמי שלוש המשתנים צריכים להיות פרופורציונליים. \\
במילים אחרות, קיים $\lambda\in \reals$ כך ש $(-6a, 2b, 1)=\lambda(6,4,1)$ \\
נסיק $\lambda=1$ ונקבל:
\begin{align*}
    \begin{cases}
        -6a=6 & \Rightarrow a=-1 \\
        2b=4  & \Rightarrow b=2
    \end{cases}
\end{align*}
ואכן, נציב ונקבל כי המישור המשיק בנקודה $(-1,2, f(-1,2))$ יהיה:
\begin{align*}
    6x+4y+z-1=0
\end{align*}
מישור זה מקביל למישור הנתון ולא מתלכד איתו (אין פרופורציה באיבר החופשי $-1\ne \lambda\cdot 5$).

\pagebreak

\section*{שאלה 3}

\subsection*{סעיף א}

נסמן בשאלה את $x(t), y(t)$ ו$\alpha(t)$ את הגדלים של שתי הצלעות הנתונות בס"מ והזווית (ברדיאנים) שביניהן בנקודת זמן מסוימת (בשניות), בהתאמה. נתון בשאלה כי בנקודת זמן מסוימת $t_0$ מתקיים $x(t_0)=4, y(t_0)=3, \alpha(t_0)=\frac{\pi}{6}$ וכמו כן ערכי הנגזרות $x', y'$ המייצגות את קצב השינוי לשנייה מקיימים $x'(t_0)=y'(t_0)=1$.
עלינו למצוא את $\alpha'(t_0)$.
\\\\
מנתוני השאלה, שטח המשולש בזמן $t_0$ הוא (לפי נוסחה ידועה) $\frac{1}{2}\cdot 4\cdot 3 \cdot \sin \frac{\pi}{6}=3$. נתון כי שטח המשולש נשאר קבוע, ולכן בכל נקודת זמן $t$ מתקיים:
\begin{align*}
    \frac{1}{2} \cdot x(t) \cdot y(t) \cdot \sin(\alpha(t))=3
\end{align*}
ומכאן נקבל $\alpha(t)=\arcsin\left(\frac{6}{x(t)\cdot y(t)}\right)$ בכל נקודת זמן $t$.
\\\\
אז נסמן $f(x,y)=\arcsin\left(\frac{6}{x\cdot y}\right)$. כאן נדרש $x\cdot y > 6$ וכן מהעובדה ש$x,y$ מייצגים אורכים חיוביים של צלעות נסיק $x,y>0$. \\
נחשב נגזרות חלקיות לפי הכלל בעמוד $68$:
\begin{align*}
    f_x & =
    \frac{1}{\sqrt{1-(\frac{6}{xy})^2}} \cdot \frac{6}{y}\cdot (\frac{1}{x})' =
    \frac{|xy|}{\sqrt{x^2y^2-36}} \cdot \frac{6}{y}\cdot \frac{-1}{x^2} =
    \frac{-6}{x\sqrt{x^2y^2-36}} \\
    f_y & =
    \frac{1}{\sqrt{1-(\frac{6}{xy})^2}} \cdot \frac{6}{x}\cdot (\frac{1}{y})' =
    \frac{|xy|}{\sqrt{x^2y^2-36}} \cdot \frac{6}{x}\cdot \frac{-1}{y^2} =
    \frac{-6}{y\sqrt{x^2y^2-36}}
\end{align*}
בפרט $f_x(4,3)=\frac{-6}{4\sqrt{108}}=-\frac{\sqrt{3}}{12}$ וכן $f_y(4,3)=\frac{-6}{3\sqrt{108}}=-\frac{\sqrt{3}}{9}$
\\\\
אי לכך, לפי כלל השרשרת 7.66 עבור $\alpha(t)=f(x(t), y(t))$ מקבלים:
\begin{align*}
    \alpha'(t)=f_x(x(t), y(t))\cdot x'(t)+f_y(x(t), y(t))\cdot y'(t)
\end{align*}
ובפרט עבור $t=t_0$ מקבלים:
\begin{align*}
    \alpha'(t_0) & =f_x(x(t_0), y(t_0))\cdot x'(t_0)+f_y(x(t_0), y(t_0))\cdot y'(t_0)=                                            \\
                 & = f_x(4,3)\cdot 1 + f_y(4,3)\cdot 1= -\frac{\sqrt{3}}{12} - \frac{\sqrt{3}}{9} = -\frac{7\sqrt{3}}{36}=-0.3367
\end{align*}
כלומר ברגע זה קצב גדילתה של הזווית הוא $-0.3367$ רדיאנים בשנייה.

\subsection*{סעיף ב}

תהא פונקציה $f(x,y)$ בעלת נגזרות חלקיות רציפות מסדר 2 בכל המישור, ומגדירים $x(u,v)=u+v, y(u,v)=u-v$. \\
כמו כן מגדירים $z(u,v)=f(x(u,v), y(u,v))$
אז לפי חוקי הגזירה מאתר הקורס מתקיים:
\begin{align*}
    \frac{\partial z}{\partial u}=z_u & = f_x \cdot x_u + f_y \cdot y_u =
    f_x \cdot 1 + f_y \cdot 1=f_x(u-v, u+v)+f_y(u-v, u+v)                                                                 \\
    z_uv                              & = (f_{xx} \cdot x_v + f_{xy} \cdot y_v) + (f_{yx} \cdot x_v + f_{yy} \cdot y_v) =
    f_{xx} \cdot 1 + f_{xy} \cdot (-1) + f_{yx} \cdot 1 + f_{yy} \cdot (-1) \equals_{\text{נתון} + 7.71}                  \\
                                      & = f_{xx} - f_{xy} + f_{xy} - f_{xx} = 0
\end{align*}
קיבלנו כי $\frac{\partial}{\partial v}(z_u)$ זהותית 0 בכל המישור, לכן הפונקציה $z_u$ אינה מושפעת מערך המשתנה $v$.\\\\
מכאן נסיק כי קיימת פונקציה במשתנה אחד $g(t)$ כך ש $z_u(u,v)\equiv g(u)$.
באופן דומה להוכחה שלנו, $z_v$ אינה תלויה ב$u$ ולכן קיימת פונקציה נוספת $h(t)$ במשתנה אחד כך ש $z_v(u,v)\equiv h(v)$. נדגיש כי $g,h$ רציפות (סכום של נגזרות חלקיות רציפות) ולכן אינטגרביליות לפי $u,v$ בהתאמה, ויהיו $G,H$ קדומות להן במשתנה אחד, כלומר מתקיים $G'\equiv g, H'\equiv h$\\
נבחר את $G,H$. מתקיים $z(u,v)=G(u)+H(v)+C$. כאשר $C$ קבוע. נראה זאת ע"י חישוב נגזרות חלקיות לפונקציה $r(u,v)=z(u,v)-G(u)-H(v)$:
\begin{align*}
    r_u=z_u-g(u)=0 \ \ \ r_v = z_v-h(v)=0
\end{align*}
מכאן שפונקציית ההפרש $r$ לא תלויה לא ב$u$ ולא ב$v$ (ולכן שווה לקבוע). נוכל לקבל את הפונקציות $h_1, h_2$ הרצויות ע"י הוספת קבועים כרצוננו.

\subsection*{סעיף ג}

תהא $h(r)$ פונקציה במשתנה אחד גזירה פעמיים ותהא $f(x,y)\equiv h(r)$ עבור $r(x,y)=\sqrt{x^2+y^2}$.\\\\
נוכיח את טענת העזר הבאה: לכל פונקציה גזירה $g(r)$ ולכל פוקנקציה בשני משתנים $k(u,v)\equiv g(r)$ כך ש$r(u,v)=\sqrt{u^2+v^2}$ מקבלים $k_u=g'(r) \cdot \frac{u}{r}$. \\
אכן, לפי עמוד 68 בכרך ג נקבל $k_u=g'(r)+r_u$, ומתקיים $r_u=\frac{2u}{2\sqrt{u^2+v^2}}=\frac{u}{\sqrt{u^2+v^2}}=\frac{u}{r}$. \\
בנקל נוכל להוכיח טענה זהה עבור $v$.\\\\
כמו כן נשים לב כי מתקיים:
\begin{align*}
    r_{uu}=(\frac{u}{r})'=\frac{1\cdot r-u\cdot r_u}{r^2}=\frac{r-u\cdot \frac{u}{r}}{r^2}=\frac{r^2-u^2}{r^3}=\frac{v^2}{r^3}
\end{align*}
וכן טענה דומה ניתן להוכיח עבור $r_{vv}$
\\\\
יהי $r>0$ מספר ממשי. אז ניקח את הנקודות $(x,y)\in \reals^2$ המקיימות $x=r\cos \theta, y = r\sin \theta$ ובפרט $(x,y)\ne (0,0)$. \\
\begin{align*}
    f_x    & = h'(r) \cdot \frac{x}{r}                                                 \\
    f_{xx} & = (h'(r))_x \cdot \frac{x}{r} + h'(r) \cdot (\frac{x}{r})_x =
    (h''(r) \cdot \frac{x}{r}) \cdot \frac{x}{r} + h'(r)+h'(r) \cdot \frac{y^2}{r^3} = \\
           & = h''(r) \cdot \frac{x^2}{r^2} + h'(r) \cdot \frac{y^2}{r^3} =
    \frac{x^2+y^2}{r^3} \cdot (rh''(r)+h'(r)) = \frac{1}{r} \cdot (rh''(r)+h'(r))
\end{align*}
באופן דומה מתקיים $f_y=h'(r)\cdot \frac{y}{r}$ וכן $f_{yy}=\frac{1}{r} \cdot (rh''(r)+h'(r))$.\\
נקבל לפי הנתון $f_{xx}+f_{yy}=2\cdot \frac{1}{r} \cdot (rh''(r)+h'(r))=0$ ומכאן, היות ו$r\ne 0$, $rh''(r)+h'(r)=0$

\pagebreak

\section*{שאלה 4}

\subsection*{סעיף א}

נחשב נגזרות חלקיות לפונקציה $f(x,y)=\cos x + \cos y + \cos (x+y)$:
\begin{enumerate}[i]
    \item $f_x=-\sin x -\sin(x+y)$
    \item $f_y = -\sin y - \sin (x+y)$
\end{enumerate}
עלינו למצוא נקודות $(x,y)\in D$ בהן $f_x=f_y=0$. נפתור את מערכת המשוואות:
\begin{align*}
    -\sin x -\sin(x+y) & =-\sin y - \sin (x+y)=0             \\
    \sin x             & = \sin y                            \\
    x = y + 2\pi k     & \ \text{או} \  x = \pi - y + 2\pi k
\end{align*}
נציב כל אחת מן האפשרויות במשוואה השנייה. עבור $x=y+2\pi k$ מקבלים:
\begin{align*}
    -\sin y - \sin (2y+2\pi k) & = 0 \\
    -\sin y - 2\sin y \cos y   & = 0 \\
    -\sin y (1+2\cos y)        & = 0
\end{align*}
נקבל שתי אפשרויות - באפשרות הראשונה $\sin y = 0$ ולכן $y=\pi k$ ובתחום הנתון מקבלים $(x,y)=(0,0)$. \\
באפשרות השנייה מקבלים $\cos y = -\frac{1}{2}$ ולכן $y=\pm\frac{2}{3}\pi+2\pi k$ - לא בתחום!\\
כמו כן, עבור $x=\pi-y+2\pi k$ מקבלים:
\begin{align*}
    -\sin y - \sin(\pi + 2\pi k) & = 0 \\
    -\sin y                      & = 0
\end{align*}
ושוב מקבלים $y=\pi k$, ועבור $y=0$ בתחום הנתון נקבל את הנקודות $(\pi + 2\pi k, 0)\notin D$\\
לסיכום - קיבלנו נקודה חשודה לקיצון יחידה $(0,0)\in D$
\\\\
נחשב נגזרות מסדר שני:
\begin{enumerate}[i]
    \item $f_{xy}=-\cos(x+y)$ $\Leftarrow$ $f_{xy}(0,0)=-1$
    \item $f_{xx}=-\cos x - \cos(x+y)$ $\Leftarrow$ $f_{xx}(0,0)=-2$
    \item $f_{yy}=-\cos y - \cos(x+y)$ $\Leftarrow$ $f_{yy}(0,0)=-2$
\end{enumerate}
מתקיימים תנאי משפט 7.72 ועבור הנקודה המדוברת מתקיים $D=(-1)^2-(-2)(-2)=-3<0$ ולכן לפי 7.72 זוהי נקודת מקסימום מקומי.\\
אין נקודות חשודות נוספות ולכן תם החישוב.

\subsection*{סעיף ב}

נבדוק את שפות התחום:
\begin{enumerate}[i]
    \item $x=\frac{\pi}{2}, y\in (-\frac{\pi}{2}, \frac{\pi}{2})$ \\
          מקבלים את הפונקציה $f(y)=0+\cos y + \cos (y+\frac{\pi}{2})=\cos y - \sin y$, ונגזרתה $f'(y)=-\sin y - \cos y$ מתאפסת כאשר:
          \begin{align*}
              \sin y & = -\cos y = \cos (\pi - y)                                             \\
              \sin y & = \sin (y-\frac{\pi}{2})                                               \\
              y      & =y-\frac{\pi}{2}+2\pi k \  \text{או} \ y = \frac{3\pi}{2} - y + 2\pi k \\
              2y     & =\frac{3\pi}{2}+2\pi k                                                 \\
              y      & = \frac{3\pi}{4}+\pi k
          \end{align*}
          נקבל נקודה לבדיקה $(\frac{\pi}{2}, -\frac{\pi}{4})$
    \item $x=-\frac{\pi}{2}, y\in(-\frac{\pi}{2}, \frac{\pi}{2})$ \\
          מקבלים את הפונקציה $g(y)=0+\cos y + \cos(y-\frac{\pi}{2})=\cos y + \sin y$, שנגזרתה $g'(y)=-\sin y + \cos y$ מתאפסת כאשר:
          \begin{align*}
              \sin y & = \cos y             \\
              \tan y & = 1                  \\
              y      & =\frac{\pi}{4}+\pi k
          \end{align*}
          נקבל נקודה לבדיקה $(-\frac{\pi}{2}, \frac{\pi}{4})$
    \item $y=\frac{\pi}{2}, x\in (-\frac{\pi}{2}, \frac{\pi}{2})$ \\
          מקבלים את הפונקציה $f(x)=\cos x + 0 + \cos(x+\frac{\pi}{2})$ והראינו כי נגזרתה מתאפסת כאשר $x=-\frac{\pi}{4}$. \\
          נקבל נקודה לבדיקה $(-\frac{\pi}{4}, \frac{\pi}{2})$
    \item $y=-\frac{\pi}{2}, x\in(-\frac{\pi}{2}, \frac{\pi}{2})$ \\
          באופן דומה, מקבלים נקודה לבדיקה $(\frac{\pi}{4}, -\frac{\pi}{2})$
\end{enumerate}
כמובן יש לבדוק את הקצוות $(-\frac{\pi}{2}, -\frac{\pi}{2}), (-\frac{\pi}{2}, \frac{\pi}{2}), (\frac{\pi}{2}, -\frac{\pi}{2}), (\frac{\pi}{2}, \frac{\pi}{2})$. סה"כ עלינו לבדוק 8 נקודות קצה ונקודת קיצון פנימית.\\
נחשב:
\begin{align*}
    \begin{matrix}
        f(0,0)=3                                 & f(\frac{\pi}{2}, -\frac{\pi}{4})=\sqrt{2} & f(-\frac{\pi}{2}, \frac{\pi}{4})=\sqrt{2} \\
        f(\frac{\pi}{2},-\frac{\pi}{4})=\sqrt{2} & f(-\frac{\pi}{2},\frac{\pi}{4})=\sqrt{2}  & f(\frac{\pi}{2}, \frac{\pi}{2})=-1        \\
        f(-\frac{\pi}{2}, \frac{\pi}{2})=1       & f(\frac{\pi}{2}, -\frac{\pi}{2})=1        & f(-\frac{\pi}{2}, \frac{-\pi}{2})=-1
    \end{matrix}
\end{align*}
נקבל ערך מינימלי $f(\frac{\pi}{2}, \frac{\pi}{2})=f(-\frac{\pi}{2}, -\frac{\pi}{2})=-1$
וערך מקסימלי $f(0,0)=3$.

\pagebreak

\section*{שאלת רשות}

תהא $f(x,y)$ דיפרנציאבילית בכל המישור ויהיו $p_1, p_2\in \reals^2$. \\
נסמן $u=p2-p1=(x_2-x_1, y_2-y_1)$, אז $v=\frac{u}{||u||}$ וקטור היחידה בכיוון מ$p_1$ ל$p_2$.\\
צריך להוכיח כי קיימת $p_0\in[p_1, p_2]$ כך ש$f(p_2)-f(p_1)=(D_v f)(p_0)\cdot ||p_1-p_2||$
\\\\
נגדיר פונקציה במשתנה יחיד $h(t)$ המוגדרת בתחום $[0,||u||]$ כך:
\begin{align*}
    h(t)=f(p_1+t\cdot v) = f(p_1+t\cdot \frac{p_2-p_1}{||p_2-p_1||})=
    f((1-\frac{t}{||p_2-p_1||})p_1+\frac{t}{||p_2-p_1||}p_2)
\end{align*}
בפרט $h(0)=f(p_1), h(||p_2-p_1||)=f(p_2)$. מטעמי נוחות נסמן:
\begin{align*}
    x(t) & =(1-\frac{t}{||p_2-p_1||})x_1+\frac{t}{||p_2-p_1||}x_2  \\
    y(t) & = (1-\frac{t}{||p_2-p_1||})y_1+\frac{t}{||p_2-p_1||}y_2
\end{align*}
ומקבלים $h(t)=f(x(t), y(t))$.
\\\\
הפונקציה גזירה בקטע זה כיוון שמייצגת "מסלול" בפונקציה דיפרנציאבילית בשני משתנים. בפרט, נגזרת הפונקציה בקטע לכל $t$ תהיה, לפי כלל השרשרת $7.66$,
\begin{align*}
    h'(t) & =f_x (x(t), y(t)) \cdot x'(t) + f_y(x(t), y(t)) \cdot y'(t)=                                                                                           \\
          & =f_x (x(t), y(t)) \cdot (-\frac{x_1}{||p_2-p_1||}+\frac{x_2}{||p_2-p_1||}) + f_y(x(t), y(t)) \cdot (-\frac{y_1}{||p_2-p_1||}+\frac{y_2}{||p_2-p_1||})= \\
          & = \frac{1}{||p_2-p_1||} \cdot (f_x(x(t), y(t)) \cdot (x_2-x_1) + f_y(x(t), y(t)) \cdot (y_2-y_1)) =                                                    \\
          & = \frac{1}{||p_2-p_1||} \cdot \triangledown f(x(t), y(t)) \cdot (p_2-p_1)=
    \triangledown f(x(t), y(t)) \cdot \frac{p_1-p_2}{||p_2-p_1||}=
    \triangledown f(x(t), y(t)) \cdot v\equals_{7.67}
    (D_v f)(x(t), y(t))
\end{align*}
\\
היות והפונקציה גזירה, נקבל לפי לגראנז' מאינפי 1 כי קיים $t_0\in[0, ||p_2-p_1||]$ כך שמתקיים:
\begin{align*}
    h'(t_0)=\frac{h(||p_2-p_1||)-h(0)}{||p_2-p_1||-0}=\frac{f(p_2)-f(p_1)}{||p_2-p_1||}
\end{align*}
נבחר $p_0=(x(t_0), y(t_0))\in[p_1, p_2]$ ונקבל:
\begin{align*}
    f(p_2)-f(p_1) & =h'(t_0) \cdot ||p_2-p_1||=(D_v f)(p_0)\cdot ||p_2-p_1||
\end{align*}
ובכך סיימנו את ההוכחה.

\end{document}