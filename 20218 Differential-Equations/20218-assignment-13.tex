\documentclass{article}
\usepackage{fontspec}
\newfontfamily\hebrewfont[Script=Hebrew]{Calibri}
\usepackage[left=2.0cm, top=2.0cm, right=2.0cm, bottom=2.0cm]{geometry}
\usepackage{polyglossia}
\usepackage{amsmath}
\usepackage{bidi}
\setdefaultlanguage{hebrew}
\setotherlanguage{english}

\title{מטלת מנחה 13 - קורס 20218}
\author{328197462}
\date{08/09/2023}

\begin{document}
\maketitle

\section*{שאלה 1}

נניח כי קיימת מעכת הומונית $x'=Ax$ עבורה הפונקציות הבאות מהוות פתרון:
\begin{align*}
    x_1(t)=e^t\begin{pmatrix}
                  1 \\
                  2 \\
                  3
              \end{pmatrix} &  &
    x_2(t)=e^{-t}\begin{pmatrix}
                     2 \\
                     4 \\
                     6
                 \end{pmatrix}
\end{align*}
אז גם הפונקציה $u=2x_1-x_2$, שהיא צירוף לינארי של פתרונות למערכת ההומוגנית. מהווה פתרון למערכת.\\
מתקיים:
\begin{align*}
    u(0)=2\cdot e^0 \begin{pmatrix}
                        1 \\
                        2 \\
                        3
                    \end{pmatrix} - e^0\begin{pmatrix}
                                           2 \\
                                           4 \\
                                           6
                                       \end{pmatrix} = \begin{pmatrix}
                                                           0 \\
                                                           0 \\
                                                           0
                                                       \end{pmatrix}
\end{align*}
ולכן לפי למה 4.1.7, $u$ הוא הפתרון הטריוויאלי, כלומר $u(t)=0$ לכל $t\in(-1,1)$.
אולם:
\begin{align*}
    u(0.5)=2\cdot e^{0.5} \begin{pmatrix}
                              1 \\
                              2 \\
                              3
                          \end{pmatrix} - e^{-0.5}\begin{pmatrix}
                                                      2 \\
                                                      4 \\
                                                      6
                                                  \end{pmatrix}=
    (e^{0.5}-e^{-0.5})\begin{pmatrix}
                          2 \\
                          4 \\
                          6
                      \end{pmatrix}=\underline{0}
\end{align*}
ונקבל סתירה היות ו $e^{0.5}-e^{-0.5}\ne 0$ וכן $\begin{pmatrix}
        2 \\
        4 \\
        6
    \end{pmatrix}\ne \underline{0}$

\pagebreak

\section*{שאלה 2}

לפנינו המשוואה $x'=Ax+b$ כך ש:
\begin{align*}
    A=\begin{pmatrix}
          2 & -5 \\
          1 & -2
      \end{pmatrix} &  & b=\begin{pmatrix}
                               5e^{2t} \\
                               0
                           \end{pmatrix}
\end{align*}
ראשית נמצא פתרון למערכת ההומוגנית $x'=Ax$. המטריצה A בעלת ערכים קבועים ולכן נמצא את ערכיה העצמיים:


\pagebreak

\section*{שאלה 3}

לפנינו המערכת $x'=Ax$ כך ש:
\begin{align*}
    A=\begin{pmatrix}
          2  & -1 & -2 \\
          -1 & 0  & -2 \\
          -2 & 1  & -1
      \end{pmatrix}
\end{align*}
נמצא ערכים עצמיים של המטריצה A:


\pagebreak

\section*{שאלה 4}

לפנינו המערכת $x'=Ax+b$ כך ש:
\begin{align*}
    A=\begin{pmatrix}
          1  & 1 & 2 \\
          0  & 2 & 2 \\
          -1 & 1 & 3
      \end{pmatrix} &  & b=e^{2t}\begin{pmatrix}
                                     -1 \\
                                     -2 \\
                                     0
                                 \end{pmatrix}
\end{align*}
ראשית נמצא פתרון למערכת ההומוגנית $x'=Ax$. נמצא ערכים עצמיים לA:

\pagebreak

\section*{שאלה 5}

לפנינו המערכת $x'=\begin{pmatrix}
        a & b \\
        b & b
    \end{pmatrix}x$ כך ש$a,b$ קבועים ממשיים.\\
עלינו להוכיח שכל רכיביו של פתרון למשוואה זו אפסים באינסוף אם ורק אם $a<b<0$. תחילה נמצא ערכים עצמיים ל-A.

\pagebreak

\section*{שאלה 6}

לפנינו המערכת $x'=Ax+b$ כך ש:
\begin{align*}
    A=\begin{pmatrix}
          \frac{7}{2t}   & \frac{-t}{2}  \\
          \frac{3}{2t^3} & \frac{-1}{2t}
      \end{pmatrix}=\frac{1}{2t^3}\begin{pmatrix}
                                      7t^2 & -t^4 \\
                                      3    & -t^2
                                  \end{pmatrix} &  & b=\begin{pmatrix}
                                                           -t \\
                                                           -t^3
                                                       \end{pmatrix}
\end{align*}
נמצא פתרון למשוואה ההומוגנית המתאימה, $x'=Ax$, שבו הרכיב השני קבוע. נסמן אפוא $x=\begin{pmatrix}
        u(t) \\
        C
    \end{pmatrix}$. מתקיים:
\begin{align*}
    \begin{pmatrix}
        u'(t) \\
        0
    \end{pmatrix}=x' & =Ax=\frac{1}{2t^3}\begin{pmatrix}
                                             7t^2 & -t^4 \\
                                             3    & -t^2
                                         \end{pmatrix}\begin{pmatrix}
                                                          u(t) \\
                                                          C
                                                      \end{pmatrix}=\frac{1}{2t^3}\begin{pmatrix}
                                                                                      7t^2u(t)-Ct^4 \\
                                                                                      3u(t)-Ct^2
                                                                                  \end{pmatrix} \\
                     & \rightarrow \begin{cases}
                                       2t^3\cdot u' = 7t^2u - Ct^4 \\
                                       3u-Ct^2=0
                                   \end{cases} \rightarrow \begin{cases}
                                                               2t^3\cdot u' = t^2(4u+3u-Ct^2) \\
                                                               3u-Ct^2 = 0
                                                           \end{cases}
\end{align*}
נציב את המשוואה השנייה בראשונה ונקבל:
\begin{align*}
    2t^3\cdot u' & = 4t^2u                   \\
    \frac{u'}{u} & = \frac{2}{t}             \\
    \frac{du}{u} & = \frac{2}{t}dt           \\
    \ln|u|       & =2\ln|t|+C_1=\ln(t^2)+C_1 \\
    |u|          & =e^{C_1}t^2               \\
    u            & =\pm e^{C_1}t^2=C_2t^2
\end{align*}
נבחר למשל $u=t^2$, אז $Ct^2=3u=3t^2$ ולכן $C=3$ ו$x_1=\begin{pmatrix}
        t^2 \\
        3
    \end{pmatrix}$ מהווה פתרון ל$x'=Ax$. \\
נמצא פתרון נוסף למערכת ההומוגנית על פי הטכניקה בסעיף 4.2.5. נמצא פונקציה וקטורית $y=\begin{pmatrix}
        y_1 \\
        y_2
    \end{pmatrix}$ ונסמן $x=\begin{pmatrix}
        y_1+t^2y_2 \\
        3y_2
    \end{pmatrix}$. \\
אז:
\begin{align*}
    \begin{pmatrix}
        y_1'+2ty_2+t^2y_2' \\
        3y_2'
    \end{pmatrix}=x' & =Ax=\frac{1}{2t^3}\begin{pmatrix}
                                             7t^2 & -t^4 \\
                                             3    & -t^2
                                         \end{pmatrix}\begin{pmatrix}
                                                          y_1+t^2y_2 \\
                                                          3y_2
                                                      \end{pmatrix}=
    \frac{1}{2t^3}\begin{pmatrix}
                      7t^2y_1+7t^4y_2-3t^4y_2 \\
                      3y_1 + 3t^2y_2-3t^2y_2
                  \end{pmatrix}=\frac{1}{2t^3}\begin{pmatrix}
                                                  7t^2y_1+4t^4y_2 \\
                                                  3y_1
                                              \end{pmatrix}                      \\
    \rightarrow           & \begin{cases}
                                y_1'+2ty_2+t^2y_2' = \frac{7}{2t}y_1+2ty_2 \\
                                y_2'=\frac{1}{2t^3}y_1
                            \end{cases} \rightarrow \begin{cases}
                                                        y_1'+t^2y_2' = \frac{7}{2t}y_1 \\
                                                        y_2'=\frac{1}{2t^3}y_1
                                                    \end{cases}
\end{align*}
נציב את המשוואה השנייה בראשונה ונקבל: \\
\begin{align*}
    y_1'+\frac{t^2}{2t^3}y_1 & =\frac{7}{2t}y_1          \\
    y_1'+\frac{1}{2t}y_1     & =\frac{7}{2t}y_1          \\
    y_1'                     & =\frac{3}{t}y_1           \\
    \frac{dy_1}{y_1}         & =\frac{3}{t}dt            \\
    \ln|y_1|                 & =3\ln|t|+C_3=\ln|t^3|+C_3 \\
    |y_1|                    & =e^{C_3}|t^3|             \\
    y_1                      & = \pm e^{C_3}t^3=C_4t^3
\end{align*}
נבחר למשל $y_1=t^3$, אז $y_2'=\frac{1}{2t^3}y_1=\frac{1}{2}$ ולכן $y_2=\frac{t}{2}+C_5$. נבחר למשל $y_2=\frac{t}{2}$ ונקבל פתרון:
\begin{align*}
    \begin{pmatrix}
        t^3+t^2\cdot \frac{t}{2} \\
        \frac{3t}{2}
    \end{pmatrix}=
    \begin{pmatrix}
        \frac{3t^3}{2} \\
        \frac{3t}{2}
    \end{pmatrix}=\frac{3}{2}\begin{pmatrix}
                                 t^3 \\
                                 t
                             \end{pmatrix}
\end{align*}
ו$x_2=\begin{pmatrix}
        t^3 \\
        t
    \end{pmatrix}$ פתרון נוסף למערכת ההומוגנית.\\
נמצא פתרון פרטי למערכת המקורית בעזרת וריאציית הפרמטרים:

\end{document}