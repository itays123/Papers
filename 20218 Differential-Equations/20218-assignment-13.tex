\documentclass{article}
\usepackage{fontspec}
\newfontfamily\hebrewfont[Script=Hebrew]{Calibri}
\usepackage[left=2.0cm, top=2.0cm, right=2.0cm, bottom=2.0cm]{geometry}
\usepackage{polyglossia}
\usepackage{amsmath}
\usepackage{bidi}
\setdefaultlanguage{hebrew}
\setotherlanguage{english}

\DeclareMathOperator*{\equals}{=}

\title{מטלת מנחה 13 - קורס 20218}
\author{328197462}
\date{08/09/2023}

\begin{document}
\maketitle

\section*{שאלה 1}

נניח כי קיימת מעכת הומונית $x'=Ax$ עבורה הפונקציות הבאות מהוות פתרון:
\begin{align*}
    x_1(t)=e^t\begin{pmatrix}
                  1 \\
                  2 \\
                  3
              \end{pmatrix} &  &
    x_2(t)=e^{-t}\begin{pmatrix}
                     2 \\
                     4 \\
                     6
                 \end{pmatrix}
\end{align*}
אז גם הפונקציה $u=2x_1-x_2$, שהיא צירוף לינארי של פתרונות למערכת ההומוגנית. מהווה פתרון למערכת.\\
מתקיים:
\begin{align*}
    u(0)=2\cdot e^0 \begin{pmatrix}
                        1 \\
                        2 \\
                        3
                    \end{pmatrix} - e^0\begin{pmatrix}
                                           2 \\
                                           4 \\
                                           6
                                       \end{pmatrix} = \begin{pmatrix}
                                                           0 \\
                                                           0 \\
                                                           0
                                                       \end{pmatrix}
\end{align*}
ולכן לפי למה 4.1.7, $u$ הוא הפתרון הטריוויאלי, כלומר $u(t)=0$ לכל $t\in(-1,1)$.
אולם:
\begin{align*}
    u(0.5)=2\cdot e^{0.5} \begin{pmatrix}
                              1 \\
                              2 \\
                              3
                          \end{pmatrix} - e^{-0.5}\begin{pmatrix}
                                                      2 \\
                                                      4 \\
                                                      6
                                                  \end{pmatrix}=
    (e^{0.5}-e^{-0.5})\begin{pmatrix}
                          2 \\
                          4 \\
                          6
                      \end{pmatrix}=\underline{0}
\end{align*}
ונקבל סתירה היות ו $e^{0.5}-e^{-0.5}\ne 0$ וכן $\begin{pmatrix}
        2 \\
        4 \\
        6
    \end{pmatrix}\ne \underline{0}$

\pagebreak

\section*{שאלה 2}

לפנינו המשוואה $x'=Ax+b$ כך ש:
\begin{align*}
    A=\begin{pmatrix}
          2 & -5 \\
          1 & -2
      \end{pmatrix} &  & b=\begin{pmatrix}
                               5e^{2t} \\
                               0
                           \end{pmatrix}
\end{align*}
ראשית נמצא פתרון למערכת ההומוגנית $x'=Ax$. המטריצה A בעלת ערכים קבועים ולכן נמצא את ערכיה העצמיים:
\begin{align*}
    |\lambda I - A| & =\begin{vmatrix}
                           \lambda - 2 & 5         \\
                           -1          & \lambda+2
                       \end{vmatrix}=(\lambda-2)(\lambda+2)-(-1)\cdot 5 = \lambda^2-4+5=\lambda^2+1
\end{align*}
קיבלנו כי למטריצה A שני ערכים עצמיים, $\pm i$. נמצא וקטור עצמי השייך לע"ע $\lambda=i$:
\begin{align*}
    iI-A & =\begin{pmatrix}
                i-2 & 5   \\
                -1  & i+2 \\
            \end{pmatrix}\xrightarrow[]{R_1\rightarrow R_1-2R_2}\begin{pmatrix}
                                                                    i  & 1-2i \\
                                                                    -1 & i+2
                                                                \end{pmatrix}\xrightarrow[]{R_1\rightarrow iR_1}\begin{pmatrix}
                                                                                                                    -1 & i+2 \\
                                                                                                                    -1 & i+2
                                                                                                                \end{pmatrix}
\end{align*}
ועבור $v=\begin{pmatrix}
        a \\
        b
    \end{pmatrix}$ וקטור עצמי נקבל $-a+(i+2)b=0$, כלומר $a=(i+2)b$ ולכן $v=b\begin{pmatrix}
        i+2 \\
        1
    \end{pmatrix}$. נבחר למשל $b=1$ ונקבל 2 פתרונות מרוכבים בת"ל למערכת ההומוגנית:
\begin{align*}
    \begin{cases}
        x(t)=e^{it}v=(\cos t + i\sin t)\begin{pmatrix}
                                           i+2 \\
                                           1
                                       \end{pmatrix} \\
        x^{\ast}(t)=e^{-it}v*=(\cos t - i\sin t)\begin{pmatrix}
                                                    -i+2 \\
                                                    1
                                                \end{pmatrix}
    \end{cases}
\end{align*}
על פי שאלה 4.2.14, החלק הממשי והחלק המדומה של $x(t)$ מהווים זוג פתרונות בת"ל גם הם וניתן להחליף בין הזוגות. אם כן,
\begin{align*}
    x(t) & =(\cos t + i\sin t)(\begin{pmatrix}
                                   2 \\
                                   1
                               \end{pmatrix}+i\begin{pmatrix}
                                                  1 \\
                                                  0
                                              \end{pmatrix}) = \cos t \begin{pmatrix}
                                                                          2 \\
                                                                          1
                                                                      \end{pmatrix} + i\sin t \begin{pmatrix}
                                                                                                  2 \\
                                                                                                  1
                                                                                              \end{pmatrix} + i \cos t \begin{pmatrix}
                                                                                                                           1 \\
                                                                                                                           0
                                                                                                                       \end{pmatrix} - \sin t \begin{pmatrix}
                                                                                                                                                  1 \\
                                                                                                                                                  0
                                                                                                                                              \end{pmatrix} \\
         & = \cos t \begin{pmatrix}
                        2 \\
                        1
                    \end{pmatrix} + \sin t \begin{pmatrix}
                                               -1 \\
                                               0
                                           \end{pmatrix} + i(\cos t \begin{pmatrix}
                                                                        1 \\
                                                                        0
                                                                    \end{pmatrix}+\sin t \begin{pmatrix}
                                                                                             2 \\
                                                                                             1
                                                                                         \end{pmatrix})
\end{align*}
ונקבל זוג פתרונות:
\begin{align*}
    \begin{cases}
        x_1(t)=Re\;x(t)=\cos t \begin{pmatrix}
                                   2 \\
                                   1
                               \end{pmatrix} + \sin t \begin{pmatrix}
                                                          -1 \\
                                                          0
                                                      \end{pmatrix} =\begin{pmatrix}
                                                                         2\cos t - \sin t \\
                                                                         \cos t
                                                                     \end{pmatrix} \\
        x_2(t)=Im\;x(t)=\cos t \begin{pmatrix}
                                   1 \\
                                   0
                               \end{pmatrix}+\sin t \begin{pmatrix}
                                                        2 \\
                                                        1
                                                    \end{pmatrix} = \begin{pmatrix}
                                                                        \cos t + 2\sin t \\
                                                                        \sin t
                                                                    \end{pmatrix}
    \end{cases}
\end{align*}
נמצא פתרון פרטי למערכת המקורית בעזרת וריאציית הפרמטרים:

\pagebreak

\section*{שאלה 3}

לפנינו המערכת $x'=Ax$ כך ש:
\begin{align*}
    A=\begin{pmatrix}
          2  & -1 & -2 \\
          -1 & 0  & -2 \\
          -2 & 1  & -1
      \end{pmatrix}
\end{align*}
נמצא ערכים עצמיים של המטריצה $A$. נפשט את הפ"א בעזרת הרמז שקיבלנו.
\begin{align*}
    |\lambda I -A| & =\begin{vmatrix}
                          \lambda-2 & 1       & 2         \\
                          1         & \lambda & 2         \\
                          2         & -1      & \lambda+1
                      \end{vmatrix}\equals^{R_1\rightarrow R_1+R_3}_{R_2\rightarrow R_2+\lambda R_3} \begin{vmatrix}
                                                                                                         \lambda    & 0  & \lambda+3           \\
                                                                                                         1+2\lambda & 0  & \lambda^2+\lambda+2 \\
                                                                                                         2          & -1 & \lambda+1
                                                                                                     \end{vmatrix} = \\
                   & = \begin{vmatrix}
                           \lambda    & \lambda+3           \\
                           1+2\lambda & \lambda^2+\lambda+2
                       \end{vmatrix} = \lambda^3+\lambda^2+2\lambda - (2\lambda+1)(\lambda+3) =                                            \\
                   & = \lambda^3+\lambda^2+2\lambda - (2\lambda^2+\lambda+6\lambda+3)=                                                     \\
                   & = \lambda^3-\lambda^2-5\lambda-3 =                                                                                    \\
                   & = \lambda^2(\lambda-3)+2\lambda^2-5\lambda-3 =                                                                        \\
                   & = \lambda^2(\lambda-3)+2\lambda^2-6\lambda+\lambda-3                                                                  \\
                   & = \lambda^2(\lambda-3)+2\lambda(\lambda-3)+\lambda-3                                                                  \\
                   & = (\lambda^2+2\lambda+1)(\lambda-3)                                                                                   \\
                   & = (\lambda+1)^2(\lambda-3)                                                                                            \\
\end{align*}
נקבל שני ערכים עצמיים - $\lambda_1=3$ עם ריבוי אלגברי 1, ו$\lambda_2=-1$ עם ריבוי אלגברי 2.
נמצא פתרון מהצורה $e^{\lambda t}v$ לכל ערך עצמי:
\begin{align*}
    3I-A=\begin{pmatrix}
             1 & 1  & 2 \\
             1 & 3  & 2 \\
             2 & -1 & 4
         \end{pmatrix} \rightarrow \begin{pmatrix}
                                       1 & 1  & 2 \\
                                       0 & 2  & 0 \\
                                       2 & -1 & 4
                                   \end{pmatrix} \rightarrow \begin{pmatrix}
                                                                 1 & 1  & 2 \\
                                                                 0 & 1  & 0 \\
                                                                 2 & -1 & 4
                                                             \end{pmatrix} \rightarrow \begin{pmatrix}
                                                                                           1 & 0 & 2 \\
                                                                                           0 & 1 & 0 \\
                                                                                           2 & 0 & 4
                                                                                       \end{pmatrix} \rightarrow \begin{pmatrix}
                                                                                                                     1 & 0 & 2 \\
                                                                                                                     0 & 1 & 0 \\
                                                                                                                     0 & 0 & 0
                                                                                                                 \end{pmatrix}
\end{align*}
ועבור $v=\begin{pmatrix}
        a \\
        b \\
        c
    \end{pmatrix}$ נקבל $b=0, a+2c=0$, כלומר $v=c\begin{pmatrix}
        -2 \\
        0  \\
        1
    \end{pmatrix}$. נבחר למשל $c=-1$ ונקבל פתרון למערכת $x_1(t)=e^{3t}\begin{pmatrix}
        2 \\
        0 \\
        -1
    \end{pmatrix}$

\begin{align*}
    -I-A=\begin{pmatrix}
             -3 & 1  & 2 \\
             1  & -1 & 2 \\
             2  & -1 & 0
         \end{pmatrix} \rightarrow \begin{pmatrix}
                                       -2 & 0  & 4 \\
                                       1  & -1 & 2 \\
                                       2  & -1 & 0
                                   \end{pmatrix} \rightarrow \begin{pmatrix}
                                                                 -2 & 0  & 4 \\
                                                                 1  & -1 & 2 \\
                                                                 0  & -1 & 4
                                                             \end{pmatrix} \rightarrow \begin{pmatrix}
                                                                                           1 & 0  & -2 \\
                                                                                           1 & -1 & 2  \\
                                                                                           0 & -1 & 4
                                                                                       \end{pmatrix}\rightarrow \begin{pmatrix}
                                                                                                                    1 & 0  & -2 \\
                                                                                                                    1 & 0  & -2 \\
                                                                                                                    0 & -1 & 4
                                                                                                                \end{pmatrix}
\end{align*}
ועבור $v=\begin{pmatrix}
        a \\
        b \\
        c
    \end{pmatrix}$ נקבל $a-2c=0, -b+4c=0$, כלומר $v=c\begin{pmatrix}
        2 \\
        4 \\
        1
    \end{pmatrix}$. נבחר למשל $c=1$ ונקבל פתרון למערכת $x_2(t)=e^{-t}\begin{pmatrix}
        2 \\
        4 \\
        1
    \end{pmatrix}$\\
עבור הפתרון השלישי במערכת הפתרונות נמצא וקטור $w$ כך ש$(-I-A)w=\begin{pmatrix}
        2 \\
        4 \\
        1
    \end{pmatrix}$. נחזור על פעולות הדירוג שביצענו קודם לכן:
\begin{align*}
    \begin{pmatrix}
        2 \\
        4 \\
        1
    \end{pmatrix}\rightarrow\begin{pmatrix}
                                6 \\
                                4 \\
                                1
                            \end{pmatrix}\rightarrow\begin{pmatrix}
                                                        6 \\
                                                        4 \\
                                                        7
                                                    \end{pmatrix}\rightarrow\begin{pmatrix}
                                                                                -3 \\
                                                                                4  \\
                                                                                7
                                                                            \end{pmatrix}\rightarrow\begin{pmatrix}
                                                                                                        -3 \\
                                                                                                        -3 \\
                                                                                                        7
                                                                                                    \end{pmatrix}
\end{align*}
ועבור $w=\begin{pmatrix}
        a
        b
        c
    \end{pmatrix}$ מקבלים $a-2c=-3, -b+4c=7$ ולכן $w=\begin{pmatrix}
        -3 \\
        -7 \\
        0
    \end{pmatrix}+c\begin{pmatrix}
        2 \\
        4 \\
        1
    \end{pmatrix}$. נבחר למשל $c=2$ ונקבל $w=\begin{pmatrix}
        1 \\
        1 \\
        2
    \end{pmatrix}$. \\
פתרון נוסף למערכת יהיה $x_3(t)=te^{-t}\begin{pmatrix}
        2 \\
        4 \\
        1
    \end{pmatrix}+e^{-t}\begin{pmatrix}
        1 \\
        1 \\
        2
    \end{pmatrix}=e^{-t}\begin{pmatrix}
        2t+1 \\
        4t+1 \\
        t+2
    \end{pmatrix}$, והפתרון הכללי יהיה:
\begin{align*}
    x=C_1e^{3t}\begin{pmatrix}
                   2 \\
                   0 \\
                   -1
               \end{pmatrix}
    +e^{-t}(C_2\begin{pmatrix}
                   2 \\
                   4 \\
                   1
               \end{pmatrix}+C_3\begin{pmatrix}
                                    2t+1 \\
                                    4t+1 \\
                                    t+2
                                \end{pmatrix})
\end{align*}

\pagebreak

\section*{שאלה 4}

לפנינו המערכת $x'=Ax+b$ כך ש:
\begin{align*}
    A=\begin{pmatrix}
          1  & 1 & 2 \\
          0  & 2 & 2 \\
          -1 & 1 & 3
      \end{pmatrix} &  & b=e^{2t}\begin{pmatrix}
                                     -1 \\
                                     -2 \\
                                     0
                                 \end{pmatrix}
\end{align*}
ראשית נמצא פתרון למערכת ההומוגנית $x'=Ax$. נמצא ערכים עצמיים לA:
\begin{align*}
    |\lambda I - A| & =\begin{vmatrix}
                           \lambda-1 & -1        & -2        \\
                           0         & \lambda-2 & -2        \\
                           1         & -1        & \lambda-3
                       \end{vmatrix}\equals^{R_2\rightarrow R_2-R_1}\begin{vmatrix}
                                                                        \lambda-1 & -1        & -2        \\
                                                                        1-\lambda & \lambda-1 & 0         \\
                                                                        1         & -1        & \lambda-3
                                                                    \end{vmatrix}=(\lambda-1)\begin{vmatrix}
                                                                                                 \lambda-1 & -1 & -2        \\
                                                                                                 -1        & 1  & 0         \\
                                                                                                 1         & -1 & \lambda-3
                                                                                             \end{vmatrix}= \\
                    & \equals^{R_2\rightarrow R_2+R_3} (\lambda-1)\begin{vmatrix}
                                                                      \lambda-1 & -1 & -2        \\
                                                                      0         & 0  & \lambda-3 \\
                                                                      1         & -1 & \lambda-3
                                                                  \end{vmatrix}=-(\lambda-1)(\lambda-3)\begin{vmatrix}
                                                                                                           \lambda-1 & -1 \\
                                                                                                           1         & -1
                                                                                                       \end{vmatrix} =  \\
                    & = -(\lambda-1)(\lambda-3)(-\lambda+1-(-1)\cdot 1)=(\lambda-1)(\lambda-2)(\lambda-3)
\end{align*}
קיבלנו 3 ע"ע. נמצא 3 פתרונות למשוואה ההומוגנית מהצורה $e^{\lambda t}v$.
\begin{align*}
    I-A=\begin{pmatrix}
            0 & -1 & -2 \\
            0 & -1 & -2 \\
            1 & -1 & -2
        \end{pmatrix}-\cdots\rightarrow\begin{pmatrix}
                                           0 & 1 & 2 \\
                                           1 & 0 & 0 \\
                                           0 & 0 & 0
                                       \end{pmatrix}
\end{align*}
ועבור וקטור עצמי $v=\begin{pmatrix}
        a \\
        b \\
        c
    \end{pmatrix}$ השייך לע"ע $\lambda=1$ נקבל $b+2c=0, a=0$ ולכן $v=c\begin{pmatrix}
        0  \\
        -2 \\
        1
    \end{pmatrix}$. \\ נבחר למשל $c=-1$ ונקבל $x_1(t)=e^t\begin{pmatrix}
        0 \\
        2 \\
        -1
    \end{pmatrix}$ פתרון למערכת ההומוגנית.
\begin{align*}
    2I-A=\begin{pmatrix}
             1 & -1 & 2  \\
             0 & 0  & -2 \\
             1 & -1 & -1
         \end{pmatrix} \rightarrow \begin{pmatrix}
                                       1 & -1 & 2  \\
                                       0 & 0  & 1  \\
                                       1 & -1 & -1
                                   \end{pmatrix} - \cdots \rightarrow \begin{pmatrix}
                                                                          1 & -1 & 0 \\
                                                                          0 & 0  & 1 \\
                                                                          0 & 0  & 0
                                                                      \end{pmatrix}
\end{align*}
ועבור וקטור עצמי $v=\begin{pmatrix}
        a \\
        b \\
        c
    \end{pmatrix}$ השייך לע"ע $\lambda=2$ נקבל $a-b=0, c=0$ ולכן $v=a\begin{pmatrix}
        1 \\
        1 \\
        0
    \end{pmatrix}$. \\ נבחר למשל $a=1$ ונקבל $x_2(t)=e^{2t}\begin{pmatrix}
        1 \\
        1 \\
        0
    \end{pmatrix}$ פתרון למערכת ההומוגנית.
\begin{align*}
    3I-A=\begin{pmatrix}
             2 & -1 & -2 \\
             0 & 1  & -2 \\
             1 & -1 & 0
         \end{pmatrix}\rightarrow \begin{pmatrix}
                                      2 & -2 & 0  \\
                                      0 & 1  & -2 \\
                                      1 & -1 & 0
                                  \end{pmatrix}-\cdots\rightarrow \begin{pmatrix}
                                                                      1 & -1 & 0  \\
                                                                      0 & 1  & -2 \\
                                                                      0 & 0  & 0
                                                                  \end{pmatrix}
\end{align*}
ועבור וקטור עצמי $v=\begin{pmatrix}
        a \\
        b \\
        c
    \end{pmatrix}$ השייך לע"ע $\lambda=3$ נקבל $a-b=0, b-2c=0$ ולכן $v=c\begin{pmatrix}
        2 \\
        2 \\
        1
    \end{pmatrix}$. \\ נבחר למשל $c=1$ ונקבל $x_3(t)=e^{3t}\begin{pmatrix}
        2 \\
        2 \\
        1
    \end{pmatrix}$ פתרון למערכת ההומוגנית.\\
נמצא פתרון פרטי למערכת המקורית בעזרת וריאציית הפרמטרים.

\pagebreak

\section*{שאלה 5}

לפנינו המערכת $x'=\begin{pmatrix}
        a & b \\
        b & b
    \end{pmatrix}x$ כך ש$a,b$ קבועים ממשיים.\\
עלינו להוכיח שכל רכיביו של פתרון למשוואה זו אפסים באינסוף אם ורק אם $a<b<0$. תחילה נמצא ערכים עצמיים ל-A.
\begin{align*}
    |\lambda I - A|=\begin{vmatrix}
                        \lambda-a & -b        \\
                        -b        & \lambda-b
                    \end{vmatrix}=(\lambda-a)(\lambda-b)-(-b)(-b)=\lambda^2-(a+b)\lambda+ab-b^2
\end{align*}
נמצא שורשים לפ"א:
\begin{align*}
    \Delta=(a+b)^2-4(ab-b^2)=a^2+2ab+b^2-4ab+4b^2=a^2-2ab+b^2+4b^2=(a-b)^2+4b^2\geq 0
\end{align*}
נחלק למקרים. אילו $\Delta=0$, כלומר יש פתרון יחיד למשוואה וע"ע יחיד למטריצה, הדבר מחייב $b=0, a-b=0$ ולכן $a=b=0$.\\
במקרה זה נקבל פתרונות מהצורה $C_1\begin{pmatrix}
        0 \\
        1
    \end{pmatrix}+C_2\begin{pmatrix}
        1 \\
        0
    \end{pmatrix}$, שעבור $C_1\ne 0$ או $C_2\ne 0$ יכילו רכיבים שאינם אפסים באינסוף.\\
נתייחס למקרה בו יש שני שורשים שונים, $\lambda_1, \lambda_2$. נקבל מערכת של פתרונות $e^{\lambda_1t}v_1, e^{\lambda_2t}v_2$ כאשר $v_1, v_2$ וקטורים עצמיים השייכים לע"ע $\lambda_1, \lambda_2$ בהתאמה.\\
היות ו$v_1, v_2$ וקטורים קבועים, רכיבי שני פתרונות אלה אפסים באינסוף אם ורק אם $\lambda_1, \lambda_2<0$. במקרה זה כל פתרונות המערכת יהיו צירוף לינארי של שני הפתרונות שלהם וגם הם יהיו אפסים באינסוף.\\
במילים אחרות, כל רכיב בכל פתרון למערכת אפס באינסוף אם ורק אם $\lambda_1, \lambda_2<0$

\pagebreak

\section*{שאלה 6}

לפנינו המערכת $x'=Ax+b$ כך ש:
\begin{align*}
    A=\begin{pmatrix}
          \frac{7}{2t}   & \frac{-t}{2}  \\
          \frac{3}{2t^3} & \frac{-1}{2t}
      \end{pmatrix}=\frac{1}{2t^3}\begin{pmatrix}
                                      7t^2 & -t^4 \\
                                      3    & -t^2
                                  \end{pmatrix} &  & b=\begin{pmatrix}
                                                           -t \\
                                                           -t^3
                                                       \end{pmatrix}
\end{align*}
נמצא פתרון למשוואה ההומוגנית המתאימה, $x'=Ax$, שבו הרכיב השני קבוע. נסמן אפוא $x=\begin{pmatrix}
        u(t) \\
        C
    \end{pmatrix}$. מתקיים:
\begin{align*}
    \begin{pmatrix}
        u'(t) \\
        0
    \end{pmatrix}=x' & =Ax=\frac{1}{2t^3}\begin{pmatrix}
                                             7t^2 & -t^4 \\
                                             3    & -t^2
                                         \end{pmatrix}\begin{pmatrix}
                                                          u(t) \\
                                                          C
                                                      \end{pmatrix}=\frac{1}{2t^3}\begin{pmatrix}
                                                                                      7t^2u(t)-Ct^4 \\
                                                                                      3u(t)-Ct^2
                                                                                  \end{pmatrix} \\
                     & \rightarrow \begin{cases}
                                       2t^3\cdot u' = 7t^2u - Ct^4 \\
                                       3u-Ct^2=0
                                   \end{cases} \rightarrow \begin{cases}
                                                               2t^3\cdot u' = t^2(4u+3u-Ct^2) \\
                                                               3u-Ct^2 = 0
                                                           \end{cases}
\end{align*}
נציב את המשוואה השנייה בראשונה ונקבל:
\begin{align*}
    2t^3\cdot u' & = 4t^2u                   \\
    \frac{u'}{u} & = \frac{2}{t}             \\
    \frac{du}{u} & = \frac{2}{t}dt           \\
    \ln|u|       & =2\ln|t|+C_1=\ln(t^2)+C_1 \\
    |u|          & =e^{C_1}t^2               \\
    u            & =\pm e^{C_1}t^2=C_2t^2
\end{align*}
נבחר למשל $u=t^2$, אז $Ct^2=3u=3t^2$ ולכן $C=3$ ו$x_1=\begin{pmatrix}
        t^2 \\
        3
    \end{pmatrix}$ מהווה פתרון ל$x'=Ax$. \\
נמצא פתרון נוסף למערכת ההומוגנית על פי הטכניקה בסעיף 4.2.5. נמצא פונקציה וקטורית $y=\begin{pmatrix}
        y_1 \\
        y_2
    \end{pmatrix}$ ונסמן $x=\begin{pmatrix}
        y_1+t^2y_2 \\
        3y_2
    \end{pmatrix}$. \\
אז:
\begin{align*}
    \begin{pmatrix}
        y_1'+2ty_2+t^2y_2' \\
        3y_2'
    \end{pmatrix}=x' & =Ax=\frac{1}{2t^3}\begin{pmatrix}
                                             7t^2 & -t^4 \\
                                             3    & -t^2
                                         \end{pmatrix}\begin{pmatrix}
                                                          y_1+t^2y_2 \\
                                                          3y_2
                                                      \end{pmatrix}=
    \frac{1}{2t^3}\begin{pmatrix}
                      7t^2y_1+7t^4y_2-3t^4y_2 \\
                      3y_1 + 3t^2y_2-3t^2y_2
                  \end{pmatrix}=\frac{1}{2t^3}\begin{pmatrix}
                                                  7t^2y_1+4t^4y_2 \\
                                                  3y_1
                                              \end{pmatrix}                      \\
    \rightarrow           & \begin{cases}
                                y_1'+2ty_2+t^2y_2' = \frac{7}{2t}y_1+2ty_2 \\
                                y_2'=\frac{1}{2t^3}y_1
                            \end{cases} \rightarrow \begin{cases}
                                                        y_1'+t^2y_2' = \frac{7}{2t}y_1 \\
                                                        y_2'=\frac{1}{2t^3}y_1
                                                    \end{cases}
\end{align*}
נציב את המשוואה השנייה בראשונה ונקבל: \\
\begin{align*}
    y_1'+\frac{t^2}{2t^3}y_1 & =\frac{7}{2t}y_1          \\
    y_1'+\frac{1}{2t}y_1     & =\frac{7}{2t}y_1          \\
    y_1'                     & =\frac{3}{t}y_1           \\
    \frac{dy_1}{y_1}         & =\frac{3}{t}dt            \\
    \ln|y_1|                 & =3\ln|t|+C_3=\ln|t^3|+C_3 \\
    |y_1|                    & =e^{C_3}|t^3|             \\
    y_1                      & = \pm e^{C_3}t^3=C_4t^3
\end{align*}
נבחר למשל $y_1=t^3$, אז $y_2'=\frac{1}{2t^3}y_1=\frac{1}{2}$ ולכן $y_2=\frac{t}{2}+C_5$. נבחר למשל $y_2=\frac{t}{2}$ ונקבל פתרון:
\begin{align*}
    \begin{pmatrix}
        t^3+t^2\cdot \frac{t}{2} \\
        \frac{3t}{2}
    \end{pmatrix}=
    \begin{pmatrix}
        \frac{3t^3}{2} \\
        \frac{3t}{2}
    \end{pmatrix}=\frac{3}{2}\begin{pmatrix}
                                 t^3 \\
                                 t
                             \end{pmatrix}
\end{align*}
ו$x_2=\begin{pmatrix}
        t^3 \\
        t
    \end{pmatrix}$ פתרון נוסף למערכת ההומוגנית.\\
נמצא פתרון פרטי למערכת המקורית בעזרת וריאציית הפרמטרים:

\end{document}