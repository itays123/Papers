\documentclass{article}
\usepackage{fontspec}
\newfontfamily\hebrewfont[Script=Hebrew]{Calibri}
\usepackage{polyglossia}
\usepackage{amsmath, amssymb}
\usepackage[left=2.0cm, top=2.0cm, right=2.0cm, bottom=2.0cm]{geometry}
\usepackage{bidi}
\setdefaultlanguage{hebrew}
\setotherlanguage{english}

\title{מטלת מנחה 14 - אלגברה לינארית 2}
\author{328197462}
\date{12/05/2023}

\def\reals{\mathbb{R}}
\def\complex{\mathbb{C}}
\def\field{\mathbb{F}}
\DeclareMathOperator*{\equals}{=}
\DeclareMathOperator{\trace}{tr}
\DeclareMathOperator{\adj}{^\ast}
\DeclareMathOperator{\tra}{^t}
\DeclareMathOperator{\inv}{^{-1}}
\DeclareMathOperator{\perc}{^\perp}
\DeclareMathOperator{\Sp}{Sp}
\DeclareMathOperator{\Image}{Im}
\DeclareMathOperator{\diag}{diag}

\begin{document}
\maketitle

\section*{שאלה 1}
\subsection*{סעיף א}

המטריצה האלכסונית המייצגת של התבנית $q$ לפי הבסיס הסטנדרטי היא:
\[
    [q]_E = \begin{pmatrix}
        0   & 1/2 & 1   & 3/2 \\
        1/2 & 0   & 1/2 & 1   \\
        1   & 1/2 & 0   & 1/2 \\
        3/2 & 1   & 1/2 & 0
    \end{pmatrix}
\]
נבצע חפיפות אלמנטריות:
\begin{align*}
    ([q]_E | I) & = \left(
    \begin{matrix}
            0   & 1/2 & 1   & 3/2 \\
            1/2 & 0   & 1/2 & 1   \\
            1   & 1/2 & 0   & 1/2 \\
            3/2 & 1   & 1/2 & 0
        \end{matrix}
    \left|
    \begin{matrix}
            1 & 0 & 0 & 0 \\
            0 & 1 & 0 & 0 \\
            0 & 0 & 1 & 0 \\
            0 & 0 & 0 & 1
        \end{matrix}
    \right.
    \right)
    \xrightarrow[]{R_i\rightarrow 2R_i}
    \left(
    \begin{matrix}
            0 & 2 & 4 & 6 \\
            2 & 0 & 2 & 4 \\
            4 & 2 & 0 & 2 \\
            6 & 4 & 2 & 0
        \end{matrix}
    \left|
    \begin{matrix}
            2 & 0 & 0 & 0 \\
            0 & 2 & 0 & 0 \\
            0 & 0 & 2 & 0 \\
            0 & 0 & 0 & 2
        \end{matrix}
    \right.
    \right)
    \xrightarrow[]{R_1\rightarrow R_1+R_2}                                     \\
                & \rightarrow
    \left(
    \begin{matrix}
            4  & 2 & 6 & 10 \\
            2  & 0 & 2 & 4  \\
            6  & 2 & 0 & 2  \\
            10 & 4 & 2 & 0
        \end{matrix}
    \left|
    \begin{matrix}
            2 & 2 & 0 & 0 \\
            0 & 2 & 0 & 0 \\
            0 & 0 & 2 & 0 \\
            0 & 0 & 0 & 2
        \end{matrix}
    \right.
    \right)
    \xrightarrow[R_3\rightarrow R_3-3/2R_1, R_4\rightarrow R_4-5/2R_1]{R_2\rightarrow R_2-1/2R_1}
    \left(
    \begin{matrix}
            4 & 0  & 0   & 0   \\
            0 & -1 & -1  & -1  \\
            0 & -1 & -9  & -13 \\
            0 & -1 & -13 & -25
        \end{matrix}
    \left|
    \begin{matrix}
            2  & 2  & 0 & 0 \\
            -1 & 1  & 0 & 0 \\
            -3 & -3 & 2 & 0 \\
            -5 & -5 & 0 & 2
        \end{matrix}
    \right.
    \right)\rightarrow                                                         \\
                & \xrightarrow[R_4\rightarrow R_4-R_2]{R_3\rightarrow R_3-R_2}
    \left(
    \begin{matrix}
            4 & 0  & 0   & 0   \\
            0 & -1 & 0   & 0   \\
            0 & 0  & -8  & -12 \\
            0 & 0  & -12 & -24
        \end{matrix}
    \left|
    \begin{matrix}
            2  & 2  & 0 & 0 \\
            -1 & 1  & 0 & 0 \\
            -2 & -4 & 2 & 0 \\
            -4 & -6 & 0 & 2
        \end{matrix}
    \right.
    \right)
    \xrightarrow[]{R_4\rightarrow R_4-3/2R_4}
    \left(
    \begin{matrix}
            4 & 0  & 0  & 0  \\
            0 & -1 & 0  & 0  \\
            0 & 0  & -8 & 0  \\
            0 & 0  & 0  & -6
        \end{matrix}
    \left|
    \begin{matrix}
            2  & 2  & 0  & 0 \\
            -1 & 1  & 0  & 0 \\
            -2 & -4 & 2  & 0 \\
            -1 & 0  & -3 & 2
        \end{matrix}
    \right.
    \right)
\end{align*}
ומכאן שבבסיס $B=(v_1=(2, 2, 0, 0), v_2=(-1, 1, 0, 0), v_3=(-2, -4, 2, 0), v_4=(-1, 0, -3, 2))$ נקבל $q=4x_1^2-x_2^2-8x_3^2-6x_4^2$. \\
הדרגה של $q$ היא 4 והסימנית $\sigma=1-3=-2$.
\subsection*{סעיף ב}

נרצה להוכיח כי $U=\Sp\{ v_1 \}$ הוא תת המרחב המקסימלי עליו $q$ חיובית. תחילה נוכיח כי $q$ חיובית על מרחב זה. \\
יהא $u\in U$, אז $[u]_B=(\alpha, 0, 0, 0)$ עבור $\alpha\in \reals$ כלשהו ולכן $q(u)=4\alpha^2>0$. \\
המרחב המשלים $U'=\Sp\{ v_2, v_3, v_4 \}$ מקיים ש$q$ שלילית לחלוטין עליו (ההוכחה זהה לחלוטין).\\
כעת, נניח בשלילה כי קיים מרחב $W$ כך ש $\dim W > 1$ וגם $q$ חיובית לחלוטין על $W$. \\
משיקולי מימדים, בהכרח $\dim (U'\cap W)>0$ ולכן קיים וקטור שאינו אפס ב$v\in U'\cap W$. \\
עבור וקטור זה מתקיים $v\in U'$ $\Leftarrow$ $q(v)<0$, אבל גם $v\in W$ ולכן $q(v)>0$ וזו סתירה!

\pagebreak

\section*{שאלה 2}

יהא $L_0$ תת הקבוצה הנתונה. נוכיח כי תת-קבוצה זו מהווה תת מרחב ממימד $n-\rho$.
נתבונן בצורה הקנונית של $q$. על פי 6.1.1 ו6.3.2, קיים בסיס $(w)=(w_1, w_2, ..., w_n)$ כלשהו של $V$ כך ש:
\[
    [q]_{(w)} = \begin{pmatrix}
        I_{\rho} & 0 \\
        0        & 0
    \end{pmatrix} = \diag\{1, 1, 1, ..., 1, 0, 0, ..., 0\}
\]
כלומר, לכל $v\in V$ כך ש $[v]_{(w)}=(x_1, x_2, ..., x_n)\tra$, מקבלים:
\[
    q(x_1, x_2, ..., x_n) = x_1^2+x_2^2+\cdots+x_\rho^2+0x_{\rho+1}^2+\cdots+0x_n^2
\]
נתבונן אפוא בתת המרחב $U=\Sp\{ w_{\rho+1}, ..., w_n \}$ ממימד $n-\rho$. נוכיח כי קבוצת איברי $W$ היא בדיוק $L_0$.\\\\
כיוון ראשון: יהא $u\in U$, אז עבור $[u]_{(w)}=(x_1, x_2, ..., x_n)\tra$ נקבל $x_1=x_2=\cdots=x_\rho=0$. לכן:
\[
    q(u)=0^2+0^2+\cdots+0^2+0x_{\rho+1}^2+\cdots+0x_n^2=0
\]
ומכאן $u\in L_0$ ולכן $U\subseteq L_0$.\\\\
כיוון שני: יהא $s\in L_0$ ונסמן $[s]_{(w)}=(s_1, s_2, ..., s_n) \tra$. \\
על פי הנתון נסיק:
\[
    q(s)=s_1^2+s_2^2+\cdots+s_\rho^2=0
\]
מכאן בהכרח $s_1=s_2=\cdots=s_\rho=0$ ולכן $s\in \Sp\{  w_{\rho+1}, ..., w_n \}=U$ ונקבל $L_0\subseteq U$. \\
קיבלנו ש$L_0$ הוא בדיוק תת-המרחב $U$ ממימד $n-\rho$ ותמה הוכחת השאלה.

\section*{שאלה 3}

נוכיח את השאלה על דרך השלילה. \\
נניח בשלילה כי $q$ אינה שומרת סימן.
בהכרח, על פי 6.3.2, בהצגה הקנונית של $q$ על פי בסיס $(w)=(w_1, w_2, ..., w_n)$, נקבל לפחות איבר אחד בעל מקדם 1 שנסמנו $x_1$, ולפחות איבר אחד בעל מקדם $(-1)$ שנסמנו $x_{\pi +1}$. ההצגה תהיה, בסימוני 6.3.2,
\[
    q(x_1, x_2, ..., x_n)=x_1^2+\cdots+x_{\pi}^2-x_{\pi+1}^2-\cdots-x_{\rho}^2+0x_{\rho+1}^2+\cdots+0x_n^2
\]
\\
יהא $u_1=w_1$, אז $[u_1]_{(w)}=(1, 0, ..., 0)\tra$ ולכן $q(u_1)=1$ ו$u_2\in L$. \\
יהא $u_2=w_{\pi+1}+2w_1$. אז $q(u_1)=2^2-1^2=3$ ולכן $u_2\in L$ \\
אבל $q(u_2-2u_1)=q(w_{\pi+1})=-1$ ולכן $u_2-2u_1\notin L$ בסתירה לתכונת הסגירות לחיבור של המרחב הלינארי $L$!

\pagebreak

\section*{שאלה 4}

\subsection*{סעיף א}

המטריצה המייצגת של $q$ לפי הבסיס הסטנדרטי תהיה:

\[
    [q]_E = \begin{pmatrix}
        1       & \lambda & 5 \\
        \lambda & 4       & 3 \\
        5       & 3       & 1
    \end{pmatrix}
\]
נשתמש בשיטת יעקובי על מנת למצוא תנאים הכרחיים ומספיקים לחיוביות של $q$: זוהי מסקנה 6.4.3. נקבל אפוא - תנאי הכרחי ומספיק לחיוביות של $q$ יהיה סיפוקם של שלושת אי השוויונות הבאים:
\begin{align*}
    \Delta_1 & =|[1]|=1>0                                                            \\
    \Delta_2 & = \begin{vmatrix}
                     1       & \lambda \\
                     \lambda & 4
                 \end{vmatrix} = 4-\lambda^2>0                                       \\
    \Delta_3 & = \begin{vmatrix}
                     1       & \lambda & 5 \\
                     \lambda & 4       & 3 \\
                     5       & 3       & 1
                 \end{vmatrix} = 1\begin{vmatrix}
                                      4 & 3 \\
                                      3 & 1
                                  \end{vmatrix}-\lambda\begin{vmatrix}
                                                           \lambda & 3 \\
                                                           5       & 1
                                                       \end{vmatrix}+5\begin{vmatrix}
                                                                          \lambda & 4 \\
                                                                          5       & 3
                                                                      \end{vmatrix}= \\
             & = 1(4-9)-\lambda(\lambda-15)+5(3\lambda-20)=                          \\
             & = -\lambda^2+30\lambda-105 > 0
\end{align*}
נקבל $\Delta_2>0$ אם ורק אם $-2<\lambda<2$.\\
כמו כן, הערכים $\lambda=15\pm2\sqrt{30}$ מאפסים את $\Delta_3$ ו$\Delta_3>0$ אם ורק אם $4.05\approx 15-2\sqrt{30}<\lambda<15+2\sqrt{30}\approx25.95$.\\
קיבלנו שני אי-שוויונות שלא ניתן לספק במקביל עבור שום ערך של $\lambda$ ולכן $q$ אינה חיובית לחלוטין עבור שום ערך של $\lambda$.

\subsection*{סעיף ב}

הסעיף עוסק בשיטת יעקובי וביישום מרכזי שלה - לכסון סימולטני. \\
בסימוני 6.5.1':
\begin{align*}
    A=[q_2]_E=\begin{pmatrix}
                  2 & 4 & 1 \\
                  4 & 8 & 2 \\
                  1 & 2 & 3
              \end{pmatrix} &  & B=[q_1]_E=\begin{pmatrix}
                                               1  & 1 & -1 \\
                                               1  & 2 & 0  \\
                                               -1 & 0 & 3
                                           \end{pmatrix}
\end{align*}
שלבי הפתרון הם כלהלן, על פי הוכחת משפט 6.5.1':
\begin{itemize}
    \item נמצא מטריצה $P$ כך ש $P\tra B P=I$.
    \item נגדיר $S=P\tra A P$. המטריצה $S$ תהא סימטרית ממשית ולכן לכסינה אורתוגונלית.
    \item נמצא מטריצה $Q$ אורתוגונלית כך ש $Q \adj S Q=\diag\{ \delta_1, \delta_2, \delta_3 \}$
    \item המטריצה המלכסנת שלנו תהיה $M=PQ$ ונקבל $q_1=\delta_1y_1^2\delta_2y_2^2+\delta_3y_3^2$
\end{itemize}
נעבור לפתרון. נמצא את $P$ בעזרת חפיפה אלמנטרית:
\begin{align*}
    (B | I) & = \left(
    \begin{matrix}
            1  & 1 & -1 \\
            1  & 2 & 0  \\
            -1 & 0 & 3
        \end{matrix}
    \left|
    \begin{matrix}
            1 & 0 & 0 \\
            0 & 1 & 0 \\
            0 & 0 & 1
        \end{matrix}
    \right.
    \right)
    \xrightarrow[R_3\rightarrow R_3+R_1]{R_2\rightarrow R_2-R_1}
    \left(
    \begin{matrix}
            1 & 1 & -1 \\
            0 & 1 & 1  \\
            0 & 1 & 2
        \end{matrix}
    \left|
    \begin{matrix}
            1  & 0 & 0 \\
            -1 & 1 & 0 \\
            1  & 0 & 1
        \end{matrix}
    \right.
    \right)
    \rightarrow
    \left(
    \begin{matrix}
            1 & 0 & 0 \\
            0 & 1 & 1 \\
            0 & 1 & 2
        \end{matrix}
    \left|
    \begin{matrix}
            1  & 0 & 0 \\
            -1 & 1 & 0 \\
            1  & 0 & 1
        \end{matrix}
    \right.
    \right)
    \rightarrow                                      \\
            & \xrightarrow[]{R_3\rightarrow R_3-R_2}
    \left(
    \begin{matrix}
            1 & 0 & 0 \\
            0 & 1 & 1 \\
            0 & 0 & 1
        \end{matrix}
    \left|
    \begin{matrix}
            1  & 0  & 0 \\
            -1 & 1  & 0 \\
            2  & -1 & 1
        \end{matrix}
    \right.
    \right)
    \rightarrow
    \left(
    \begin{matrix}
            1 & 0 & 0 \\
            0 & 1 & 0 \\
            0 & 0 & 1
        \end{matrix}
    \left|
    \begin{matrix}
            1  & 0  & 0 \\
            -1 & 1  & 0 \\
            2  & -1 & 1
        \end{matrix}
    \right.
    \right)=(I | P\tra)
\end{align*}
נקבל אפוא כי $B$ אכן חיובית לחלוטין וכן $P=\begin{pmatrix}
        1 & -1 & 2  \\
        0 & 1  & -1 \\
        0 & 0  & 1
    \end{pmatrix}$

\pagebreak

נחשב את $S$:
\begin{align*}
    S=P\tra A P = \begin{pmatrix}
                      1  & 0  & 0 \\
                      -1 & 1  & 0 \\
                      2  & -1 & 1
                  \end{pmatrix} \begin{pmatrix}
                                    2 & 4 & 1 \\
                                    4 & 8 & 2 \\
                                    1 & 2 & 3
                                \end{pmatrix} \begin{pmatrix}
                                                  1 & -1 & 2  \\
                                                  0 & 1  & -1 \\
                                                  0 & 0  & 1
                                              \end{pmatrix} =
    \begin{pmatrix}
        2 & 4 & 1 \\
        2 & 4 & 1 \\
        1 & 2 & 3
    \end{pmatrix} \begin{pmatrix}
                      1 & -1 & 2  \\
                      0 & 1  & -1 \\
                      0 & 0  & 1
                  \end{pmatrix} = \begin{pmatrix}
                                      2 & 2 & 1 \\
                                      2 & 2 & 1 \\
                                      1 & 1 & 3
                                  \end{pmatrix}
\end{align*}
המטריצה $S$ היא מטריצה סימטרית ממשית ולכן לכסינה אורתוגונלית. נמצא את הערכים העצמיים של $S$:
\begin{align*}
    p(x)=|xI-S| & =\begin{vmatrix}
                       x-2 & -2  & -1  \\
                       -2  & x-2 & -1  \\
                       -1  & -1  & x-3
                   \end{vmatrix} \equals^{R_1=\Sigma R_i}
    \begin{vmatrix}
        x-5 & x-5 & x-5 \\
        -2  & x-2 & -1  \\
        -1  & -1  & x-3
    \end{vmatrix} =                                                                   \\
                & = (x-5)\begin{vmatrix}
                             1  & 1   & 1   \\
                             -2 & x-2 & -1  \\
                             -1 & -1  & x-3
                         \end{vmatrix} \equals^{R_3\rightarrow R_3+R_1}
    (x-5)\begin{vmatrix}
             1  & 1   & 1   \\
             -2 & x-2 & -1  \\
             0  & 0   & x-2
         \end{vmatrix} =                                                              \\
                & = (x-5)(x-2)\begin{vmatrix}
                                  1  & 1   \\
                                  -2 & x-2
                              \end{vmatrix} = (x-5)(x-2)(x-2-(-2\cdot 1))=x(x-5)(x-2)
\end{align*}
נקבל שלושה ערכים עצמיים $0, 2, 5$ עם ריבוי אלגברי וגיאומטרי 1. נמצא בסיסים אורתוגונליים לכל אחד משלוש המרחבים העצמיים $V_0, V_2, V_5$:
\begin{itemize}
    \item עבור $V_0$ נקבל את מרחב האפס של $S$. דירוג ייתן לנו: \[
              \begin{pmatrix}
                  2 & 2 & 1 \\
                  2 & 2 & 1 \\
                  1 & 1 & 3
              \end{pmatrix} \xrightarrow[]{R_1\leftrightarrow R_3}
              \begin{pmatrix}
                  1 & 1 & 3 \\
                  2 & 2 & 1 \\
                  2 & 2 & 1
              \end{pmatrix} \xrightarrow[]{R_3\rightarrow R_3-R_1}
              \begin{pmatrix}
                  1 & 1 & 3 \\
                  2 & 2 & 1 \\
                  0 & 0 & 0
              \end{pmatrix} \xrightarrow[]{R_2\rightarrow R_2-2R_1}
              \begin{pmatrix}
                  1 & 1 & 3  \\
                  0 & 0 & -4 \\
                  0 & 0 & 0
              \end{pmatrix}\rightarrow
              \begin{pmatrix}
                  1 & 1 & 0 \\
                  0 & 0 & 1 \\
                  0 & 0 & 0
              \end{pmatrix}
          \]
          נקבל וקטור עצמי $v_0=(1,-1,0)$. ננרמל ונקבל $v_0*=\frac{1}{\sqrt{2}}(1,-1,0)$.
    \item עבור $V_2$ נרצה למצוא את מרחב האפס של המטריצה: \[
              \begin{pmatrix}
                  0  & -2 & -1 \\
                  -2 & 0  & -1 \\
                  -1 & -1 & -1
              \end{pmatrix}\rightarrow
              \begin{pmatrix}
                  -1 & -1 & -1 \\
                  -2 & 0  & -1 \\
                  0  & -2 & -1
              \end{pmatrix}\rightarrow
              \begin{pmatrix}
                  1  & 1  & 1  \\
                  -2 & 0  & -1 \\
                  0  & -2 & -1
              \end{pmatrix} \xrightarrow[]{R_2\rightarrow R_2+2R_1}
              \begin{pmatrix}
                  1 & 1  & 1  \\
                  0 & 2  & 1  \\
                  0 & -2 & -1
              \end{pmatrix} \rightarrow
              \begin{pmatrix}
                  1 & -1 & 0 \\
                  0 & 2  & 1 \\
                  0 & 0  & 0
              \end{pmatrix}
          \]
          נקבל וקטור עצמי $v_2=(1, 1, -2)$ ווקטור מנורמל $v_2*=\frac{1}{\sqrt{6}}(1, 1, -2)$
    \item עבור המרחב העצמי $V_5$ ניעזר בעובדה שסכום כל שורה במטריצה הוא 5, ולכן ניקח $v_5=(1,1,1)$ ו-$v_5*=\frac{1}{\sqrt{3}}(1,1,1)$
\end{itemize}
מקבלים:
\begin{align*}
    Q    & =\begin{pmatrix}
                \frac{1}{\sqrt{2}}   & \frac{1}{\sqrt{6}}  & \frac{1}{\sqrt{3}} \\
                - \frac{1}{\sqrt{2}} & \frac{1}{\sqrt{6}}  & \frac{1}{\sqrt{3}} \\
                0                    & \frac{-2}{\sqrt{6}} & \frac{1}{\sqrt{3}}
            \end{pmatrix} = \frac{1}{\sqrt{6}} \begin{pmatrix}
                                                   \sqrt{3}  & 1  & \sqrt{2} \\
                                                   -\sqrt{3} & 1  & \sqrt{2} \\
                                                   0         & -2 & \sqrt{2}
                                               \end{pmatrix}, \\
    M=PQ & = \frac{1}{\sqrt{6}}\begin{pmatrix}
                                   1 & -1 & 2  \\
                                   0 & 1  & -1 \\
                                   0 & 0  & 1
                               \end{pmatrix}\begin{pmatrix}
                                                \sqrt{3}  & 1  & \sqrt{2} \\
                                                -\sqrt{3} & 1  & \sqrt{2} \\
                                                0         & -2 & \sqrt{2}
                                            \end{pmatrix}=\frac{1}{\sqrt{6}}\begin{pmatrix}
                                                                                2\sqrt{3} & -4 & 2\sqrt{2} \\
                                                                                -\sqrt{3} & 3  & 0         \\
                                                                                0         & -1 & \sqrt{2}
                                                                            \end{pmatrix}
\end{align*}
מקבלים $M\tra B M = Q \tra P \tra B P Q = Q\inv I Q = I$, וכן $M\tra A M=Q\tra S Q = \diag\{0, 2, 5\}$.
הבסיס המבוקש הוא שורות המטריצה $M$, וערכי $\delta_i$ הם $0, 2, 5$.

\pagebreak

\section*{שאלה 5}

\subsection*{סעיף א}

עלינו להוכיח כי מטריצה סימטרי כלשהי $A_{n\times n}$ המייצגת את $q$ אינה הפיכה. \\
על פי 6.2.1, $A$ חופפת למטריצה אלכסונית $B$. על פי חלק ב של אותו המשפט, למטריצה $B$ אותה דרגה ונסמן $\rho=\rho(B)=\rho(A)$. \\
על פי 6.3.2 נקבל $0<\rho<n$. לכן $\rho(A)<n$ ו$A$ סינגולרית!

\subsection*{סעיף ב}

המטריצה $A=[\alpha_{ij}]$ מטריצה סימטרית ממשית ולכן לפי 3.2.1 לכסינה אורתוגונלית על ידי מטריצה אוניטרית $Q$. \\
מהנתון נסיק כי לכל $x=(\xi_1, \xi_2, ..., \xi_n)\in \reals^n$,
\[
    q(x)=\sum_{i=1}^{n}\sum_{j=1}^{n} \alpha_{ij}\xi_i\xi_j>0
\]
אי לכך, על פי הגדרה $A$ חיובית לחלוטין ולכן לפי 3.3.2 כל ערכיה העצמיים של $A$ ממשיים חיוביים. \\
הכיוון הראשון טריוויאלי: אם $A=I$ אז בפרט $A$ אורתוגונלית. \\
אילו $A$ אורתוגונלית אז לכל ערך עצמי $\lambda$ של $A$ מתקיים $|\lambda|=1$ ולכן ל$A$ ערך עצמי יחיד $\lambda=1$.
היות ו$A$ לכסינה, הריבוי הגיאומטרי של ערך עצמי זה הוא $n$ ו$A$ דומה ל$I$. \\
נקבל אפוא:
\[
    A=Q\adj I Q = Q \inv Q = I
\]

\end{document}