\documentclass{article}
\usepackage{fontspec}
\newfontfamily\hebrewfont[Script=Hebrew]{Calibri}
\usepackage{polyglossia}
\usepackage{amsmath, amssymb}
\usepackage[left=2.0cm, top=2.0cm, right=2.0cm, bottom=2.0cm]{geometry}
\usepackage{bidi}
\setdefaultlanguage{hebrew}
\setotherlanguage{english}

\title{מטלת מנחה 15 - אלגברה לינארית 2}
\author{328197462}
\date{02/06/2023}

\def\reals{\mathbb{R}}
\def\complex{\mathbb{C}}
\def\field{\mathbb{F}}
\DeclareMathOperator*{\equals}{=}
\DeclareMathOperator{\trace}{tr}
\DeclareMathOperator{\adj}{^\ast}
\DeclareMathOperator{\tra}{^t}
\DeclareMathOperator{\inv}{^{-1}}
\DeclareMathOperator{\perc}{^\perp}
\DeclareMathOperator{\Sp}{Sp}
\DeclareMathOperator{\Image}{Im}
\DeclareMathOperator{\diag}{diag}

\begin{document}
\maketitle

\section*{שאלה 1}

\subsection*{סעיף א}

נחפש מחרבים $T$-שמורים מכל מימד, כאשר $V=\reals^2$ וכאשר $V=\complex^2$. \\
ממימד 0: נקבל את המרחב הטריוויאלי $\{ 0 \}$ הן עבור $\reals^2$ והן עבור $\complex^2$. \\
ממימד 1: על פי שאלה 8.4.3, כל מרחב $T$-שמור חד מימדי נפרש על ידי וקטור עצמי כלשהו. נמצא ערכים עצמיים של $T$ ומכאן וקטורים עצמיים: \\
\begin{align*}
    P_T(x)=|xI-[T]_E| & =\begin{vmatrix}
                             x-1 & -5  \\
                             10  & x+1
                         \end{vmatrix} = (x-1)(x+1)-10(-5)=x^2+49
\end{align*}
מעל $\reals$, נקבל כי לפולינום אין שורשים, ולהעתקה אין ערכים עצמיים. אין מרחבים $T$-שמורים ממימד 1 עבור $V=\reals^2$. \\
מעל $\complex$, שורשי הפולינום האופייני יהיו $\lambda_1=7i, \lambda_2=-7i$. נמצא וקטורים עצמיים השייכים לע"ע אלה:
\begin{itemize}
    \item עבור $\lambda=7i$ יש למצוא וקטור במרחב האפס של \[
              \begin{pmatrix}
                  7i-1 & -5   \\
                  10   & 7i+1
              \end{pmatrix} \xrightarrow[]{R_1\rightarrow R_1 (-1-7i)}
              \begin{pmatrix}
                  50 & 5+35i \\
                  10 & 7i+1
              \end{pmatrix}\rightarrow
              \begin{pmatrix}
                  10 & 7i+1 \\
                  0  & 0
              \end{pmatrix}
          \]
          הוקטור $v_1=(1+7i, -10)$ מקיים את המשוואה.
    \item עבור $\lambda=-7i$ יש למצוא וקטור במרחב האפס של \[
              \begin{pmatrix}
                  -7i-1 & -5    \\
                  10    & -7i+1
              \end{pmatrix} \xrightarrow[]{R_1\rightarrow R_1 (-1+7i)}
              \begin{pmatrix}
                  50 & 5-35i \\
                  10 & -7i+1
              \end{pmatrix}\rightarrow
              \begin{pmatrix}
                  10 & -7i+1 \\
                  0  & 0
              \end{pmatrix}
          \]
          הוקטור $v_2=(1-7i, -10)$ מקיים את המשוואה.
\end{itemize}
נקבל 2 מרחבים $T$-שמורים $\Sp\{v_1\}, \Sp\{ v_2 \}$. \\
ממימד 2 נקבל את תת-המרחב הטריוויאלי $V$ הן עבור $\complex^2$ והן עבור $\reals^2$.

\subsection*{סעיף ב}

תהא $T$ העתקה כמוגדר.
יהא $U\subseteq V$ תת-מרחב של $V$ ממימד 1. על פי הנתון, $U$ תת-מרחב $T$-שמור. \\
כלומר, קיים $u_0\in U$ כך שלכל $u\in U$ מתקיים $Tu=\lambda u_0$. בפרט, $Tu_0=\alpha u_0$ עבור $\alpha\in \field$ כלשהו.\\\\
נבחר ערך $\alpha$ זה ונוכיח כי $T=\alpha I$. \\
עבור $u\in U$ מקבלים $Tu=T(\lambda u_0)= \lambda Tu_0 = \alpha \cdot \lambda u_0=\alpha u$. \\
נבחר אם כן $v\in V - U$ ונוכיח כי $Tv=\alpha v$. \\
נתבונן בתת-המרחב $W=\Sp\{v\}$. תת-מרחב זה הוא $T$-שמור, לכן $Tv=\beta v$. \\
נתבונן בתת-המרחב $W'=\Sp\{ u_0+v \}$. שוב, מתקיים $T(u_0+v)=\gamma \cdot (u_0+v)=\gamma u_0+\gamma v$.\\
מצד שני, $T(u_0+v)=Tu_0+Tv=\alpha u_0 + \beta v$. \\
הוקטורים $u_0,v$ בלתי-תלויים לינארית (אינם פרופורציוניים), לכן לוקטור $T(u_0+v)$ יש הצגה יחידה כקומבינציה לינארית של $u_0$ ו-$v$. \\
מכאן נסיק $\alpha=\gamma=\beta$ ולכן $Tv=\alpha v$ והשלמנו את מלאכת ההוכחה.

\pagebreak

\section*{שאלה 2}

\subsection*{סעיף א}

נסמן ב $m(x)$ את הפולינום המינימלי של $T_W$, וב$M(x)$ את הפולינום המינימלי של $T$. עלינו להוכיח כי $m$ מחלק את $M$.\\
על פי הגדרה, $T$ מאפסת את $M$. מכאן שלכל $v\in V$, $M(T)v=0$, ובפרט עבור $v\in W$.
כמו כן, לכל $v\in W$ מקבלים $T_Wv=Tv$, ולכן $M(T_W)v=0$. \\
קיבלנו כי $M$ מאפסת את $T_W$. לכן, משאלה 9.9.1א, $m$ מחלק את $M$. \\\\
כעת נניח כי ההעתקה $T$ לכסינה. לפי 10.2.11 בהתאמה למטריצות, נקבל כי $M(x)=(x-\lambda_1)(x-\lambda_2)\cdots(x-\lambda_k)$ כאשר הסקלארים $\lambda_i$ שונים זה מזה. \\
היות ו$m$ מחלק את $M$, $m$ הוא מכפלת חלק או כל הגורמים הלינאריים $x-\lambda_i$ השונים זה מזה ומחלקים את $M$, ולכן לפי 10.2.11 ההעתקה $T_W$ לכסינה.

\subsection*{סעיף ב}

נציין כי $T$ בעלת 3 ערכים עצמיים שונים על מרחב ממימד 3 ולכן לכסינה. הריבוי הגיאומטרי של כל ערך עצמי, לפי לינארית 1, הוא 1. \\
הפולינום המינימלי והאופייני של $V$ לפי 10.2.11 יהיה:
\[
    M(x)=(x-1)(x-2)(x-3)
\]
נסקור את תתי-המרחבים העצמיים לפי מימד: \\
ממימד 0, נקבל את תת-המרחב הטריוויאלי $\{ 0 \}$. \\\\
ממימד 1, נוכיח ראשית את הטענה הבאה: לכל העתקה $S$, כל תת-מרחב $S$-שמור חד ממדי נפרש על ידי וקטור עצמי. \\
כיוון ראשון: נוכיח כי וקטור עצמי פורש תת-מרחב $S$-שמור. יהא וקטור עצמי $v$ השייך לערך העצמי $\lambda$. \\
לכל $u\in \Sp\{ v \}$ נקבל:
\[
    S(u)=S(av)=aS(v)=a\lambda v\in \Sp\{ v \}
\]
הכיוון השני של ההוכחה, וקטור הפורש תת-מרחב $S$-שמור חד ממדי הוא וקטור עצמי, הוכח במהלך סעיף ב של שאלה 1.\\
מהטענה נסיק כי כל המרחבים ה$T$-שמורים ממימד 1 הם $\Sp\{ v_1 \}, \Sp\{ v_2 \}, \Sp\{ v_3 \}$. שלושה תתי-מרחבים אלה הם, כמובן, שונים, שכן וקטורים עצמיים השייכים לערכים עצמיים שונים הם בת"ל (טענה 3.2.6). \\\\
נמצא את תתי-המרחבים ממימד 2 בעזרת סעיף א. נוכיח ראשית כי כל תת-מרחב כזה נפרש על ידי בדיוק 2 וקטורים עצמיים של $T$. יהא $W$ תת-מרחב $T$-שמור דו-ממדי ותהא $T_W$ הצמצום של $T$ על $W$. \\
על פי סעיף א, $T_W$ לכסינה ולכן הפולינום המינימלי שלה מתפרק לגורמים לינאריים שונים ומחלק את $M(x)$. \\
יהא $(x-i)$ גורם לינארי כזה, $i\in \{ 1,2,3 \}$. בפרט, $i$ ערך עצמי של $T_W$, ולכן קיים וקטור עצמי $u_i\in W$. \\
נציין כי $u_i$ הוא ערך עצמי גם של $T$, לכן פרופורציוני ל $v_i$ וגם $v_i\in W$. \\
אילו $(x-i)$ גורם לינארי יחיד, מקבלים כי $i$ ערך עצמי יחיד ל$T_W$ ולכן $W$ נפרשת ע"י שני וקטורים עצמיים בת"ל של ערך עצמי זה, בסתירה לכך שהריבוי הגיאומטרי של כל ערך עצמי הוא 1! \\
נסיק כי קיים גורם לינארי נוסף $(x-j)$, ובאופן דומה $v_j\in W$. מצאנו קבוצה בת"ל בעלת 2 איברים ב $W$ ולכן $W=\Sp\{ v_i, v_j \}$.
כעת, נוכיח כי כל תת-מרחב הנפרש על ידי שני וקטורים עצמיים של $T$ הוא $T$-שמור. \\
יהא $W$ מרחב כזה. אז לכל $w\in W$ מקבלים:
\[
    T(w)=T(\alpha v_i + \beta v_j)=\alpha i v_i + \beta j v_j \in \Sp\{ v_i, v_j \}
\]
לסיכום, תתי-המרחבים ממימד 2 יהיו בדיוק $\Sp\{ v_1, v_2 \}, \Sp\{ v_1, v_3 \}, \Sp\{ v_2, v_3 \}$.\\\\
ממימד 3, נקבל את תת-המרחב הטריוויאלי $V$.

\pagebreak

\section*{שאלה 3}

\subsection*{סעיף א}

נמצא ערכים עצמיים של $T$:
\begin{align*}
    P_T(x)=|xI-[T]_E| & =\begin{vmatrix}
                             x-3 & -1  & 0   \\
                             0   & x-3 & 0   \\
                             0   & 0   & x-2
                         \end{vmatrix}=(x-3)^2(x-2)
\end{align*}
מצאנו שני ערכים עצמיים. נמצא וקטורים עצמיים השייכים לערכים עצמיים אלה:
\begin{itemize}
    \item עבור $\lambda=2$ מדובר בוקטורים ממרחב האפס של \[
              \begin{pmatrix}
                  -1 & -1 & 0 \\
                  0  & -1 & 0 \\
                  0  & 0  & 0
              \end{pmatrix}
          \]
          נקבל מרחב פתרונות $V_{\lambda=2}=\Sp\{ (0,0,1) \}$. זהו, על פי סעיף ב בשאלה 2, מרחב $T$-שמור. \\
    \item עבור $\lambda=3$ מדובר בוקטורים ממרחב האפס של \[
              \begin{pmatrix}
                  0 & -1 & 0 \\
                  0 & 0  & 0 \\
                  0 & 0  & 1
              \end{pmatrix}
          \]
          נקבל מרחב פתרונות $V_{\lambda=3}=\Sp\{ (1,0,0) \}$. זהו מרחב $T$-שמור.
\end{itemize}

\subsection*{סעיף ב}

מצאנו בסעיף א כי $\ker(T-3I)=V_{\lambda=3}=\Sp\{ (1,0,0) \}$.
נניח בשלילה כי קיים מרחב $T$-שמור $U$ כך ש $\reals^3=W\oplus U$. \\
המרחב $U$ ממימד 2. נבחר לו בסיס $(v')=\{ v_2, v_3 \}$ ו $(v)=\{ (1,0,0), v_2, v_3 \}$ בסיס ל$\reals^3$.\\
מהנתון כי $U$ תת-מרחב $T$-שמור מקבלים $Tv_2=\alpha v_2 + \beta v_3$, $Tv_3=\gamma v_2 + \delta v_3$. לכן נקבל:
\[
    [T]_{(v)}=\begin{pmatrix}
        3 & 0      & 0      \\
        0 & \alpha & \gamma \\
        0 & \beta  & \delta
    \end{pmatrix} \ \ \ \
    [T_U]_{(v')}=\begin{pmatrix}
        \alpha & \gamma \\
        \beta  & \delta
    \end{pmatrix}
\]
הפולינום האופייני של $T$ יהיה:
\begin{align*}
    P_T(x)=|xI-[T]_{(v)}| & =\begin{vmatrix}
                                 x-3 & 0        & 0        \\
                                 0   & x-\alpha & -\gamma  \\
                                 0   & -\beta   & x-\delta
                             \end{vmatrix}= \\ &=(x-3)\begin{vmatrix}
        x-\alpha & -\gamma  \\
        -\beta   & x-\delta
    \end{vmatrix}=(x-3)|xI-[T_U]_{(v')}| = (x-3)P_{T_U}
\end{align*}
מצד שני, $P_T(x)=(x-3)^2(x-2)$ ולכן $P_{T_U}(x)=(x-3)(x-2)$. \\
נקבל כי $3$ ערך עצמי של $T_U$, ויש לו וקטור עצמי $u\in \Sp\{ v_2, v_3 \}$. ברור כי $\{ u, (1,0,0) \}$ בת"ל (אחרת הקבוצה $(v)$ ת"ל) ולכן מצאנו 2 וקטורים עצמיים בת"ל השייכים לערך $\lambda=3$ עבור $T$, בסתירה למציאתנו בסעיף א, בה הריבוי הגיאומטרי של $\lambda=3$ הוא 1!

\pagebreak

\section*{שאלה 4}

עלינו למצוא פירוק של הפולינום המינימלי של $T|_W$, שנסמנו $M_W$, ל $k$ פולינומים זרים בזוגות $P_1, P_2, ..., P_k$ כך שלכל $i$, $\ker P_i(T|_W)=W\cap W_i$. \\
כמו כן, על פי שאלה 2 במטלה זו, הפולינום המינימלי של $T|_W$ מחלק את $M(t)$. היות והפולינומים $M_1, M_2, ..., M_k$ זרים בזוגות, המשמעות היא שכל גורם בכל פירוק של $M_W$ יחלק אחד בדיוק מבין סדרת פולינומים אלו (אחרת, יהיה להם מחלק משותף שאינו 1). \\
נסמן אפוא ב $M_W=p_1\cdot p_2\cdots p_m$ את הפירוק המקסימלי של $M_w$, ונבחר את $P_i$ להיות מכפלת כל הפולינומיים האי-פריקים $p_j$ המחלקים את $M_i$.
ברור כי כל פולינום אי-פריק $p_j$ יהיה גורם במכפלה אחת בדיוק, ולכן $M_W=P_1\cdot P_2\cdot ... \cdot P_k$. \\\\
נסמן $U_i=\ker P_i(T_W)$. לפי הפירוק הפרימרי, $W=U_1\oplus U_2\oplus\cdots\oplus U_k$. \\
נוכיח כי $U_i\subseteq W\cap W_i$. יהא $w\in U_i$. ברור כי $w\in W$, שכן $U_i\subseteq W$. עלינו להראות כי $w\in W_i=\ker M_i(T)$. \\
נניח בשלילה כי $M_i(T)w=M_i(T_W)w\ne 0$. היות ו$P_i$ מחלק את $M_i$, נקבל גם $P_i(T_W)w\ne 0$ ולכן $w\notin U_i$ וזו סתירה! \\\\
נקבל מצד אחד כי $W=U_1\oplus U_2\oplus\cdots\oplus U_k\subseteq (W\cap W_1)+(W\cap W_2)+\cdots + (W\cap W_k)$. \\
מצד שני, $(W\cap W_1)+(W\cap W_2)+\cdots+(W\cap W_k)\subseteq W$ ולכן מתקיים שוויון. \\
נוסיף כי הקבוצות $(W\cap W_1),(W\cap W_2), ...$ זרות: אם יש איבר משותף בין שתי קבוצות כלשהן בסכום, אז בפרט קיימים $i,j$ כך ש $W_i\cap W_j\ne \{0\}$, בסתירה לסכום הישר בפירוק הפרימרי של $V$!\
אי-לכך המרחב $W$ הוא סכום ישר של המרחבים $(W\cap W_i)$.

\section*{שאלה 5}

תהא $T$ העתקה נורמלית במרחב אוניטרי ויהא $W$ תת-מרחב $T$-שמור. \\
לפי משפט הלכסון האוניטרי, ההעתקה $T$ לכסינה, ולפי שאלה 2 בממן זה גם הצמצום שלה $T_W$ מהווה העתקה לכסינה. \\
אי-לכך, קיים בסיס $(w)=(w_1, w_2, ..., w_k)$ של $W$ המורכב מוקטורים עצמיים. \\
וקטורים אלה, על פי למה 3.2.5, הם גם וקטורים עצמיים של $T_W\adj$, השייכים לערכים עצמיים $\lambda_1, \lambda_2, ..., \lambda_k$.\\\\
לכל $w\in W$ מקבלים:
\begin{align*}
    T\adj(w) & =
    T\adj ( \sum_{i=1}^k a_i w_i ) =
    \sum_{i=1}^k a_i T\adj (w_i) =
    \sum_{i=1}^k a_i \lambda_i w_i \in W
\end{align*}
קיבלנו כי $W$ תת-מרחב $T\adj$-שמור.


\end{document}