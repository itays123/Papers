% !TEX program = xelatex
\documentclass{article}
\usepackage[]{amsthm} %lets us use \begin{proof}
\usepackage{amsmath}
\usepackage{enumerate}
\usepackage{xparse}
\usepackage[makeroom]{cancel}
\usepackage[]{amssymb} %gives us the character \varnothing
\usepackage{fontspec}
\usepackage{enumerate}
\usepackage{polyglossia}
\usepackage{relsize}
\usepackage[left=2.0cm, top=2.0cm, right=2.0cm, bottom=2.0cm]{geometry}

\setdefaultlanguage{hebrew}
\setotherlanguage{english}

\setmainfont{[Arial.ttf]}
\newfontfamily\hebrewfont{[Arial.ttf]}

\newcommand\underrel[2]{\mathrel{\mathop{#2}\limits_{#1}}}
\DeclareMathOperator*{\equals}{=}

\title{מטלת מנחה 15 - אינפי 2}
\author{328197462}
\date{06/01/2023}

\begin{document}
\long\def\/*#1*/{}
\maketitle

\section*{שאלה 1}

\subsection*{סעיף א}

נפריד לשני טורים ונבצע פעולות חיבור בעזרת משפט $5.9$. נסמן אפוא:
\[
    a_n = \frac{1}{\sqrt{n}\cdot 2^n}>0 \ \ \
    b_n = \frac{(-2)^n\cdot \sin \frac{1}{n}}{\sqrt{n}\cdot 2^n}
    = \frac{(-1)^n \cdot \sin \frac{1}{n}}{\sqrt{n}}
\]
\\
ראשית, עבור הסדרה החיובית $a_n$ נקבל
\[
    0\leq a_n = \frac{1}{\sqrt{n}\cdot 2^n} \underrel{\sqrt{n}\geq 1}{\leq} \frac{1}{2^n}
\]
כאשר הטור $\sum \frac{1}{2^n}$ הוא טור הנדסי שמנתו $q=\frac{1}{2}$, ולכן מתכנס. \\
אי לכך, לפי מבחן ההשוואה $5.14$, גם הטור $\sum a_n$ מתכנס, וכן מאחר ומדובר בסדרה חיובית הטור מתכנס בהחלט.
\\\\
כעת נבחן את התכנסות הטור $\Sigma|b_n|$. מתקיים:
\[
    0\leq|b_n|=\left|\frac{(-1)^n \cdot \sin \frac{1}{n}}{\sqrt{n}}\right| =
    \frac{|(-1)^n| |\sin \frac{1}{n}|}{\sqrt{n}}=
    \frac{|\sin \frac{1}{n}|}{\sqrt{n}} \underrel{|\sin x|\leq |x| \text{מאינפי 1 }}{\leq}
    \frac{1}{n\sqrt{n}}
\]
הטור $\Sigma \frac{1}{n\sqrt{n}}=\Sigma \frac{1}{n^{1.5}}$
מתכנס (לפי דוגמה 5.8א ביחידה 5 כאשר $\alpha=1.5>1$), \\
ולכן לפי מבחן ההשוואה $5.14$ נסיק כי גם הטור $\Sigma |b_n|$ מתכנס.
\\\\
כעת נתבונן בטור הנתון בסעיף. מדובר בטור
\[
    \sum_{n=1}^\infty \frac{1+(-2)^n\sin\frac{1}{n}}{\sqrt{n}\cdot 2^n}=
    \sum_{n=1}^\infty (a_n+b_n)
\]
נשים לב כי לפי אי-שוויון המשולש,
\[
    |a_n+b_n|\leq |a_n|+|b_n| \underrel{a_n>0}{=} a_n + |b_n|
\]
הטור $\Sigma(a_n +|b_n|)$ מתכנס כסכום של טורים מתכנסים לפי משפט .5.9 \\
אי לכך, גם הטור $\Sigma|a_n+b_n|$ מתכנס, ומכאן נסיק התכנסות בהחלט של הטור הנתון בשאלה.

\pagebreak
\subsection*{סעיף ב}
נפריד שוב לשני טורים. נסמן אפוא:
\[
    a_n=\frac{n\cdot \cos 2n}{n^2-1} = \frac{n}{n^2-1} \cdot \cos 2n \ \ \
    b_n = \frac{\cos \pi n}{\ln n \cdot \ln(n^n+n)}=\frac{(-1)^n}{\ln n \cdot \ln(n^n+n)}
\]

\end{document}