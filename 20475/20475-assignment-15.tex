% !TEX program = xelatex
\documentclass{article}
\usepackage[]{amsthm} %lets us use \begin{proof}
\usepackage{amsmath}
\usepackage{enumerate}
\usepackage{xparse}
\usepackage[makeroom]{cancel}
\usepackage[]{amssymb} %gives us the character \varnothing
\usepackage{fontspec}
\usepackage{enumerate}
\usepackage{polyglossia}
\usepackage{relsize}
\usepackage[left=2.0cm, top=2.0cm, right=2.0cm, bottom=2.0cm]{geometry}

\setdefaultlanguage{hebrew}
\setotherlanguage{english}

\setmainfont{[Arial.ttf]}
\newfontfamily\hebrewfont{[Arial.ttf]}

\newcommand\underrel[2]{\mathrel{\mathop{#2}\limits_{#1}}}
\DeclareMathOperator*{\equals}{=}

\title{מטלת מנחה 15 - אינפי 2}
\author{328197462}
\date{06/01/2023}

\begin{document}
\long\def\/*#1*/{}
\maketitle

\section*{שאלה 1}

\subsection*{סעיף א}

נפריד לשני טורים ונבצע פעולות חיבור בעזרת משפט $5.9$. נסמן אפוא:
\[
    a_n = \frac{1}{\sqrt{n}\cdot 2^n}>0 \ \ \
    b_n = \frac{(-2)^n\cdot \sin \frac{1}{n}}{\sqrt{n}\cdot 2^n}
    = \frac{(-1)^n \cdot \sin \frac{1}{n}}{\sqrt{n}}
\]
\\
ראשית, עבור הסדרה החיובית $a_n$ נקבל
\[
    0\leq a_n = \frac{1}{\sqrt{n}\cdot 2^n} \underrel{\sqrt{n}\geq 1}{\leq} \frac{1}{2^n}
\]
כאשר הטור $\sum \frac{1}{2^n}$ הוא טור הנדסי שמנתו $q=\frac{1}{2}$, ולכן מתכנס. \\
אי לכך, לפי מבחן ההשוואה $5.14$, גם הטור $\sum a_n$ מתכנס, וכן מאחר ומדובר בסדרה חיובית הטור מתכנס בהחלט.
\\\\
כעת נבחן את התכנסות הטור $\Sigma|b_n|$. מתקיים:
\[
    0\leq|b_n|=\left|\frac{(-1)^n \cdot \sin \frac{1}{n}}{\sqrt{n}}\right| =
    \frac{|(-1)^n| |\sin \frac{1}{n}|}{\sqrt{n}}=
    \frac{|\sin \frac{1}{n}|}{\sqrt{n}} \underrel{|\sin x|\leq |x| \text{מאינפי 1 }}{\leq}
    \frac{1}{n\sqrt{n}}
\]
הטור $\Sigma \frac{1}{n\sqrt{n}}=\Sigma \frac{1}{n^{1.5}}$
מתכנס (לפי דוגמה 5.8א ביחידה 5 כאשר $\alpha=1.5>1$), \\
ולכן לפי מבחן ההשוואה $5.14$ נסיק כי גם הטור $\Sigma |b_n|$ מתכנס.
\\\\
כעת נתבונן בטור הנתון בסעיף. מדובר בטור
\[
    \sum_{n=1}^\infty \frac{1+(-2)^n\sin\frac{1}{n}}{\sqrt{n}\cdot 2^n}=
    \sum_{n=1}^\infty (a_n+b_n)
\]
נשים לב כי לפי אי-שוויון המשולש,
\[
    |a_n+b_n|\leq |a_n|+|b_n| \underrel{a_n>0}{=} a_n + |b_n|
\]
הטור $\Sigma(a_n +|b_n|)$ מתכנס כסכום של טורים מתכנסים לפי משפט .5.9 \\
אי לכך, גם הטור $\Sigma|a_n+b_n|$ מתכנס, ומכאן נסיק התכנסות בהחלט של הטור הנתון בשאלה.

\pagebreak
\subsection*{סעיף ב}
נפריד שוב לשני טורים. נסמן אפוא:
\[
    a_n=\frac{n\cdot \cos 2n}{n^2-1} = \frac{n}{n^2-1} \cdot \cos 2n \ \ \
    b_n = \frac{\cos \pi n}{\ln n \cdot \ln(n^n+n)}=\frac{(-1)^n}{\ln n \cdot \ln(n^n+n)}
\]
\\\\
נתבונן תחילה בטור $\Sigma a_n$. נרצה להוכיח כי הטור מתכנס לפי מבחן דיריכלה:
\begin{enumerate}[i]
    \item נראה כי הסדרה $\mu_n=\frac{n}{n^2-1}$ מונוטונית יורדת. \\
          נגדיר $\mu(x)=\frac{x}{x^2-1}$. מתקיים:
          \[
              \mu'(x)=\frac{x^2-1-x\cdot 2x}{(x^2-1)^2}=
              -\frac{x^2+1}{(x^2-1)^2} < 0
          \]
          ולכן $\mu$ פונקצייה יורדת, ובפרט לכל $n$ טבעי נקבל $\mu_{n+1}=\mu(n+1)<\mu(n)=\mu_n$.
    \item כמו כן, $\mu_n=\frac{n}{n^2-1}\xrightarrow[n\rightarrow\infty]{}0$
    \item בנוסף, על פי שאלה 33א ביחידה 5, $\Sigma \cos 2n$ סכום חסום.
\end{enumerate}
נקבל, על פי משפט $5.22$, כי הטור $\Sigma_{n=2}^\infty (\mu_n \cdot \cos 2n)=\Sigma_{n=2}^\infty a_n$ מתכנס.
\\\\
נבחן כעת את התכנסות הטור $|b_n|$. נשים לב כי $\ln(n^n+n)\geq \ln(n^n)=n\ln n$, ואי לכך:
\[
    |b_n|=\left|\frac{(-1)^n}{\ln n \cdot \ln(n^n+n)}\right| \underrel{x\geq 1 \Rightarrow \ln x > 0}{=}
    \frac{|(-1)^n|}{\ln n \cdot \ln(n^n+n)} =
    \frac{1}{\ln n \cdot \ln(n^n+n)} \underrel{\text{הסבר מקודם}}{\leq}
    \frac{1}{\ln n \cdot n \ln n} = \frac{1}{n\ln^2n}
\]
הטור $\Sigma_{n=2}^\infty \frac{1}{n\ln^2n}$
מתכנס לפי שאלה 27א ביחידה 5 עבור $\alpha = 2$,
ולכן לפי מבחן ההשוואה $5.14$ נסיק כי $\Sigma_{n=2}^\infty |b_n|$ מתכנס, \\
כלומר $\Sigma_{n=2}^\infty b_n$ מתכנס בהחלט ובפרט מתכנס.
\\\\
אי לכך, לפי משפט 5.9, הטור
\[
    \sum_{n=2}^\infty (\frac{n\cdot \cos 2n}{n^2-1}+\frac{\cos \pi n}{\ln n \cdot \ln(n^n+n)})=
    \sum_{n=2}^\infty (a_n + b_n)
\]
מתכנס. כעת, נניח בשלילה כי הטור מתכנס בהחלט.
אז לפי משפט $5.9$,
הטור $\Sigma_{n=2}^\infty(|a_n+b_n|+|b_n|)$ מתכנס.
מתקיים:
\[
    0\leq |a_n|=|(a_n+b_n)-b_n|\underrel{\text{א"ש המשולש}}{\leq}
    |a_n+b_n|+|b_n|
\]
ועל כן, לפי מבחן ההשוואה $5.14$
נסיק כי $\Sigma_{n=2}^\infty |a_n|$ מתכנס.
ונקבל אפוא
\[
    |a_n|=\left|\frac{n\cdot \cos 2n}{n^2-1}\right| =
    \frac{n\cdot |\cos 2n|}{n^2-1}\underrel{|\cos x| \geq \cos^2x}{\geq}
    \frac{n\cdot \cos^2 2n}{n^2-1}\geq 0
\]
ומכאן, שוב לפי מבחן ההשוואה, $\Sigma_{n=2}^\infty \frac{n\cdot \cos^2 2n}{n^2-1}$ \\
הוא טור מתכנס גם הוא. לפי הזהות $\cos^2 x = \frac{1}{2} + \frac{1}{2}\cos 2x$, נסיק:
\[
    \frac{n\cdot \cos^2 2n}{n^2-1} =
    \frac{n}{2(n^2-1)} + \frac{n\cdot \cos 4n}{2(n^2-1)}
\]
הטור $\Sigma_{n=2}^\infty$ מתכנס, לפי מבחן דיריכלה, ובהוכחה שקולה להוכחה מקודם. \\
אי לכך, לפי שאלה 11א ביחידה 5, נסיק כי
גם הטור $\Sigma_{n=2}^\infty \frac{n}{2(n^2-1)}$ מתכנס.
עם זאת,
\[
    \frac{\frac{n}{2(n^2-1)}}{\frac{1}{n}} =
    \frac{n^2}{2(n^2-1)}\xrightarrow[n\rightarrow \infty ]{} \frac{1}{2}
\]
ולכן לפי מבחן ההשוואה $5.15$ גם הטור $\Sigma_{n=2}^\infty \frac{1}{n}$ מתכנס. \\
נסיק כי לפי משפט 5.12 גם הטור $\Sigma_{n=1}^\infty \frac{1}{n}$ מתכנס, בסתירה לדוגמה 5.8א בספר!
\\
מהסתירה נובע כי הטור $\Sigma_{n=2}^\infty |a_n+b_n|$ לא מתכנס, ולכן הטור שבשאלה מתכנס בתנאי.

\pagebreak

\section*{שאלה 2}

תהא $(u_n)$ המתכנסת לגבול שלילי, וכן יהא $a$ מספר חיובי. \\
נסמן $a_n=a^{u_1+u_2+\cdots +u_n}$ ונראה כי $\Sigma a_n$ מתכנס אם ורק אם $a>1$.
\\\\
נחשב את הגבול:
\[
    c=\lim_{n\rightarrow \infty} \left| \frac{a_{n+1}}{a_n} \right| =
    \lim_{n\rightarrow \infty} \left| \frac{a^{u_1+u_2+\cdots +u_n+u_{n+1}}}{a^{u_1+u_2+\cdots +u_n}} \right| =
    \lim_{n\rightarrow \infty} |a^{u_{n+1}}|\underrel{a>0\Rightarrow a^x>0}{=}
\]
\[
    \lim_{n\rightarrow \infty} a^{u_{n+1}}\underrel{\text{לפי אינפי 1}}{=}
    (\lim_{n\rightarrow \infty} a)^{\lim_{n\rightarrow \infty}{u_{n+1}}}\underrel{\text{הזזה}}{=}
    a^{{\lim_{n\rightarrow \infty}{u_{n}}}}\underrel{\text{נתון}}{=}
    a^u
\]
אילו $a > 1$,
אז מהנתון $u<0$ נסיק $c=a^u<1$ ולכן לפי משפט $5.17**$ הטור $\Sigma a_n$ מתכנס בהחלט, ובפרט מתכנס.\\
אילו $a<1$,
אז מהנתון $u<0$ נסיק $c=a^u>1$ ולכן לפי משפט $5.17**$ הטור $\Sigma a_n$ מתבדר.
\\\\
אילו $a=1$, אז נקבל לכל $n$ טבעי $a_n=1^{\cdots}=1=\frac{1}{n^0}$\\
ולכן הטור $\Sigma a_n$ מתבדר, לפי דוגמה 5.8א ביחידה 5, כי $\alpha = 0 < 1$
בכך הוכחנו כי הטור $\Sigma a_n$ מתכנס אם ורק אם $a > 1$.

\section*{שאלה 3}

נסמן $a=\lim_{n\rightarrow\infty}a_n\ne 0$.
אז מהגדרת הגבול עבור $\epsilon=\frac{|a|}{2}$,
קיים $n_0\in \mathbb{N}$ כך שלכל $n>n_0$ מתקיים $|a_n-a|<\epsilon$, \\
כלומר, מאי-שוויון המשולש $-\frac{|a|}{2}<|a_n|-|a|<\frac{|a|}{2}\Rightarrow \frac{|a|}{2}<|a_n|<\frac{3|a|}{2}$. \\
אי לכך, לכל $n+1>n_0$ נקבל:
\[
    \frac{a^2}{4}<|a_{n+1}a_n|<\frac{9a^2}{4}
\]
כעת, נשים לב שהחל מ$n_0$ מתקיים:
\[
    \frac{a^2}{4}|a_n-a_{n+1}|<
    |\frac{1}{a_{n+1}}-\frac{1}{a_n}|=
    \frac{|a_n-a_{n+1}|}{|a_na_{n+1}|}<
    \frac{9a^2}{4}|a_n-a_{n+1}|
\]
\\
כעת נניח כי $\Sigma (a_{n+1}-a_{n})$ מתכנס בהחלט, כלומר
$\Sigma|a_{n+1}-a_n|$ מתכנס,\\
ונוכיח לפי מבחן ההשוואה כי הטור החיובי $\Sigma |\frac{1}{a_{n+1}}-\frac{1}{a_n}|$ מתכנס.\\
לפי משפט 5.10 (נדגיש $\frac{9a^2}{4}\ne 0$) נסיק כי $\Sigma \frac{9a^2}{4}|a_{n+1}-a_n|$ מתכנס.
הראינו כי כמעט לכל $n$ מתקיים $|\frac{1}{a_{n+1}}-\frac{1}{a_n}|<\frac{9a^2}{4}|a_{n+1}-a_n|$,
ואי לכך לפי מבחן ההשוואה 5.14 נסיק כי הטור מתכנס, כלומר $\Sigma(\frac{1}{a_{n+1}}-\frac{1}{a_n})$ מתכנס בהחלט.
\\\\
בכיוון השני, אם נניח כי $\Sigma |\frac{1}{a_{n+1}}-\frac{1}{a_n}|$ מתכנס,
אז לפי מבחן ההשוואה 5.14 ומשפט 5.10 נסיק כי הטור החיובי $\Sigma |a_{n+1}-a_n|$ מתכנס.

\pagebreak

\section*{שאלה 4}

\subsection*{סעיף א}

הטענה לא נכונה. \\
יהא טור מתכנס $\Sigma a_n$,
מתקיים לפי משפט 5.5 התנאי ההכרחי $a_n\xrightarrow[n\rightarrow\infty]{}0$. אי לכך,
\[
    \lim_{n\rightarrow\infty}\cos(a_n)=
    \begin{bmatrix}
        \text{לפי היינה} \\
        x=a_n\rightarrow 0
    \end{bmatrix} =
    \lim_{x\rightarrow 0} \cos x \underrel{\text{רציפות}}{=}
    \cos(0)=1
\]
לא מתקיים התנאי ההכרחי להתכנסות טור ולכן הטור $\Sigma \cos(a_n)$ לא מתכנס.

\subsection*{סעיף ב}

הטענה נכונה. נוכיח לפי מבחן ההשוואה 5.15
\\
נתון כי איברי הסדרה $a_n$ חיוביים. כמו כן, $a_n^2>0\Rightarrow a_n^2+a_n>0$
כמו כן מתקיים:
\[
    \lim_{n\rightarrow \infty} \frac{a_n^2+a_n}{a_n}=
    \lim_{n\rightarrow \infty} \left(\frac{a_n^2}{a_n}+\frac{a_n}{a_n}\right)=
    \lim_{n\rightarrow \infty} (a_n+1) \underrel{\text{התנאי ההכרחי 5.5}}{=}
    0+1=1>0
\]
אי לכך, לפי מבחן ההשוואה 5.5 הטורים $\Sigma a_n, \Sigma (a_n^2+a_n)$ מתכנסים ומתבדרים ביחד.

\subsection*{סעיף ג}

הטענה נכונה. נניח כי הטור $\Sigma|a_{n+1}-a_n|$ מתכנס.  \\
אז מכאן נסיק שהטור $\Sigma(a_{n+1}-a_n)$ מתכנס,
ולכן סדרת הסכומים החלקיים של הטור,
\[
    S_k=(a_2-a_1)+(a_3-a_2)+\cdots+(a_{k+1}-a_k)=a_{k+1}-a_1 \xrightarrow[k\rightarrow\infty]{} S
\]
מתכנסת.
אי לכך,
\[
    \lim_{k\rightarrow \infty} a_k \underrel{\text{הזזה}}{=}
    \lim_{k\rightarrow \infty} a_{k+1} =
    \lim_{k\rightarrow \infty} (a_{k+1}-a_1+a_1) =
    \lim_{k\rightarrow \infty} (a_{k+1}-a_1) + \lim_{k\rightarrow \infty}a_1 =
    S+a_1
\]
הוכחנו כי לסדרה $(a_n)$ יש גבול סופי ולכן היא מתכנסת.

\pagebreak

\section*{שאלה 5}



\end{document}