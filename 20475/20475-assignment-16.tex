% !TEX program = xelatex
\documentclass{article}
\usepackage[]{amsthm} %lets us use \begin{proof}
\usepackage{amsmath}
\usepackage{enumerate}
\usepackage{xparse}
\usepackage[makeroom]{cancel}
\usepackage[]{amssymb} %gives us the character \varnothing
\usepackage{fontspec}
\usepackage{enumerate}
\usepackage{polyglossia}
\usepackage{relsize}
\usepackage[left=2.0cm, top=2.0cm, right=2.0cm, bottom=2.0cm]{geometry}

\setdefaultlanguage{hebrew}
\setotherlanguage{english}

\setmainfont{[Arial.ttf]}
\newfontfamily\hebrewfont{[Arial.ttf]}

\newcommand\underrel[2]{\mathrel{\mathop{#2}\limits_{#1}}}
\DeclareMathOperator*{\equals}{=}
\def\reals{\mathbb{R}}
\def\naturals{\mathbb{N}}

\title{מטלת מנחה 15 - אינפי 2}
\author{328197462}
\date{20/01/2023}

\begin{document}
\long\def\/*#1*/{}
\maketitle

\section*{שאלה 1}

נתונה סדרת הפונקציות $f_n(x)=\frac{nx}{e^x+n+x}$ המוגדרות (ורציפות) ב$[0, \infty)$. \\
נחשב את הפונקציה הגבולית. לכל $x\in[0, \infty)$ מתקיים:

\begin{align*}
    f(x)=
    \lim_{n\rightarrow \infty} f_n(x)=
    \lim_{n\rightarrow \infty} \frac{nx}{e^x+n+x}=
    \lim_{n\rightarrow \infty} \frac{x}{e^x \cdot \frac{1}{n}+1+x\cdot \frac{1}{n}}=
    \frac{x}{1}=x
\end{align*}

כמו כן, מתקיים לכל $n\in \mathbb{N}, x\in[0, \infty)$:
\begin{align*}
    |f_n(x)-f(x)|=
    |\frac{nx}{e^x+n+x}-x|=
    |\frac{nx-x(e^x+n+x)}{e^x+n+x}|=
    |\frac{-x^2-xe^x}{e^x+n+x}|=
    \frac{x^2+xe^x}{e^x+n+x}
\end{align*}

\subsection*{סעיף א}

ניקח את הסדרה $x_n=n$. מתקיים:
\begin{align*}
    \sup_{x\in[0, \infty)} |f_n(x)-f(x)| \geq
    |f_n(x_n)-f(x_n)|=
    \frac{n^2+ne^n}{e^n+2n}=
    \frac{\frac{n^2}{e^n}+n}{1+2\cdot \frac{n}{e^n}}\xrightarrow[n\rightarrow \infty]{}
    \infty
\end{align*}

מכאן, לפי למה $6.3$, נסיק כי $(f_n)$ לא מתכנסת במידה שווה ל$f$.

\subsection*{סעיף ב}

יהיו $0\leq a < b$ כלשהם. \\
נדגיש כי מתקיים $[a,b]\subseteq [0,\infty)$ והפונקציה הגבולית $f$ נשארת זהה. \\
נבחר את הסדרה $\mu_n=\frac{b^2+be^b}{e^a+n+a}$. לכל $x\in [a,b], n\in\mathbb{N}$ נקבל:
\begin{align*}
    |f_n(x)-f(x)| =
    \frac{x^2+xe^x}{e^x+n+x} \underrel{a\leq x \leq b}{\leq}
    \frac{b^2+be^b}{e^a+n+a}=\mu_n
\end{align*}
מתקיים $\mu_n\xrightarrow[n\rightarrow\infty]{}0$ ($a,b$ קבועים) ולכן לפי שאלה 7 ביחידה 6 נסיק כי $(f_n)$ מתכנסת במ"ש ל$f$. \\
לכן, לפי משפט $6.8$ נקבל
\begin{align*}
    \lim_{n\rightarrow \infty} \int_a^bf_n(x)dx= \int_a^b f(x)dx
\end{align*}
ובכך סיימנו את ההוכחה.

\pagebreak

\section*{שאלה 2}

מגדירים פונקציה רציפה $f: [0,1]\rightarrow \reals$, וכן לכל $n$ טבעי מגדירים $f_n: [0,1] \rightarrow\reals$ כך שלכל $x\in [0,1]$ מתקיים $f_n(x)=f(x^n)$. \\
נגדיר $g(x)\equiv f(0)$ פונקציה קבועה.

\subsection*{סעיף א}

יהא $0<a<1$ כלשהו. \\
נראה תחילה התכנסות נקודתית לפונקציה $g$. לכל $x\in [0,a]$:
\begin{align*}
    \lim_{n\rightarrow \infty}f_n (x) =
    \lim_{n\rightarrow \infty} f(x^n) \equals_{t=x^n\rightarrow 0^+}
    \lim_{t\rightarrow 0^+}f(t) \equals_{\text{רציפות $f$}}
    f(0)
\end{align*}
\\\\
נוכיח התכנסות במידה שווה לפי הגדרה. יהא $\epsilon_0>0$. \\
הפונקציה $\delta(x)=|f(x)-f(0)|$ היא פונקציה רציפה ב$[0,1]$ כהפרש והרכבה של פונקציות רציפות. \\
לכן, לפי אינפי 1, יש לה ערך מקסימלי בקטע $[0,1]$. נסמן ערך זה ב$x_\Delta\in [0,1]$. \\
עבור $\epsilon_0 > \delta(x_\Delta)$, לכל $n$ טבעי ולכל $x\in [0,a]$ מתקיים $x^n\in [0,1]$ ולכן $|f_n(x)-f(0)|<|f(x_\Delta)-f(0)|<\epsilon_0$. \\
בפרט, כאשר $x_\Delta=0$ נקבל $\delta(x_\Delta)=0$ ולכן $\epsilon_0>\delta(x_\Delta)$.\\
נוכיח עבור שארית המקרים - $\epsilon_0<\delta(x_\Delta)$ וכן $x_\Delta\in(0,1]$.
    \\\\
    מרציפות פונקציית המרחק ואי-השוויון $\delta(0)=0<\epsilon_0<\delta(x_\Delta)$, קיים לפי משפט ערך הביניים מאיפני 1 $x_\epsilon\in [0,1]$ כלשהו כך ש $\delta(x_\epsilon)=\epsilon_0$.
    לא ייתכן $x_\epsilon=0$ כי $\delta(x_\epsilon)>\delta(0)=0$. \\
    נרצה לבחור את הנקודה $x_\epsilon$ השמאלית ביותר האפשרית,
    כלומר, אילו קיים $x'\in (0,x_\epsilon)$ כך ש $\delta(x')>\epsilon_0$, משפט ערך הביניים מבטיח לנו קיומו של $x_\epsilon\in(0,x')$ נוסף כך ש $\delta(x_\epsilon)=\epsilon_0$.
    בחירה זו מבטיחה לנו כי לכל $x\in [0, x_\epsilon)$ מתקיים $\delta(x)<\epsilon_0$
\\\\
לכן, מהגבול הידוע מאינפי 1 $\sqrt[n]{x_\epsilon}\xrightarrow[n\rightarrow \infty]{}1$ נסיק כי עבור $\epsilon=1-a>0$, החל מ $N\in\naturals$ כלשהו נקבל:
\begin{align*}
    |\sqrt[n]{x_\epsilon}-1|<1-a \underrel{\sqrt[n]{x_\epsilon}\leq 1}{\Rightarrow} 1 - \sqrt[n]{x_\epsilon} < 1 - a \Rightarrow \sqrt[n]{x_\epsilon} > a
\end{align*}
נבחר $N$ זה. לכל $n>N$ נקבל בקטע $[0,a]$ כי הפונקציה $x^n$ מונוטונית עולה,\\
ולכן לכל $0\leq x \leq a < \sqrt[n]{x_\epsilon}$ מקבלים $0\leq x^n \leq a^n < x_\epsilon\Rightarrow x\in [0, x_\epsilon)$.
            אי לכך,
            \begin{align*}
                |f_n(x)-f(0)| =|f(x^n)-f(0)|
                =\delta(x^n) \underrel{x^n\in[0,x_\epsilon)}{<} \epsilon_0
            \end{align*}

            \subsection*{סעיף ב}

            הפונקציה $f$ רציפה ולכן אינטגרבילית, וכמו כן הפונקציות $f_n$ רציפות ולכן אינטגרביליות כהרכבה של פונקציות רציפות. \\
            נוכיח לפי הגדרת הגבול, בהשראת הוכחת משפט $6.8$. אם כן, יהא $\epsilon>0$ ועלינו להוכיח שהחל מ$N\in\naturals$ מסוים מתקיים:
            \begin{align*}
                \left|
                \int_0^1 f_n(x)dx- f(0)
                \right| < \epsilon
            \end{align*}
            נשים לב כי מאחר ו$f(0)$ פונקציה קבועה ורציפה ולכן אינטגרבילית, נקבל לפי שאלה 51 ביחידה 1:
            \begin{align*}
                f(0)=(1-0)\cdot f(0) \leq \int_0^1 f(0) \leq (1-0) \cdot f(0)=f(0)
            \end{align*}
            ואכן נקבל לכל $N$ טבעי:
            \begin{align*}
                \left|
                \int_0^1 f_n(x)dx- \int_0^1 f(0)dx
                \right| \equals_{1.24}
                \left|
                \int_0^1 (f_n(x)-f(x))dx
                \right| \underrel{\text{שאלה 1.50}}{\leq}
                \int_0^1 |f_n(x)-f(x)|dx\underrel{3.3}{=}
                \lim_{a\rightarrow 1^-} \int_0^a |f_n(x)-f(x)|dx
            \end{align*}

            לכל $0<a<1$, לפי סעיף א, מתקיים $|f_n(x)-f(x)|<\frac{\epsilon}{2}$ החל מ$N$ מסוים.
            לכן, לפי $1.26$ ואינפי 1:
            \begin{align*}
                \lim_{a\rightarrow 1^-} \int_0^a |f_n(x)-f(x)|dx \leq
                \lim_{a\rightarrow 1^-} \int_0^a \frac{\epsilon}{2} =
                \lim_{a\rightarrow 1^-} a \cdot \frac{\epsilon}{2} = \frac{\epsilon}{2} < \epsilon
            \end{align*}

            ובזאת סיימנו את ההוכחה.

            \pagebreak

            \section*{שאלה 3}

            \subsection*{סעיף א}

            לפנינו טור חזקות מהצורה $\sum_{n=1}^\infty a_nx^n$, כאשר לכל $n$ טבעי נקבל $a_n=\frac{n!}{(2n)!}>0$ \\
            נמצא רדיוס התכנסות לפי למה $6.11$:
            \begin{align*}
                R=\lim_{n\rightarrow\infty} \frac{a_{n+1}}{a_n} =
                \lim_{n\rightarrow\infty} \frac{n!}{(2n)!} \cdot \frac{(2n+2)!}{(n+1)!} =
                \lim_{n\rightarrow\infty} \frac{n!\cdot (2n)! \cdot  (2n+1) (2n+2)}{(2n)! \cdot n!\cdot  (n+1)} =
                \lim_{n\rightarrow\infty} \frac{(2n+1) (2n+2)}{(n+1)} = \infty
            \end{align*}
            ותחום ההתכנסות הוא $\reals$.

            \subsection*{סעיף ב}

            לפנינו טור חזקות מהצורה $\sum_{k=10}^\infty a_k(x-1)^k$ כאשר לכל $k$ טבעי נקבל:
            \begin{align*}
                a_k=\begin{cases}
                    \frac{(-1)^{5n}}{n\ln n\cdot 5^n} & k=5n    \\
                    0                                 & k\ne 5n
                \end{cases}
            \end{align*}
            רדיוס ההתכנסות נתון לנו ע"י משפט 6.10 - $\frac{1}{R}=\overline{\lim_{k\rightarrow\infty}}\sqrt[k]{|a_k|}$ \\
            תתי הסדרות $(a_{5n})$, $(a_m): m\ne 5n$ מכסות את $(a_k)$ ומאחר ו$(a_m)$ סדרת אפסים נקבל $\sqrt[m]{|a_m|}=0\xrightarrow[m\rightarrow\infty]{}0$. \\
            הגבול עבור תת-הסדרה $a_{5n}$ יהיה:
            \begin{align*}
                \lim_{n\rightarrow\infty} \sqrt[5n]{\left|\frac{(-1)^{5n}}{n\ln n\cdot 5^n}\right|}=
                \lim_{n\rightarrow\infty} \frac{1}{\sqrt[5n]{n\ln n}\cdot \sqrt[5]{5}}\equals_\ast\frac{1}{\sqrt[5]{5}}
            \end{align*}
            * כי החל מ$N$ מסוים מתקיים אי-השוויון הבא וממנו נובע כלל הסנדוויץ' $\sqrt[5n]{n\ln n}\xrightarrow[n\rightarrow\infty]{}1$
            \begin{align*}
                (\sqrt[n]{n})^\frac{1}{5}= \sqrt[5n]{n} \leq \sqrt[5n]{n\ln n} \leq \sqrt[5n]{n^2} = (\sqrt[n]{n})^\frac{2}{5}
            \end{align*}
            לסיכום, רדיוס ההתכנסות יתקבל לפי $\frac{1}{R}=\overline{\lim_{k\rightarrow\infty}}\sqrt[k]{|a_k|}=\max\{ 0, \frac{1}{\sqrt[5]{5}} \}=\frac{1}{\sqrt[5]{5}}\Rightarrow R=\sqrt[5]{5}$ \\\\
            נבדוק קצוות. בקצה $x=1+\sqrt[5]{5}$ נקבל את טור המספרים $\sum_{n=2}^\infty \frac{(-1)^{5n} \cdot (\sqrt[5]{5})^{5n}}{n\ln n \cdot 5^n} = \sum_{n=2}^\infty \frac{(-1)^n}{n\ln n}$. \\
            טור זה הוא טור לייבניץ $\sum_{n=2}^\infty (-1)^n \lambda_n$ כאשר הסדרה $\lambda_n=\frac{1}{n\ln n}$ חיובית, אפסה, ומונוטונית יורדת כי לכל $n$ טבעי מקבלים:
            \begin{align*}
                \lambda_{n+1}=\frac{1}{(n+1)\ln(n+1)}\leq
                \frac{1}{n\ln(n+1)} \underrel{\text{ln עולה}}{\leq}
                \frac{1}{n\ln n}=\lambda_n
            \end{align*}
            אי-לכך, לפי משפט 5.20 נסיק את התכנסות טור המספרים $\sum_{n=2}^\infty \frac{(-1)^n}{n\ln n}$.\\
            בקצה $x=1-\sqrt[5]{5}$ יתקבל טור המספרים $\sum_{n=2}^\infty \frac{(-1)^{5n} \cdot (-\sqrt[5]{5})^{5n}}{n\ln n \cdot 5^n}=\sum_{n=2}^\infty \frac{1}{n\ln n}$. טור זה מתבדר לפי שאלה $27$א ביחידה 5.\\\\
            לסיכום, נקבל כי תחום ההתכנסות הוא $(1-\sqrt[5]{5}, 1+\sqrt[5]{5}]$.

\subsection*{סעיף ג}

לפנינו טור חזקות $\sum_{n=1}^\infty d(n)x^n$ כאשר $d(n)$ הוא מספר המחלקים של $n$.\\
רדיוס ההתכנסות נתון לנו, לפי $6.10$, ע"י הגבול $R=\overline{\lim_{n\rightarrow\infty}}\sqrt[n]{d(n)}$. \\
מאחר ולכל $n$ טבעי מתקיים $1\leq d(n)\leq n$, ולכן $1\leq \sqrt[n]{d(n)}\leq \sqrt[n]{n}\rightarrow 1$, נסיק מכלל הסנדוויץ' כי $\sqrt[n]{d(n)}\xrightarrow[n\rightarrow\infty]{}1$,\\
ובפרט $R=\overline{\lim_{n\rightarrow\infty}}\sqrt[n]{d(n)}=1$. \\\\
נבדוק התכנסות בקצוות. יתקבלו טורי המספרים $\Sigma (\pm 1)^n d(n)$ בהתאמה. \\
היות ולא מתקיים התנאי ההכרחי להתכנסות טורים $d(n)\rightarrow 0ת (-1)^n d(n)\rightarrow 0$, נסיק כי הטורים מתבדרים.\\\\
לסיכום, תחום ההתכנסות של טור החזקות הוא $(-1, 1)$

\pagebreak

\section*{שאלה 4}

\subsection*{סעיף א}

הטענה נכונה. \\
נסמן לכל $n$ טבעי, $u_n(x): \reals\rightarrow\reals $ פונקציה כך שלכל $x$ ממשי $u_n(x)=\frac{x}{4+n^4x^2}$.
נוכיח את ההתכנסות במ"ש של $f(x)=\Sigma u_n(x)$ בעזרת מבחן ויירשטראס.\\\\
נבחר $\alpha_n=\frac{1}{4n^2}$ \\
לכל $n\in \naturals$, $x\in \reals$ נקבל:
\begin{align*}
    u'_n(x) & =\frac{1(4+n^4x^2)-x(2n^4x)}{(4+n^4x^2)^2}= \\
            & =\frac{4+n^4x^2-2n^4x^2}{(4+n^4x^2)^2}=     \\
            & =\frac{4-n^4x^2}{(4+n^4x^2)^2}
\end{align*}
נקבל נקודות החשודות לערכי קיצון מקומיים כאשר $u'_n(x)=0$, כלומר עבור $x=\pm \frac{2}{n^2}$. \\
מתקיים: $u_n(x)\xrightarrow[x\rightarrow \pm \infty]{}0$ ולכן:
\begin{align*}
    |u_n(x)| & \leq |u_n(\pm \frac{2}{n^2})|=                           \\
             & =|\frac{\pm \frac{2}{n^2}}{4+n^4(\pm \frac{2}{n^2})^2}|= \\
             & =\frac{\frac{|\pm 2|}{n^2}}{4+n^4\cdot \frac{4}{n^4}}=   \\
             & =\frac{\frac{2}{n^2}}{8}=\frac{1}{4n^2}=\alpha_n
\end{align*}
\\\\
הטור $\Sigma \alpha_n=\Sigma (\frac{1}{4} \cdot \frac{1}{n^2})$ מתכנס לפי דוגמה $5.8$א עבור $\alpha=2>1$ ולפי משפט 5.10 עבור $c=\frac{1}{4}\ne 0$
אי לכך, לפי מבחן ויירשטראס 6.7 נסיק כי טור הפונקציות $\Sigma u_n(x)$ מתכנס במ"ש ב$\reals$.\\
כעת, היות ו $u_n$ פנונקציות רציפות ב $\reals$ ומתכנסות במ"ש ב$\reals$ ל $f$, נסיק לפי $6.4*$ כי $f$ רציפה ב$\reals$.

\subsection*{סעיף ב}

הטענה לא נכונה. \\
גם כאן נסמן $u_n: [0,1]\rightarrow \reals$ כך שלכל $x\in[0,1]$ נקבל $u_n(x)=(1-x)x^n=x^n-x^{n+1}$. \\
נציין כי הפונקציות $u_n$ רציפות ב$\reals$ ובפרט ב$[0,1]$. \\\\
נמצא התכנסות נקודתית של טור המספרים $\Sigma u_n(x_0)$ עבור $x_0\in[0,1]$ כלשהו. לכל $k$ נקבל:
\begin{align*}
    S_k & =\sum_{n=1}^k u_n(x_0)=\sum_{n=1}^k x_0^{n}-x_0^{n+1} \equals^{\text{טור טלסקופי}}= x_0 - x_0^{k+1}
\end{align*}
ולכן:
\begin{align*}
    S(x_0)=
    \lim_{k\rightarrow \infty} S_k=
    \lim_{k\rightarrow \infty} x_0 - x_0^{k+1}=
    \begin{cases}
        x_0 & 0\leq x_0 < 1 \\
        0   & x_0 = 1
    \end{cases}
\end{align*}
\\\\
אילו היה טור הפונקציות $\Sigma u_n(x)$ מתכנס במידה שווה ב $[0,1]$, היינו מקבלים לפי $6.4*$ כי הפונקציה $S(x)$ רציפה, בסתירה לכך ש $S(x)$ אינה רציפה בנקודה $x=1$!

\pagebreak

\section*{שאלה 5}

נגדיר פונקציה $f$ וטור פונקציות להלן:
\begin{align*}
    f(x)=\sum_{n=0}^\infty \frac{x^{n+3}}{n! \cdot (n+3)}
    =\sum_{k=3}^\infty \frac{x^{k}}{(k-3)! \cdot k}
\end{align*}
ונרצה לחשב $\frac{1}{27}f(3)$\\\\
ראשית נמצא תחום התכנסות. \\
לפנינו טור חזקות $\Sigma_{k=3}^\infty a_kx^k$ כך שלכל $k\geq 3$ נקבל $a_k=\frac{1}{(k-3)! \cdot k}>0$.\\
רדיוס ההתכנסות יינתן לנו מלמה $6.11$:
\begin{align*}
    R=
    \lim_{k\rightarrow\infty} \frac{a_k}{a_{k+1}}=
    \lim_{k\rightarrow\infty} \frac{\frac{1}{(k-3)! \cdot k}}{\frac{1}{(k-2)! \cdot (k+1)}}=
    \lim_{k\rightarrow\infty} \frac{(k-2)! \cdot (k+1)}{(k-3)! \cdot k}=
    \lim_{k\rightarrow\infty} \frac{(k-2) \cdot (k+1)}{k}=
    \infty
\end{align*}
מכאן שהטור מתכנס לכל $x$ ממשי.\\\\
לכן לפי $6.12$ ניתן לגזור איבר-איבר, ותתקבל פונקציה $f'$ רציפה ב $\reals$ וערכה לכל $x\in \reals$ יהיה:
\begin{align*}
    f'(x)=(\sum_{n=0}^\infty \frac{x^{n+3}}{n! \cdot (n+3)})'\equals_{6.12}
    \sum_{n=0}^\infty (\frac{x^{n+3}}{n! \cdot (n+3)})'=
    \sum_{n=0}^\infty \frac{x^{n+2}}{n!}=
    x^2\sum_{n=0}^\infty \frac{x^{n}}{n!}\equals_{\text{טור ידוע}}
    x^2e^x
\end{align*}
פונקציה זו רציפה לכל אורך הישר ולכן אינטגרבילית. לפי הנוסחה היסודית, נקבל לכל $x\in \reals$:
\begin{align*}
    f(x)-f(0)=\int_0^x f'(t)dt
\end{align*}
היות ו $f(0)$ טור אפסים ולכן מתכנס לאפס, נקבל בסה"כ כי:
\begin{align*}
    f(x)=
    \int_0^x t^2e^tdt =
    \begin{bmatrix}
        u=t^2 & v'=e^t \\
        u'=2t & v=e^t
    \end{bmatrix}=
    t^2e^t\bigg|_0^x-\int_0^x 2t e^tdt =
    x^2e^x - 2\int_0^x t e^tdt = \\
    =\begin{bmatrix}
        u=t  & v'=e^t \\
        u'=1 & v=e^t
    \end{bmatrix}=
    x^2e^x - 2(te^t\bigg|_0^x-\int_0^x e^tdt)=
    x^2e^x - 2xe^x+2e^t\bigg|_0^x=
    (x^2-2x+2)e^x-2
\end{align*}
ובפרט,
\begin{align*}
    \sum_{n=0}^\infty \frac{3^n}{n!\cdot (n+3)} = \frac{1}{27}f(3) = \frac{5e^3-2}{27}
\end{align*}

\end{document}