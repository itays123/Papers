% !TEX program = xelatex
\documentclass{article}
\usepackage[]{amsthm} %lets us use \begin{proof}
\usepackage{amsmath}
\usepackage{enumerate}
\usepackage{xparse}
\usepackage[makeroom]{cancel}
\usepackage[]{amssymb} %gives us the character \varnothing
\usepackage{fontspec}
\usepackage{enumerate}
\usepackage{polyglossia}
\usepackage{relsize}
\usepackage[left=2.0cm, top=2.0cm, right=2.0cm, bottom=2.0cm]{geometry}

\setdefaultlanguage{hebrew}
\setotherlanguage{english}

\setmainfont{[Arial.ttf]}
\newfontfamily\hebrewfont{[Arial.ttf]}

\newcommand\underrel[2]{\mathrel{\mathop{#2}\limits_{#1}}}
\DeclareMathOperator*{\equals}{=}
\def\reals{\mathbb{R}}
\def\naturals{\mathbb{N}}

\title{מטלת מנחה 15 - אינפי 2}
\author{328197462}
\date{20/01/2023}

\begin{document}
\long\def\/*#1*/{}
\maketitle

\section*{שאלה 1}

נתונה סדרת הפונקציות $f_n(x)=\frac{nx}{e^x+n+x}$ המוגדרות (ורציפות) ב$[0, \infty)$. \\
נחשב את הפונקציה הגבולית. לכל $x\in[0, \infty)$ מתקיים:

\begin{align*}
    f(x)=
    \lim_{n\rightarrow \infty} f_n(x)=
    \lim_{n\rightarrow \infty} \frac{nx}{e^x+n+x}=
    \lim_{n\rightarrow \infty} \frac{x}{e^x \cdot \frac{1}{n}+1+x\cdot \frac{1}{n}}=
    \frac{x}{1}=x
\end{align*}

כמו כן, מתקיים לכל $n\in \mathbb{N}, x\in[0, \infty)$:
\begin{align*}
    |f_n(x)-f(x)|=
    |\frac{nx}{e^x+n+x}-x|=
    |\frac{nx-x(e^x+n+x)}{e^x+n+x}|=
    |\frac{-x^2-xe^x}{e^x+n+x}|=
    \frac{x^2+xe^x}{e^x+n+x}
\end{align*}

\subsection*{סעיף א}

ניקח את הסדרה $x_n=n$. מתקיים:
\begin{align*}
    \sup_{x\in[0, \infty)} |f_n(x)-f(x)| \geq
    |f_n(x_n)-f(x_n)|=
    \frac{n^2+ne^n}{e^n+2n}=
    \frac{\frac{n^2}{e^n}+n}{1+2\cdot \frac{n}{e^n}}\xrightarrow[n\rightarrow \infty]{}
    \infty
\end{align*}

מכאן, לפי למה $6.3$, נסיק כי $(f_n)$ לא מתכנסת במידה שווה ל$f$.

\subsection*{סעיף ב}

יהיו $0\leq a < b$ כלשהם. \\
נדגיש כי מתקיים $[a,b]\subseteq [0,\infty)$ והפונקציה הגבולית $f$ נשארת זהה. \\
נבחר את הסדרה $\mu_n=\frac{b^2+be^b}{e^a+n+a}$. לכל $x\in [a,b], n\in\mathbb{N}$ נקבל:
\begin{align*}
    |f_n(x)-f(x)| =
    \frac{x^2+xe^x}{e^x+n+x} \underrel{a\leq x \leq b}{\leq}
    \frac{b^2+be^b}{e^a+n+a}=\mu_n
\end{align*}
מתקיים $\mu_n\xrightarrow[n\rightarrow\infty]{}0$ ($a,b$ קבועים) ולכן לפי שאלה 7 ביחידה 6 נסיק כי $(f_n)$ מתכנסת במ"ש ל$f$. \\
לכן, לפי משפט $6.8$ נקבל
\begin{align*}
    \lim_{n\rightarrow \infty} \int_a^bf_n(x)dx= \int_a^b f(x)dx
\end{align*}
ובכך סיימנו את ההוכחה.

\pagebreak

\section*{שאלה 2}

\pagebreak

\section*{שאלה 4}

\subsection*{סעיף א}

הטענה נכונה. \\
נסמן לכל $n$ טבעי, $u_n(x): \reals\rightarrow\reals $ פונקציה כך שלכל $x$ ממשי $u_n(x)=\frac{x}{4+n^4x^2}$.
נוכיח את ההתכנסות במ"ש של $f(x)=\Sigma u_n(x)$ בעזרת מבחן ויירשטראס.\\\\
נבחר $\alpha_n=\frac{1}{4n^2}$ \\
לכל $n\in \naturals$, $x\in \reals$ נקבל:
\begin{align*}
    u'_n(x) & =\frac{1(4+n^4x^2)-x(2n^4x)}{(4+n^4x^2)^2}= \\
            & =\frac{4+n^4x^2-2n^4x^2}{(4+n^4x^2)^2}=     \\
            & =\frac{4-n^4x^2}{(4+n^4x^2)^2}
\end{align*}
נקבל נקודות החשודות לערכי קיצון מקומיים כאשר $u'_n(x)=0$, כלומר עבור $x=\pm \frac{2}{n^2}$. \\
מתקיים: $u_n(x)\xrightarrow[x\rightarrow \pm \infty]{}0$ ולכן:
\begin{align*}
    |u_n(x)| & \leq |u_n(\pm \frac{2}{n^2})|=                           \\
             & =|\frac{\pm \frac{2}{n^2}}{4+n^4(\pm \frac{2}{n^2})^2}|= \\
             & =\frac{\frac{|\pm 2|}{n^2}}{4+n^4\cdot \frac{4}{n^4}}=   \\
             & =\frac{\frac{2}{n^2}}{8}=\frac{1}{4n^2}=\alpha_n
\end{align*}
\\\\
הטור $\Sigma \alpha_n=\Sigma (\frac{1}{4} \cdot \frac{1}{n^2})$ מתכנס לפי דוגמה $5.8$א עבור $\alpha=2>1$ ולפי משפט 5.10 עבור $c=\frac{1}{4}\ne 0$
אי לכך, לפי מבחן ויירשטראס 6.7 נסיק כי טור הפונקציות $\Sigma u_n(x)$ מתכנס במ"ש ב$\reals$.\\
כעת, היות ו $u_n$ פנונקציות רציפות ב $\reals$ ומתכנסות במ"ש ב$\reals$ ל $f$, נסיק לפי $6.4*$ כי $f$ רציפה ב$\reals$.

\subsection*{סעיף ב}

הטענה לא נכונה. \\
גם כאן נסמן $u_n: [0,1]\rightarrow \reals$ כך שלכל $x\in[0,1]$ נקבל $u_n(x)=(1-x)x^n=x^n-x^{n+1}$. \\
נציין כי הפונקציות $u_n$ רציפות ב$\reals$ ובפרט ב$[0,1]$. \\\\
נמצא התכנסות נקודתית של טור המספרים $\Sigma u_n(x_0)$ עבור $x_0\in[0,1]$ כלשהו. לכל $k$ נקבל:
\begin{align*}
    S_k & =\sum_{n=1}^k u_n(x_0)=\sum_{n=1}^k x_0^{n}-x_0^{n+1} \equals^{\text{טור טלסקופי}}= x_0 - x_0^{k+1}
\end{align*}
ולכן:
\begin{align*}
    S(x_0)=
    \lim_{k\rightarrow \infty} S_k=
    \lim_{k\rightarrow \infty} x_0 - x_0^{k+1}=
    \begin{cases}
        x_0 & 0\leq x_0 < 1 \\
        0   & x_0 = 1
    \end{cases}
\end{align*}
\\\\
אילו היה טור הפונקציות $\Sigma u_n(x)$ מתכנס במידה שווה ב $[0,1]$, היינו מקבלים לפי $6.4*$ כי הפונקציה $S(x)$ רציפה, בסתירה לכך ש $S(x)$ אינה רציפה בנקודה $x=1$!


\end{document}