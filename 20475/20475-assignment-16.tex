% !TEX program = xelatex
\documentclass{article}
\usepackage[]{amsthm} %lets us use \begin{proof}
\usepackage{amsmath}
\usepackage{enumerate}
\usepackage{xparse}
\usepackage[makeroom]{cancel}
\usepackage[]{amssymb} %gives us the character \varnothing
\usepackage{fontspec}
\usepackage{enumerate}
\usepackage{polyglossia}
\usepackage{relsize}
\usepackage[left=2.0cm, top=2.0cm, right=2.0cm, bottom=2.0cm]{geometry}

\setdefaultlanguage{hebrew}
\setotherlanguage{english}

\setmainfont{[Arial.ttf]}
\newfontfamily\hebrewfont{[Arial.ttf]}

\newcommand\underrel[2]{\mathrel{\mathop{#2}\limits_{#1}}}
\DeclareMathOperator*{\equals}{=}

\title{מטלת מנחה 15 - אינפי 2}
\author{328197462}
\date{20/01/2023}

\begin{document}
\long\def\/*#1*/{}
\maketitle

\section*{שאלה 1}

נתונה סדרת הפונקציות $f_n(x)=\frac{nx}{e^x+n+x}$ המוגדרות (ורציפות) ב$[0, \infty)$. \\
נחשב את הפונקציה הגבולית. לכל $x\in[0, \infty)$ מתקיים:

\begin{align*}
    f(x)=
    \lim_{n\rightarrow \infty} f_n(x)=
    \lim_{n\rightarrow \infty} \frac{nx}{e^x+n+x}=
    \lim_{n\rightarrow \infty} \frac{x}{e^x \cdot \frac{1}{n}+1+x\cdot \frac{1}{n}}=
    \frac{x}{1}=x
\end{align*}

כמו כן, מתקיים לכל $n\in \mathbb{N}, x\in[0, \infty)$:
\begin{align*}
    |f_n(x)-f(x)|=
    |\frac{nx}{e^x+n+x}-x|=
    |\frac{nx-x(e^x+n+x)}{e^x+n+x}|=
    |\frac{-x^2-xe^x}{e^x+n+x}|=
    \frac{x^2+xe^x}{e^x+n+x}
\end{align*}

\subsection*{סעיף א}

ניקח את הסדרה $x_n=n$. מתקיים:
\begin{align*}
    \sup_{x\in[0, \infty)} |f_n(x)-f(x)| \geq
    |f_n(x_n)-f(x_n)|=
    \frac{n^2+ne^n}{e^n+2n}=
    \frac{\frac{n^2}{e^n}+n}{1+2\cdot \frac{n}{e^n}}\xrightarrow[n\rightarrow \infty]{}
    \infty
\end{align*}

מכאן, לפי למה $6.3$, נסיק כי $(f_n)$ לא מתכנסת במידה שווה ל$f$.

\subsection*{סעיף ב}

יהיו $0\leq a < b$ כלשהם. \\
נדגיש כי מתקיים $[a,b]\subseteq [0,\infty)$ והפונקציה הגבולית $f$ נשארת זהה. \\
נבחר את הסדרה $\mu_n=\frac{b^2+be^b}{e^a+n+a}$. לכל $x\in [a,b], n\in\mathbb{N}$ נקבל:
\begin{align*}
    |f_n(x)-f(x)| =
    \frac{x^2+xe^x}{e^x+n+x} \underrel{a\leq x \leq b}{\leq}
    \frac{b^2+be^b}{e^a+n+a}=\mu_n
\end{align*}
מתקיים $\mu_n\xrightarrow[n\rightarrow\infty]{}0$ ($a,b$ קבועים) ולכן לפי שאלה 7 ביחידה 6 נסיק כי $(f_n)$ מתכנסת במ"ש ל$f$. \\
לכן, לפי משפט $6.8$ נקבל
\begin{align*}
    \lim_{n\rightarrow \infty} \int_a^bf_n(x)dx= \int_a^b f(x)dx
\end{align*}
ובכך סיימנו את ההוכחה.


\end{document}