% !TEX program = xelatex
\documentclass{article}
\usepackage[]{amsthm} %lets us use \begin{proof}
\usepackage{amsmath}
\usepackage{enumerate}
\usepackage{xparse}
\usepackage[makeroom]{cancel}
\usepackage[]{amssymb} %gives us the character \varnothing
\usepackage{fontspec}
\usepackage{enumerate}
\usepackage{polyglossia}
\usepackage{relsize}
\usepackage[left=2.0cm, top=2.0cm, right=2.0cm, bottom=2.0cm]{geometry}

\setdefaultlanguage{hebrew}
\setotherlanguage{english}

\setmainfont{[Arial.ttf]}
\newfontfamily\hebrewfont{[Arial.ttf]}

\newcommand\underrel[2]{\mathrel{\mathop{#2}\limits_{#1}}}
\DeclareMathOperator*{\equals}{=}

\title{מטלת מנחה 14 - אינפי 2}
\author{328197462}
\date{23/12/2022}

\begin{document}
\long\def\/*#1*/{}
\maketitle

\section*{שאלה 1}

הפונקציה לפיתוח: $f(x)=\sqrt{1+x^2}=(1+x^2)^{1/2}$ \\
נקודת הפיתוח: $a=0$, וצריך למצוא $f(0.1)$. \\
סדר הפיתוח: $n\in\mathbb{N}$, כך ש $|R_n(x)|\leq 0.5\cdot 10^{-4}$.
\\\\
הפונקציה גזירה בקטע. נחשב נגזרות עד סדר 4:
\begin{align*}
    f'(x)      & =\frac{1}{2} (1+x^2)^{-1/2} \cdot 2x = \frac{x}{(1+x^2)^{1/2}}                                                               \\
    f^{(2)}(x) & = \frac{1 \cdot (1+x^2)^{1/2} - x \cdot \frac{x}{(1+x^2)^{1/2}}}{(1+x^2)} = \frac{1+x^2-x^2}{(1+x^2)^{3/2}} = (1+x^2)^{-3/2} \\
    f^{(3)}(x) & = -\frac{3}{2} (1+x^2)^{-5/2} \cdot 2x = -3 \cdot \frac{x}{(1+x^2)^{5/2}}                                                    \\
    f^{(4)}(x) & = -3 \cdot \frac{1\cdot (1+x^2)^{5/2} - x \cdot \frac{5}{2}(1+x^2)^{3/2}\cdot 2x}{(1+x^2)^5}                                 \\
               & = -3(1+x^2)^{3/2} \cdot \frac{(1+x^2)^1-5x^2}{(1+x^2)^5} = \frac{-3(1-4x^2)}{(1+x^2)^{7/2}}
\end{align*}
לפי לגראנז', קיים $c\in (0, 0.1)$ כך ש $R_3(0.1)=\frac{f^{(4)}(c)}{4!}\cdot 0.1^4$. לכן,
\begin{align*}
    |R_3(0.1)| & = \left| \frac{-3(1-4c^2)}{(1+c^2)^{7/2}\cdot 24} \cdot 0.1^4 \right| =
    \frac{3(1-4c^2)}{24 (1+c^2)^{7/2}} \cdot 10^{-4}                                              \\
               & \leq \frac{3}{24} \cdot 10^-4 = \frac{1}{8} \cdot 10^{-4} \leq 0.5 \cdot 10^{-4}
\end{align*}
ערכי הנגזרת בנקודה $a=0$ יהיו:
\begin{align*}
    \begin{matrix}
        f(0)=1 & f'(0)=0 & f^{(2)}(0)=1 & f^{(3)}(0) = 0
    \end{matrix}
\end{align*}
ולכן פיתוח מקלורן מסדר $n=3$ יהיה:
\begin{align*}
    f(x) & =1 + \frac{0}{1!}x + \frac{1}{2!} x^2 + \frac{0}{3!} x^3 + R_3(x) \\
         & = 1 + \frac{1}{2}x^2 + R_3(x)
\end{align*}
ונקבל
\[
    \sqrt{1.01} = f(0.1) = 1.005 \pm 0.5 \cdot 10^{-4}
\]
בדיקה: מחישוב נקבל $\sqrt{1.01}=1.00498756211$, והשארית תהיה $0.00005>0.00001243788$

\pagebreak

\section*{שאלה 2}

יהיו $f,g$ פונקציות כנדרש. ננסה להוכיח באופן דומה לשאלה 2א בעמוד 93 בכרך ב. \\
נסמן $P(x)=\sum_{k=0}^{n}(\sum_{j=0}^ka_jb_{k-j})x^k$
ועלינו להוכיח
\[
    \lim_{x\rightarrow 0} \frac{f(x)g(x)-P(x)}{x^n}=0
\]
לפי הנתון מתקיים $f(x)=\sum_{k=0}^na_kx^k + R_n(x), g(x)=\sum_{k=0}^nb_kx^k+\rho_n(x)$, \\
כאשר לפי משפט $4.7$:
\[
    \begin{matrix}
        \lim_{x\rightarrow 0}\frac{R_n(x)}{x^n} = 0 &
        \lim_{x\rightarrow 0}\frac{\rho_n(x)}{x^n} = 0
    \end{matrix}
\]
כפל הפולינומים ייתן לנו פולינום ממעלה $2n$:
\[
    \sum_{k=0}^na_kx^k \cdot \sum_{k=0}^nb_kx^k =
    P(x) + \sum_{k=n+1}^{2n} c_kx^k
\]
כאשר ניתן לחשב את הערכים $c_k$, אך אין בכך צורך. מכאן נסיק:
\[
    f(x)g(x) = (\sum_{k=0}^na_kx^k + R_n(x))(\sum_{k=0}^nb_kx^k+\rho_n(x)) =
\]
\[
    P(x) + \sum_{k=n+1}^{2n} c_kx^k + R_n(x)(\sum_{k=0}^nb_kx^k+\rho_n(x)) + \rho_n(x) \cdot \sum_{k=0}^na_kx^k
\]
ולכן
\[
    \frac{f(x)g(x)-P(x)}{x_n} =
    \sum_{k=n+1}^{2n} c_kx^{k-n} + \frac{R_n(x)}{x^n}\cdot(\sum_{k=0}^nb_kx^k+\rho_n(x)) + \frac{\rho_n(x)}{x^n}\cdot \sum_{k=0}^na_kx^k
\]
נראה כי שלושת הביטויים הנ"ל שואפים לאפס כאשר $x\rightarrow 0$. \\
הביטוי הראשון, $\sum_{k=n+1}^{2n} c_kx^{k-n}=\sum_{k=1}^{n} c_{k+n}x^k$,
הוא פולינום ומכאן רציף ב-0 והגבול שלו באפס הוא $0$ (אין לו מקדם חופשי). \\
באופן דומה, גם הפולינומים $\sum_{k=0}^na_kx^kת \sum_{k=0}^nb_kx^k$ שואפים ל $a_0, b_0$ בהתאמה ב$0$. \\
לכן מתקיים:
\[
    \lim_{x\rightarrow 0} \frac{\rho_n(x)}{x^n}\cdot \sum_{k=0}^na_kx^k =
    \lim_{x\rightarrow 0} \frac{\rho_n(x)}{x^n}\cdot \lim_{x\rightarrow 0}\sum_{k=0}^na_kx^k =
    0 \cdot a_0 = 0
\]
לבסוף, הפונקציה $\rho_n(x)=f(x)-\sum_{k=0}^nb_kx^k$
רציפה באפס כהפרש של פונקציות רציפות (הרציפות של $f$ נובעת מגזירותה). \\
לכן $\rho_n(x)\xrightarrow[x\rightarrow 0]{}\rho_n(0)=0$ ונקבל:
\[
    \lim_{x\rightarrow 0} \frac{R_n(x)}{x^n}\cdot(\sum_{k=0}^nb_kx^k+\rho_n(x)) =
    \lim_{x\rightarrow 0} \frac{R_n(x)}{x^n}\cdot \lim_{x\rightarrow 0}(\sum_{k=0}^nb_kx^k+\rho_n(x)) =
    0 (b_0+0)=0
\]
לסיכום, נקבל
\[
    \lim_{x\rightarrow 0}\frac{f(x)g(x)-P(x)}{x_n}=0
\]
ולכן לפי משפט 4.8 הפולינום $P(x)$
הוא פולינום מקלורן של $f(x)g(x)$.

\pagebreak

\section*{שאלה 3}

\subsection*{סעיף א}

נחשב את הגבול בעזרת פיתוחי מקלורן ידועים. נתחיל מפיתוח המכנה.
לפי שאלה 19ג ביחידה 4, $\tan x = x + \frac{x^3}{3}+R_3(x)$. \\
אי לכך,
\[
    x(\tan x -x) =
    x (x + \frac{x^3}{3}+R_3(x) - x) =
    \frac{x^4}{3} + x\cdot R_3(x)
\]
מאחר ובמכנה נתקלנו בביטוי ממעלה 4, נפתח את המונה בסדר $n=4$.\\
לפי פיתוחים ידועים:
\[
    \begin{matrix}
        e^t = 1 + t + \frac{t^2}{2} + \frac{t^3}{6}+\frac{t^4}{24}+S_4(t) & \underrel{\text{שאלה 2א בעמוד 93}}{\Rightarrow} e^{x^2}=1+x^2+\frac{x^4}{4}+S_4(x) \\
        \tan x = x+ \frac{x^3}{3} + V_3(x)                                & \sin x = x - \frac{x^3}{6} + M_3(x)
    \end{matrix}
\]
נקבל, ע"פ שאלה 2 בממ"ן זה, כי פיתוח מקלורן של הביטוי $x(1+e^{x^2})\tan x$ יהיה:
\[
    x(2+x^2+\frac{x^4}{4}+\cdots)(x+\frac{x^3}{3}+\cdots)=(2x+x^3+\cdots)(x+\frac{x^3}{3}+\cdots)=2x^2+\frac{5x^4}{3}+T_4(x)
\]
וכן,
\[
    \sin^2 x = \sin x \cdot \sin x=
    (x-\frac{x^3}{6}+\cdots)(x-\frac{x^3}{6}+\cdots) =
    x^2-\frac{x^4}{3}+Q_4(x)
\]
נציב בגבול הנתון את הביטויים שקיבלנו:
\[
    \lim_{x\rightarrow 0} \frac{x(1+e^{x^2})\tan x - 2\sin^2 x}{x(\tan x - x)}=
    \lim_{x\rightarrow 0} \frac{2x^2+\frac{5}{3}x^4+T_4(x) - 2x^2+\frac{2x^4}{3}-2Q_4(x)}{\frac{x^4}{3}+x\cdot R_3(x)}=
    \lim_{x\rightarrow 0} \frac{\frac{7}{3}x^4+T_4(x)-2Q_4(x)}{\frac{x^4}{3}+x\cdot R_3(x)}=
\]
\[
    \lim_{x\rightarrow 0} \frac{\frac{7}{3}+\frac{T_4(x)}{x^4}-2\frac{Q_4(x)}{x^4}}{\frac{1}{3}+\frac{R_3(x)}{x^3}} \underrel{4.7}{=}
    \frac{\frac{7}{3}}{\frac{1}{3}} = 7
\]

\subsection*{סעיף ב}

נגדיר פונקציה $f$ כלהלן:
\[
    f(x)=\frac{e^x -\sin x-\cos x}{\ln(1-x^2)}
\]
ונניח (נוכיח את הדבר בהמשך) כי קיים ל$f$ גבול ב-0, סופי או אינסופי. \\
אז לפי הגדרת היינה מאינפי 1,עבור כל סדרה $(x_n)_{n=1}^\infty$ המקיימת $x_n\xrightarrow[n\rightarrow \infty]{} 0$,
ובפרט עבור הסדרה $x_n=\frac{1}{n}$,
נקבל:
\[
    \lim_{n\rightarrow \infty}f(x_n)=\lim_{x\rightarrow 0}f(x)
\]
נחשב:
\[
    f(x_n)=
    f(\frac{1}{n})=
    \frac{e^{1/n} -\sin \frac{1}{n} -\cos \frac{1}{n}}{\ln(1-\frac{1}{n^2})} \underrel{\ast}{=}
    \frac{e^{1/n} -\sin \frac{1}{n} -\cos \frac{1}{n}}{\ln(n^2-1)-2\ln n}
\]

* כי $\ln(1-\frac{1}{n^2}) = \ln(\frac{n^2-1}{n^2})=\ln(n^2-1)-2\ln n$. \\
אי לכך,
\[
    \lim_{n\rightarrow\infty} \frac{e^{1/n} -\sin \frac{1}{n} -\cos \frac{1}{n}}{\ln(n^2-1)-2\ln n} =
    \lim_{x\rightarrow 0} \frac{e^x -\sin x-\cos x}{\ln(1-x^2)}
\]
\\\\
על מנת לחשב את גבול הפונקציה ננסה לפתח את המונה והמכנה לפי $n=2$. \\
ניעזר בפיתוחים הידועים:
\[
    \begin{matrix}
        \ln(1+t) = t + R_1(t)               & \underrel{\text{שאלה 2א בעמוד 93}}{\Rightarrow} & \ln(1-x^2)=-x^2+R_2(x) \\
        e^x = 1 + x + \frac{x^2}{2}+ S_2(x) &
        \sin x = x - Q_1(x)                 &
        \cos x = 1 - \frac{x^2}{2} + T_2(x)
    \end{matrix}
\]
לכן המונה יהיה:
\[
    e^x-\sin x - \cos x =
    (1 + x + \frac{x^2}{2}+ S_2(x)) - (x + Q_1(x)) - (1 - \frac{x^2}{2} + T_2(x)) =
\]
\[
    1+x+\frac{x^2}{2}+ S_2(x) - x - Q_1(x) - 1 + \frac{x^2}{2} - T_2(x)=
    x^2+S_2(x)-Q_1(x)-T_2(x)
\]
והמכנה יהיה $\ln(1-x^2)=-x^2+R_2(x)$. נקבל אפוא:
\[
    \lim_{x\rightarrow 0} f(x) =
    \lim_{x\rightarrow 0} \frac{x^2+S_2(x)-Q_1(x)-T_2(x)}{-x^2+R_2(x)} =
    \lim_{x\rightarrow 0} \frac{1+\frac{S_2(x)}{x^2}-\frac{Q_1(x)}{x^2}-\frac{T_2(x)}{x^2}}{-1+\frac{R_2(x)}{x^2}} = -1
\]
כי לפי הערת השוליים בעמוד 65 נקבל $Q_1(x)=Q_2(x)$, וכן ממשפט 4.7 נסיק:
\[
    \lim_{x\rightarrow 0} \frac{R_2(x)}{x^2} =
    \lim_{x\rightarrow 0} \frac{S_2(x)}{x^2} =
    \lim_{x\rightarrow 0} \frac{Q_2(x)}{x^2} =
    \lim_{x\rightarrow 0} \frac{T_2(x)}{x^2} = 0
\]
לסיכום נקבל
\[
    \lim_{n\rightarrow \infty} f(\frac{1}{n}) = -1
\]

\pagebreak

\section*{שאלה 4}

יהי $x_0\in [0,1]$ ונרצה להוכיח $|f'(x_0)|\leq \frac{A}{2}$ \\
נרשום פיתוח טיילור של הפונקציה $f(x)$ מסדר 1 סביב הנקודה $x=x_0$,
עם שגיאה בצורת לגראנז':
\[
    f(x)=f(x_0)+f'(x_0)(x-x_0)+\frac{f''(c)}{2}(x-x_0)^2 \;\;\;\; (\text{$c$ בין 0 ו$x_0$})
\]
נציב $x=0,1$:
\[
    \begin{cases}
        f(0)=f(x_0)+f'(x_0)(-x_0)+\frac{f''(c_1)}{2}x_0^2      & c_1\in (0,x_0) \\
        f(1)=f(x_0)+f'(x_0)(1-x_0)+\frac{f''(c_2)}{2}(1-x_0)^2 & c_2\in(x_0,1)
    \end{cases}
\]
לפי הנתון $f(0)=f(1)$. נחסר בין המשוואות ונקבל:
\[
    0=f(0)-f(1)=
    f'(x_0)(-x_0-1+x_0)+\frac{1}{2}f''(c_1)\cdot x_0^2 - \frac{1}{2}f''(c_2)\cdot (1-x_0)^2
\]
ומכאן
\[
    f'(x_0)=\frac{1}{2}f''(c_1)\cdot x_0^2 - \frac{1}{2}f''(c_2)\cdot (1-x_0)^2
\]
ולפי אי-שוויון המשולש:
\[
    |f'(x_0)|\leq
    \frac{1}{2} |f''(c_1)| \cdot x_0^2 + \frac{1}{2}|f''(c_2)|\cdot (1-x_0)^2 \underrel{\text{לפי הנתון}}{\leq}
    \frac{1}{2} A \cdot x_0^2 + \frac{1}{2}A\cdot (1-x_0)^2=
    \frac{A}{2} (x_0^2 + (1-x_0)^2)\underrel{\ast}{\leq} \frac{A}{2}
\]

* מאחר ו $0\leq x_0, (1-x_0) \leq 1$, נקבל $x_0^2+(1-x_0)^2\leq x_0+(1-x_0)=1$
\\\\
מכאן נסיק כי לכל $x\in[0,1]$ מתקיים $|f'(x)|\leq \frac{A}{2}$ ובכך סיימנו את ההוכחה.

\section*{שאלה 5}

לפי הנתונים, יש נקודה פנימית $x_m\in (a,b)$
כך ש $f(x_m)= m =\min f([a,b])$.\\
כלומר - $x_m$ נקודת קיצון מקומית,
ולכן לפי אינפי 1 מתקיים $f'(x_m)=0$.\\
נרשום פיתוח טיילור של $f(x)$ מסדר 1 סביב הנקודה $x_m$ עם השארית בצורת לגראנז':
\[
    f(x)=f(x_m)+f'(x_m)(x-x_m)+\frac{f''(c)}{2}(x-x_m)^2=
    m + \frac{f''(c)}{2}(x-x_m)^2 \;\;\;\; \text{$c$ בין $x_m$ ו$x$}
\]

לפי הנתון, יש $x_M\in [a,b]$
כך ש $f(x_M)=M$. נציב בפיתוח ונקבל כי קיים $c$ בין $x_m, x_M$ כך ש:
\[
    M=f(x_M) = m + \frac{f''(c)}{2}(x_M-x_m)^2
\]
נקבל $0\leq M-m=\frac{f''(c)}{2}(x_M-x_m)^2$.\\
נשים לב כי $|x_M - x_m|$, המרחק בין שני איברים בקטע $[a,b]$, בוודאות קטן או שווה לאורך הקטע $|b-a|$. \\
אי לכך, $(x_M-x_m)^2\leq (b-a)^2$, ומכאן נסיק $0\leq M-m\leq \frac{f''(c)}{2}(b-a)^2$\\
נכפול את אי-השוויון ב-2 ונחלק ב $(b-a)^2\ne 0$ (ידוע כי $x_m\in (a,b)$ ולכן $|b-a|\ne 0$), ונקבל:
\[
    f''(c)\geq \frac{2(M-m)}{(b-a)^2} \geq 0
\]
ובאופן ישיר $|f''(c)|\geq \frac{2(M-m)}{(b-a)^2}$.

\end{document}