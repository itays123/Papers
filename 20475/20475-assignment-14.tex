% !TEX program = xelatex
\documentclass{article}
\usepackage[]{amsthm} %lets us use \begin{proof}
\usepackage{amsmath}
\usepackage{enumerate}
\usepackage{xparse}
\usepackage[makeroom]{cancel}
\usepackage[]{amssymb} %gives us the character \varnothing
\usepackage{fontspec}
\usepackage{enumerate}
\usepackage{polyglossia}
\usepackage{relsize}
\usepackage[left=2.0cm, top=2.0cm, right=2.0cm, bottom=2.0cm]{geometry}

\setdefaultlanguage{hebrew}
\setotherlanguage{english}

\setmainfont{[Arial.ttf]}
\newfontfamily\hebrewfont{[Arial.ttf]}

\newcommand\underrel[2]{\mathrel{\mathop{#2}\limits_{#1}}}
\DeclareMathOperator*{\equals}{=}

\title{מטלת מנחה 13 - אינפי 2}
\author{328197462}
\date{23/12/2022}

\begin{document}
\long\def\/*#1*/{}
\maketitle

\section*{שאלה 1}

ניעזר בפיתוח הידוע לפונקציה $(1+t)^{1/2}$
שבעמוד 66 בכרך ב:
\[
    (1+t)^{1/2} = \sum_{k=0}^n \binom{1/2}{k}t^k+R_n(t)
\]
נציב $t=x^2$ ונקבל:
\[
    (1+x^2)^{1/2} = \sum_{k=0}^n \binom{1/2}{k}x^{2k}+R_n(x^2)
\]
לפי שאלה 2א בעמוד 93 בכרך ב, הצבנו פולינום $t=x^2$ המתאפס ב$x=0$,
ולכן ההצגה לעיל היא פיתוח מקלורן מסדר $n$ של $(1+x^2)^2$\\
נעתיק אך ורק את המחוברים ממעלה $n$ ומטה:
\[
    (1+x^2)^{1/2} =
    \sum_{k=0}^{\lfloor n/2 \rfloor} \binom{1/2}{k}x^{2k} + R_n(x) = 1 + \frac{1}{2}x^2+\frac{-1}{8}x^4+\cdots + \binom{1/2}{\lfloor n/2 \rfloor}x^{2\lfloor n/2 \rfloor} + R_n(x)
\]
\\\\
נמצא $n$ כך שהשגיאה $|R_n(0.1)|$ לא תעלה על $0.5*10^{-4}$:
לשם כך, ניעזר בפיתוח המקורי. לפי עמוד 66, עבור $f(x)=(1+t)^{1/2}$ לכל $k$ מתקיים:
\[
    \begin{matrix}
        f^{(k)}(t)=\frac{1}{2}\cdot \frac{-1}{2} \cdots (\frac{1}{2}-k+1)(1+t)^{1/2-k} \\
        f^{(n+1)}(t)=\frac{1}{2}\cdot \frac{-1}{2} \cdots (\frac{1}{2}-n+1)(\frac{1}{2}-n)(1+t)^{-1/2-n}
    \end{matrix}
\]
ולכן, פונקציית השגיאה $R_n(t)$ בהצגת לגראנז' תהא:
\[
    R_n(0.1) = \frac{f^{(n+1)}(c)}{(n+1)!} \cdot 0.1^n=
    \binom{1/2}{n+1}\cdot (1+c)^{-1/2-n} \cdot 0.1^n
\]

%% Comment
\/*
הפונקציה לפיתוח: $f(x)=\sqrt{1+x^2}$ \\
נקודת הפיתוח: $a=0$ \\
סדר הפיתוח: נמצא $n$,
כך שהשגיאה לא תעלה על $0.5\cdot 10^-4=\frac{1}{2}\cdot \frac{1}{10000}=\frac{1}{20000}$ \\
צריך לחשב $f(x)=P_n(x)+R_n(x)$ כך ש:
\[
    P_n(x) \equals_{(4.2)}\sum_{k=0}^n \frac{f^{(k)}}{k!}\cdot x^n
\]
ולהציב $f(0.1)=\sqrt{1+0.1^2}=\sqrt{1.01}$
\\\\
ניעזר בהצגת השארית לפי לגראנז': \\
\[
    R_n(0.1)=\frac{f^{(n+1)}(c)}{(n+1)!}\cdot {0.1}^n, \;\; c\in (0,0.1)
\]
ננסה כעת להציב ערכים קטנים של $n$ עד שנגיע לשגיאה הדרושה.
\[
    \begin{matrix}
        f(x) = (1+x^2)^{-\frac{1}{2}}                                                \\
        f'(x) = \frac{1}{2}(1+x^2)^{-\frac{1}{2}} \cdot 2x = x(1+x^2)^{-\frac{1}{2}} \\
        f''(x) = (1+x^2)^{-\frac{1}{2}} - x \cdot \frac{1}{2}(1+x^2)^{-\frac{3}{2}} \cdot 2x = (1+x^2)^{-\frac{1}{2}} - x^2(1+x^2)^{-\frac{3}{2}}
    \end{matrix}
\]
*/

\end{document}