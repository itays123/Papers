% !TEX program = xelatex
\documentclass{article}
\usepackage[]{amsthm} %lets us use \begin{proof}
\usepackage{amsmath}
\usepackage{enumerate}
\usepackage{xparse}
\usepackage[makeroom]{cancel}
\usepackage[]{amssymb} %gives us the character \varnothing
\usepackage{fontspec}
\usepackage{polyglossia}
\usepackage{relsize}
\usepackage[left=2.0cm, top=2.0cm, right=2.0cm, bottom=2.0cm]{geometry}

\setdefaultlanguage{hebrew}
\setotherlanguage{english}

\setmainfont{[Arial.ttf]}
\newfontfamily\hebrewfont{[Arial.ttf]}

\newcommand\underrel[2]{\mathrel{\mathop{#2}\limits_{#1}}}

\title{מטלת מנחה 13 - אינפי 2}
\author{328197462}
\date{09/12/2022}

\begin{document}
\maketitle

\section*{שאלה 1}

\pagebreak

\section*{שאלה 2}

נסמן $f(x)=\frac{x^\alpha(1-x^2)^\beta}{1-\cos x}$.
עלינו לבדוק את ההתכנסות את האינטגרל $\int_0^1f(x)dx$. \\
נדגיש כי הפונקציה חיובית לכל אורך קטע האינטגרציה הנתון.

\[
    \int_0^1 f(x) dx = \int_0^{1/2}f(x)dx + \int_{1/2}^1f(x)dx
\]
האינטגרל משמאל מתכנס אם ורק אם שני האינטגרלים מימין מתכנסים. נבדוק את ההתכנסות של כל אחד מהם בנפרד.
\\\\
נבדוק את התכנסות האינטגרל $\int_0^{1/2}f(x)dx$. \\
נבחר $g(x) = \frac{x^\alpha}{x^2}=\frac{1}{x^{2-\alpha}}$. אז:
\[
    \lim_{x\rightarrow 0^+}\frac{f(x)}{g(x)} =
    \lim_{x\rightarrow 0^+}\frac{x^\alpha(1-x^2)^\beta \cdot x^2}{(1-\cos x)\cdot x^\alpha} =
    \lim_{x\rightarrow 0^+} (1-x^2)^\beta \cdot \lim_{x\rightarrow 0^+} \frac{x^2}{1-\cos x} \underrel{\text{לופיטל}}{=}
    1 \cdot \lim_{x\rightarrow 0^+} \frac{2x}{\sin x} = 2 > 0
\]
האינטגרל $\int_0^{1/2}g(x)dx=\int_0^{1/2}\frac{1}{x^{2-\alpha}}dx$
מתכנס אם ורק אם $2-\alpha < 1$,
כלומר $\alpha > 1$
(לפי למה $3.2$). \\
לכן לפי מבחן ההשוואה $3.5*$
גם האינטגרל $\int_0^{1/2}f(x)dx$
מתכנס אם ורק אם $\alpha > 1$.
\\\\
נמשיך ונבדוק את התכנסות האינטרל $\int_{1/2}^1f(x)dx$. \\
נבחר, הפעם, $g(x)=(1-x)^\beta=\frac{1}{(1-x)^{-\beta}}$. אז:
\[
    \lim_{x\rightarrow 1^-} \frac{f(x)}{g(x)} =
    \lim_{x\rightarrow 1^-} \frac{x^\alpha (1-x^2)^\beta}{(1-\cos x)\cdot (1-x)^\beta} =
    \lim_{x\rightarrow 1^-} \frac{x^\alpha}{1-\cos x} \cdot \lim_{x\rightarrow 1^-} \frac{((1-x)(1+x))^\beta}{(1-x)^\beta} =
\]
\[
    \frac{1}{1-\cos 1} \cdot \lim_{x\rightarrow 1^-} (1+x)^\beta =
    \frac{2^\beta}{1-\cos 1} > 0
\]
האינטגרל $\int_{1/2}^1g(x)dx = \int_{1/2}^1\frac{1}{(1-x)^{-\beta}}$
מתכנס, לפי שאלה 5 ביחידה 3, אם ורק אם $-\beta < 1$, כלומר $\beta > -1$ \\
לכן לפי מבחן ההשוואה $3.5*$
גם האינטגרל $\int_{1/2}^1f(x)dx$
מתכנס אם ורק אם $\beta > -1$.
\\\\
לסיכום נקבל שהאינטגרל $\int_0^1f(x)dx$
מתכנס אם ורק אם $\alpha > 1\wedge \beta > -1$

\section*{שאלה 3}

\subsection*{סעיף א}

הטענה נכונה. נוכיח את נכונותה ישירות מהגדרת הגבול. \\
תהא, אם כן, $f$
אינטגרבילית ב$[0,t]$
לכל $t>0$
כך ש $\int_0^\infty f(x)dx$ מתכנס,
ותהא $g$ פונקציה המקיימת $g(x)\geq x$ לכל $x\geq 0$.
\\\\
יהא $\epsilon >0$.
עלינו למצוא $M$
ממשי כך שלכל $x>M$ מתקיים $\left| \int_x^{g(x)}f(x)dx \right| < \epsilon$. \\
לפי מבחן קושי (משפט $3.15$),
עבור $\epsilon$ יש $M_0>0$
כך שלכל שני מספרים $r,s\in(M_0, \infty)$ נקבל $\left| \int_r^sf(x)dx \right| < \epsilon$. \\
נבחר $M=M_0$. נקבל אפוא כי לכל $x>M$ מתקיים, לפי הנתון, $g(x)\geq x > M$. \\
לכן $x, g(x)\in (M_0, \infty)$ מקיימים $\left| \int_x^{g(x)}f(x)dx \right| < \epsilon$
ובכך סיימנו את ההוכחה.

\subsection*{סעיף ב}

הטענה שגויה. נציג דוגמה נגדית. \\
נבחר $f(x)=-x$. ברור כי $f$ רציפה בקטע $[1, \infty)$,
וכן בקטע זה נקבל $f(x) \leq -1 < 0 < \frac{1}{x^2}$ כנדרש.
\\\\
האינטגרל $\int_1^{\infty} f(x)dx$ מתכנס אם ורק אם מתכנס האינטגרל $\int_1^{\infty}-f(x)dx$,
ובמקרה זה נקבל $\int_1^{\infty}f(x)dx=-\int_1^{\infty}-f(x)dx$. \\
אבל האינטגרל $\int_1^{\infty}-f(x)dx=\int_1^{\infty}xdx=\int_1^{\infty} \frac{1}{x^{-1}} dx$
לא מתכנס (כי $-1<1$)
ולכן גם האינטגרל $\int_1^{\infty}f(x)dx$ לא מתכנס.

\pagebreak

\section*{שאלה 4}

\subsection*{סעיף א}

הטענה לא שגויה. ניקח למשל $f$ כך שלכל $x\leq 1$:
\[
    f(x) = \begin{cases}
        0 & x\notin \mathbb{Z} \\
        x & x\in \mathbb{Z}
    \end{cases}
\]
בסעיף 3.2.6 (עמוד 38 בכרך ב)
מוכיחים בעזרת למה 1.25 כי $\int_1^\infty f(x)dx$
אכן מתכנס.
\\\\
אבל $\lim_{x\rightarrow\infty}\frac{f(x)}{x}\ne 0$,
כי עבור $\epsilon=1$ עבור כל $x$ בתחום ההגדרה ניקח את $\lfloor x \rfloor + 1 > x$ ונקבל:
\[
    \frac{f(\lfloor x \rfloor + 1)}{\lfloor x \rfloor + 1} \underrel{הארגומנט שלם}{=}
    \frac{\lfloor x \rfloor + 1}{\lfloor x \rfloor + 1} =
    1 \geq \epsilon
\]

\end{document}