% !TEX program = xelatex
\documentclass{article}
\usepackage[]{amsthm} %lets us use \begin{proof}
\usepackage{amsmath}
\usepackage{enumerate}
\usepackage{xparse}
\usepackage[makeroom]{cancel}
\usepackage[]{amssymb} %gives us the character \varnothing
\usepackage{fontspec}
\usepackage{enumerate}
\usepackage{polyglossia}
\usepackage{relsize}
\usepackage[left=2.0cm, top=2.0cm, right=2.0cm, bottom=2.0cm]{geometry}

\setdefaultlanguage{hebrew}
\setotherlanguage{english}

\setmainfont{[Arial.ttf]}
\newfontfamily\hebrewfont{[Arial.ttf]}

\newcommand\underrel[2]{\mathrel{\mathop{#2}\limits_{#1}}}

\title{מטלת מנחה 13 - אינפי 2}
\author{328197462}
\date{09/12/2022}

\begin{document}
\maketitle

\section*{שאלה 1}

\subsection*{סעיף א}

נסמן $f(x)=\frac{\sqrt{x-1}}{\sqrt{x^2-x+1}-1}$ חיובית בקטע $(1,2]$.
הפונקציה רציפה בקטע זה (ולכן אינטגרבילית בכל קטע סגור החלקי לקטע זה) ואינה חסומה אך ורק בסביבת $x=1$,
ולכן נשתמש במבחן ההשוואה לפונקציות שאינן חסומות בקטע סגור.\\\\
ניקח $g(x)=\frac{\sqrt{x-1}}{x-1}=\frac{1}{(x-1)^{1/2}}$. אז:
\[
    \lim_{x\rightarrow 1^+}\frac{f(x)}{g(x)} =
    \lim_{x\rightarrow 1^+}\frac{x-1}{\sqrt{x^2-x+1}-1} \underrel{\text{לופיטל}}{=}
    \lim_{x\rightarrow 1^+}\frac{1}{\frac{2x-1}{2\sqrt{x^2-x+1}}} =
    \lim_{x\rightarrow 1^+}\frac{2\sqrt{x^2-x+1}}{2x-1} =
    \frac{2\sqrt{1^2-1+1}}{2\cdot 1-1} = 2 > 0
\]
האינטגרל $\int_1^2 g(x)dx = \int_1^2 \frac{1}{(x-1)^{1/2}}dx$
מתכנס לפי שאלה 5 ביחידה 3 כי $\frac{1}{2}<1$, \\
ולכן לפי מבחן ההשוואה $*3.5$
גם $\int_1^2f(x)dx$ מתכנס.
מאחר והפונקציה חיובית נסיק כי $\int_1^2f(x)dx$ מתכנס בהחלט.

\subsection*{סעיף ב}

נסמן $f(x)=\frac{\arctan x}{\sqrt{x^2+x}}$ חיוביות בקטע $(0,\infty)$.
\[
    \int_0^\infty f(x)dx =
    \int_0^1 f(x)dx + \int_1^\infty f(x)dx
\]
האינטגרל משמאל מתכנס אם ורק אם שני האינטגרלים מימין מתכנסים.
\\\\
נבדוק את התכנסות האינטגרל $\int_1^{\infty}f(x)dx$. \\
ניקח $g(x)=\frac{1}{x}$, אז:
\[
    \lim_{x\rightarrow\infty}\frac{f(x)}{g(x)} =
    \lim_{x\rightarrow\infty} \frac{x}(\sqrt{x^2+x}) \cdot \arctan x =
    \lim_{x\rightarrow\infty} \sqrt{\frac{x^2}{x^2+x}} \cdot \lim_{x\rightarrow\infty} \arctan x =
    1 \cdot \frac{\pi}{2} = \frac{\pi}{2} > 0
\]
האינטגרל $\int_1^\infty g(x)dx=\int_1^\infty \frac{dx}{x}$
מתבדר לפי למה $3.2$, לכן לפי מבחן ההשוואה $*3.16$
גם $\int_1^\infty f(x)dx$ מתבדר.
\\\\
מכאן נסיק כי $\int_0^\infty f(x)$ מתבדר.

\pagebreak

\section*{שאלה 2}

נסמן $f(x)=\frac{x^\alpha(1-x^2)^\beta}{1-\cos x}$.
עלינו לבדוק את ההתכנסות את האינטגרל $\int_0^1f(x)dx$. \\
נדגיש כי הפונקציה חיובית לכל אורך קטע האינטגרציה הנתון.

\[
    \int_0^1 f(x) dx = \int_0^{1/2}f(x)dx + \int_{1/2}^1f(x)dx
\]
האינטגרל משמאל מתכנס אם ורק אם שני האינטגרלים מימין מתכנסים. נבדוק את ההתכנסות של כל אחד מהם בנפרד.
\\\\
נבדוק את התכנסות האינטגרל $\int_0^{1/2}f(x)dx$. \\
נבחר $g(x) = \frac{x^\alpha}{x^2}=\frac{1}{x^{2-\alpha}}$. אז:
\[
    \lim_{x\rightarrow 0^+}\frac{f(x)}{g(x)} =
    \lim_{x\rightarrow 0^+}\frac{x^\alpha(1-x^2)^\beta \cdot x^2}{(1-\cos x)\cdot x^\alpha} =
    \lim_{x\rightarrow 0^+} (1-x^2)^\beta \cdot \lim_{x\rightarrow 0^+} \frac{x^2}{1-\cos x} \underrel{\text{לופיטל}}{=}
    1 \cdot \lim_{x\rightarrow 0^+} \frac{2x}{\sin x} = 2 > 0
\]
האינטגרל $\int_0^{1/2}g(x)dx=\int_0^{1/2}\frac{1}{x^{2-\alpha}}dx$
מתכנס אם ורק אם $2-\alpha < 1$,
כלומר $\alpha > 1$
(לפי למה $3.2$). \\
לכן לפי מבחן ההשוואה $3.5*$
גם האינטגרל $\int_0^{1/2}f(x)dx$
מתכנס אם ורק אם $\alpha > 1$.
\\\\
נמשיך ונבדוק את התכנסות האינטרל $\int_{1/2}^1f(x)dx$. \\
נבחר, הפעם, $g(x)=(1-x)^\beta=\frac{1}{(1-x)^{-\beta}}$. אז:
\[
    \lim_{x\rightarrow 1^-} \frac{f(x)}{g(x)} =
    \lim_{x\rightarrow 1^-} \frac{x^\alpha (1-x^2)^\beta}{(1-\cos x)\cdot (1-x)^\beta} =
    \lim_{x\rightarrow 1^-} \frac{x^\alpha}{1-\cos x} \cdot \lim_{x\rightarrow 1^-} \frac{((1-x)(1+x))^\beta}{(1-x)^\beta} =
\]
\[
    \frac{1}{1-\cos 1} \cdot \lim_{x\rightarrow 1^-} (1+x)^\beta =
    \frac{2^\beta}{1-\cos 1} > 0
\]
האינטגרל $\int_{1/2}^1g(x)dx = \int_{1/2}^1\frac{1}{(1-x)^{-\beta}}$
מתכנס, לפי שאלה 5 ביחידה 3, אם ורק אם $-\beta < 1$, כלומר $\beta > -1$ \\
לכן לפי מבחן ההשוואה $3.5*$
גם האינטגרל $\int_{1/2}^1f(x)dx$
מתכנס אם ורק אם $\beta > -1$.
\\\\
לסיכום נקבל שהאינטגרל $\int_0^1f(x)dx$
מתכנס אם ורק אם $\alpha > 1\wedge \beta > -1$

\section*{שאלה 3}

\subsection*{סעיף א}

הטענה נכונה. נוכיח את נכונותה ישירות מהגדרת הגבול. \\
תהא, אם כן, $f$
אינטגרבילית ב$[0,t]$
לכל $t>0$
כך ש $\int_0^\infty f(x)dx$ מתכנס,
ותהא $g$ פונקציה המקיימת $g(x)\geq x$ לכל $x\geq 0$.
\\\\
יהא $\epsilon >0$.
עלינו למצוא $M$
ממשי כך שלכל $x>M$ מתקיים $\left| \int_x^{g(x)}f(x)dx \right| < \epsilon$. \\
לפי מבחן קושי (משפט $3.15$),
עבור $\epsilon$ יש $M_0>0$
כך שלכל שני מספרים $r,s\in(M_0, \infty)$ נקבל $\left| \int_r^sf(x)dx \right| < \epsilon$. \\
נבחר $M=M_0$. נקבל אפוא כי לכל $x>M$ מתקיים, לפי הנתון, $g(x)\geq x > M$. \\
לכן $x, g(x)\in (M_0, \infty)$ מקיימים $\left| \int_x^{g(x)}f(x)dx \right| < \epsilon$
ובכך סיימנו את ההוכחה.

\subsection*{סעיף ב}

הטענה שגויה. נציג דוגמה נגדית. \\
נבחר $f(x)=-x$. ברור כי $f$ רציפה בקטע $[1, \infty)$,
וכן בקטע זה נקבל $f(x) \leq -1 < 0 < \frac{1}{x^2}$ כנדרש.
\\\\
האינטגרל $\int_1^{\infty} f(x)dx$ מתכנס אם ורק אם מתכנס האינטגרל $\int_1^{\infty}-f(x)dx$,
ובמקרה זה נקבל $\int_1^{\infty}f(x)dx=-\int_1^{\infty}-f(x)dx$. \\
אבל האינטגרל $\int_1^{\infty}-f(x)dx=\int_1^{\infty}xdx=\int_1^{\infty} \frac{1}{x^{-1}} dx$
לא מתכנס (כי $-1<1$)
ולכן גם האינטגרל $\int_1^{\infty}f(x)dx$ לא מתכנס.

\pagebreak

\section*{שאלה 4}

\subsection*{סעיף א}

הטענה לא שגויה. ניקח למשל $f$ כך שלכל $x\leq 1$:
\[
    f(x) = \begin{cases}
        0 & x\notin \mathbb{Z} \\
        x & x\in \mathbb{Z}
    \end{cases}
\]
בסעיף 3.2.6 (עמוד 38 בכרך ב)
מוכיחים בעזרת למה 1.25 כי $\int_1^\infty f(x)dx$
אכן מתכנס.
\\\\
אבל $\lim_{x\rightarrow\infty}\frac{f(x)}{x}\ne 0$,
כי עבור $\epsilon=1$ עבור כל $x$ בתחום ההגדרה ניקח את $\lfloor x \rfloor + 1 > x$ ונקבל:
\[
    \frac{f(\lfloor x \rfloor + 1)}{\lfloor x \rfloor + 1} \underrel{הארגומנט שלם}{=}
    \frac{\lfloor x \rfloor + 1}{\lfloor x \rfloor + 1} =
    1 \geq \epsilon
\]

\subsection*{סעיף ב}

הפונקציה $f^2$
היא פונקציה חיובית, ולכן נרצה להשתמש במבחן ההשוואה על מנת להוכיח כי
האינטגרל $\int_1^\infty f^2(x)dx$
מתכנס.
\\\\
ניקח $g(x)=|f(x)|+f^4(x)$.
ברור כי $g$ פונקציה חיובית ורציפה ב $[1,\infty)$ כסכום של שתי פונקציות חיוביות ורציפות. \\
מכאן נסיק, לפי 1.18, כי $g$ אינטגרבילית בקטע $[1,t]$ לכל $t\geq 1$. \\
לפי הנתון האינטגרל $\int_1^\infty |f(x)|dx$
והאינטגרל $\int_1^\infty f^4(x)dx$ מתכנסים.\\
לכן, הגבול
\[
    \lim_{t\rightarrow \infty}\int_1^t g(x)dx \underrel{1.24}{=}
    \lim_{t\rightarrow \infty}\int_1^t |f(x)|dx + \lim_{t\rightarrow \infty}\int_1^t f^4(x)dx
\]
קיים וסופי כסכום של של מספרים סופיים, ומכאן $\int_1^\infty g(x)dx$ מתכנס.
\\\\
כעת, נרצה להוכיח כי יש $A\geq 1$
כלשהו כך ש $g(x)\geq f^2(x)$
לכל $x\in[A,\infty)$.
ניקח $A=1$. לכל $x\in[1,\infty)$,

\begin{enumerate}[I]
    \item אילו $|f(x)|<1$,
          אז $f^2(x)<|f(x)|\leq |f(x)|+f^4(x)=g(x)$
    \item אחרת, $|f(x)|\geq 1$
          ונקבל $f^2(x)\leq|f(x)|^3\leq f^4(x)\leq |f(x)|+f^4(x)=g(x)$
\end{enumerate}
לכן, לפי מבחן ההשוואה $3.16$,
האינטגרל $\int_1^\infty f^2(x)dx$
מתכנס וסיימנו.

\pagebreak

\section*{שאלה 5}

ראשית, הפונקציה
$f(x)\sin(e^x)$
רציפה בקטע $[a,\infty)$
כמכפלה והרכבה של פונקציות רציפות. \\
בפרט, הפונקציה רציפה (ואינטגרבילית לפי $1.18$) בתת-הקטע
$[a,t]$, וכן בכל תת-קטע $[t,s]$ כך ש $s \geq t$. \\
לכן, לפי תכונת האדיטיביות המוכללת,
\[
    \int_a^\infty f(x)\sin(e^x)dx \underrel{1.23}{=}
    \int_a^t f(x)\sin(e^x)dx + \int_t^\infty f(x)\sin(e^x)dx
\]
האינטגרל $\int_a^t f(x)\sin(e^x)dx$
הוא אינטגרל מסוים - הפונקציה אינטגבילית בקטע סגור זה. \\
מכאן נסיק כי האינטגרל $\int_a^\infty f(x)\sin(e^x)dx$
מתכנס אם רק אם האינטגרל $\int_t^\infty f(x)\sin(e^x)dx$ מתכנס. \\
משיקולים דומים, $\int_a^\infty |f(x)\sin(e^x)|dx$ מתכנס אם ורק אם $\int_t^\infty |f(x)\sin(e^x)|dx$ מתכנס. \\
וכן $\int_t^\infty f(x)dx$ מתבדר כי לפי הנתון $\int_a^\infty f(x)dx$ מתבדר.
\\\\
כעת, נעבור לעסוק בשאלת ההתכנסות של האינטגרל $\int_t^\infty f(x)\sin(e^x)dx$.
מתקיים:
\[
    \int_t^\infty f(x)\sin(e^x)dx =
    \begin{bmatrix}
        u=e^x                                                    \\
        \ln u = x           & \Rightarrow & \frac{du}{u} = dx    \\
        x = t               & \Rightarrow & u = e^t              \\
        x\rightarrow \infty & \Rightarrow & u \rightarrow \infty
    \end{bmatrix} =
    \int_{e^t}^\infty \frac{f(\ln u)}{u} \cdot \sin u \;du
\]
נרצה להשתמש במבחן דיריכלה להוכיח את התכנסות אינטגרל זה. לשם כך נסמן $\xi(u) = \frac{f(\ln u)}{u}$.
\begin{enumerate}[I]
    \item נראה כי $\xi$ יורדת וגזירה ברציפות בקטע $[e^t, \infty)$. \\
          הפונקציה גזירה בקטע כמנה והרכבה של פונקציות גזירות כאשר מובטח לנו $u>0$ מההצבה $u=e^x$. נקבל אפוא:
          \[
              \xi'(x) = [\frac{f(\ln u)}{u}]' =
              \frac{f'(\ln u)\cdot \frac{1}{u} \cdot u - f(\ln u)}{u^2} =
              \frac{f'(\ln u) - f(\ln u)}{u^2}
          \]
          לכל $u\in[e^t, \infty)$ יש $x\in [t,\infty)$ כך ש $u=e^x$. \\
          נקבל, לפי הנתון עבור $x\geq t$, כי $f'(\ln u)-f(\ln u)=f'(x)-f(x)<0$. \\
          לכן $\xi'(u)<0$ והפונקציה יורדת בקטע $[e^t, \infty)$. \\
          פוקנציית הנגזרת $\xi'$ רציפה כהפרש, מנה והרכבה של פונקציות רציפות כאשר $u>0$ מובטח.
    \item מהמשפט האנלוגי "חסומה כפול אפסה" עבור $f(\ln u)$ חסומה לפי הנתון ו $\frac{1}{u}$ אפסה,
          נסיק $\xi(u)=\frac{f(\ln u)}{u}\xrightarrow[x\rightarrow \infty]{}0$
    \item כמו כן, מתקיים
          \[
              |G(x)| = |\int_{e^t}^x \sin u \; du| = |\cos x - \cos e^t| \leq 2
          \]
\end{enumerate}
הראינו כי מתקיימים תנאי מבחן דיריכלה ולכן לפי $3.19$ האינטגרל הנ"ל מתכנס.
\\\\
נראה כי האינטגרל מתכנס בתנאי, כלומר $\int_{e^t}^\infty |\xi(u)\sin u| du$ מתבדר. \\
תחילה, נשים לב כי $\xi$ חיובית בקטע $[t,\infty)$
(כי היא יורדת ברציפות ושואפת לאפס), ולכן $\int_{e^t}^\infty |\xi(u)\sin u| du = \int_{e^t}^\infty \xi(u)|\sin u| du$ \\
ידוע כי $|\sin u|\leq 1$ ולכן $|\sin u| \geq \sin^2u$.
אי לכך, לפי מבחן ההשוואה $3.16$, אילו $\int_{e^t}^\infty \xi(u)\sin^2 u \; du$ מתבדר
אז $\int_{e^t}^\infty \xi(u)|\sin u| du$.
\\\\
נניח בשלילה כי $\int_{e^t}^\infty \xi(u)\sin^2 u \; du$ מתכנס. \\
מהזהות הטריגונומטרית $\cos 2u = 1 - 2\sin^2u$ נסיק:
\[
    \int_{e^t}^\infty \xi(u)\sin^2 u \; du =
    \frac{1}{2}\int_{e^t}^\infty \xi(u) du - \frac{1}{2} \int_{e^t}^\infty \xi(u)\cos 2u \; du
\]
האינטגרל $\int_{e^t}^\infty \xi(u)\cos 2u \; du$ מתכנס לפי מבחן דיריכלה משיקולים דומים לחלק הקודם של ההוכחה.\\
לכן, מהשוויון שלעיל, נקבל כי גם האינטגרל $\int_{e^t}^\infty \xi(u) du$ מתכנס
וערכו הוא $2\int_{e^t}^\infty \xi(u)\sin^2 u \; du + \int_{e^t}^\infty \xi(u)\cos 2u \; du$.
\[
    \int_{e^t}^\infty \xi(u)du =
    \int_{e^t}^\infty \frac{f(\ln u)}{u} du =
    \begin{bmatrix}
        x = \ln u                       & dx = \frac{du}{u}    \\
        u = e^t \Rightarrow             & x = t                \\
        u\rightarrow \infty \Rightarrow & x \rightarrow \infty
    \end{bmatrix} =
    \int_t^\infty f(x)dx
\]
ולכן $\int_t^\infty f(x)dx$ מתכנס,
אבל בתחילת ההוכחה הראינו שהוא מתבדר וזו סתירה! מכאן נסיק כי האינטגרל מתכנס בתנאי.

\end{document}